% ============================================
% Chapter 00: Defense Strategy / 答辩策略
% SOURCE: Jan_Koller_Opponent_Research folder
% ============================================
\section{答辩策略分析 / Defense Strategy Analysis}

% ============================================
% Section 1: The Golden Rule
% ============================================
\begin{warnbox}[黄金法则 / Golden Rule]
    \textbf{老师并不想挂你。} (Teachers don't want to fail you.)
    他们问基础题是为了确认你具备基本的工程师素养。
    如果你遇到难题卡住了,\textbf{立刻}把话题引导到你熟悉的这些"保命题"上。
\end{warnbox}

% ============================================
% Section 2: SZZ Committee FAQ (Life-Saving Questions)
% ============================================
\section{国考委员会常问基础题 / SZZ Committee FAQ}
\textit{Note: These are "Safe Questions" (保命题). Memorize these simple definitions.}

% --------------------------------------------
% 2.1 Mathematics
% --------------------------------------------
\subsection{数学 / Mathematics}
\begin{studybox}{导数与积分 / Derivative \& Integral}
    \textbf{[CN]}: 老师可能会问:“什么是导数?”
    \textbf{回答}: 
    \begin{itemize}[leftmargin=*]
        \item \textbf{导数 (Derivative)}: 是\textbf{变化率} (Rate of Change)。例如:速度是位置的导数 ($v = dx/dt$)。
        \item \textbf{积分 (Integral)}: 是\textbf{累积量} (Accumulation) 或曲线下的面积。例如:距离是速度的积分。
    \end{itemize}
    \tcblower
    \textbf{[EN]}: 
    \begin{itemize}[leftmargin=*]
        \item \textbf{Derivative}: Represents the \textbf{Rate of Change}. (e.g., Velocity is the derivative of Position).
        \item \textbf{Integral}: Represents \textbf{Accumulation} or Area under the curve. (e.g., Distance is the integral of Velocity).
    \end{itemize}
\end{studybox}

\begin{studybox}{线性代数 / Linear Algebra}
    \textbf{[CN]}: “矩阵是什么?”
    \textbf{回答}: 矩阵是一个\textbf{线性变换} (Linear Transformation)。它可以表示旋转、缩放或平移。特征值 (Eigenvalue) 表示变换中方向不变的向量的缩放比例。
    \tcblower
    \textbf{[EN]}: A Matrix represents a \textbf{Linear Transformation} (Rotation, Scaling, Translation). Eigenvalues represent the scaling factor of vectors that do not change direction.
\end{studybox}

% --------------------------------------------
% 2.2 Microcontrollers (MIK)
% --------------------------------------------
\subsection{微控制器 / Microcontrollers}
\begin{studybox}{哈佛 vs 冯诺依曼 / Harvard vs Von Neumann}
    \textbf{[CN]}: 这是 MIK 必问的基础题。
    \textbf{回答}: 
    \begin{itemize}[leftmargin=*]
        \item \textbf{冯诺依曼 (Von Neumann)}: 指令和数据共享\textbf{同一个}存储器和总线。瓶颈在于无法同时读写。(如 PC x86)。
        \item \textbf{哈佛 (Harvard)}: 指令和数据拥有\textbf{物理分离}的存储器和总线。速度更快。ESP32 使用哈佛架构。
    \end{itemize}
    \tcblower
    \textbf{[EN]}: 
    \begin{itemize}[leftmargin=*]
        \item \textbf{Von Neumann}: Instructions and Data share the \textbf{SAME} memory and bus. (Bottleneck: Cannot fetch and write simultaneously).
        \item \textbf{Harvard}: Instructions and Data have \textbf{SEPARATE} memories and buses. (Faster). \textbf{ESP32 uses Harvard architecture}.
    \end{itemize}
\end{studybox}

\begin{studybox}{中断 / Interrupt vs Polling}
    \textbf{[CN]}: 
    \begin{itemize}[leftmargin=*]
        \item \textbf{轮询 (Polling)}: CPU 既然不断检查外设状态(浪费资源)。
        \item \textbf{中断 (Interrupt)}: 外设主动通知 CPU。CPU 暂停当前任务,处理中断服务程序 (ISR),然后返回。
    \end{itemize}
    \tcblower
    \textbf{[EN]}: 
    \begin{itemize}[leftmargin=*]
        \item \textbf{Polling}: CPU constantly checks peripheral status (Wastes cycles).
        \item \textbf{Interrupt}: Peripheral signals the CPU. CPU pauses, executes ISR (Interrupt Service Routine), and resumes.
    \end{itemize}
\end{studybox}

% --------------------------------------------
% 2.3 Physics & Circuits
% --------------------------------------------
\subsection{物理与电路 / Physics \& Circuits}
\begin{studybox}{基本定律 / Laws of Physics}
    \textbf{[CN]}: 
    \begin{itemize}[leftmargin=*]
        \item \textbf{牛顿第二定律}: $F=ma$。力等于质量乘以加速度。这是加速度计的基础。
        \item \textbf{欧姆定律}: $V=IR$。电压等于电流乘以电阻。
        \item \textbf{基尔霍夫定律 (KCL/KVL)}: 电荷守恒 ($\sum I=0$) 和能量守恒 ($\sum V=0$)。
    \end{itemize}
    \tcblower
    \textbf{[EN]}: 
    \begin{itemize}[leftmargin=*]
        \item \textbf{Newton's 2nd Law}: $F=ma$. Force equals Mass times Acceleration. (Basis of Accelerometers).
        \item \textbf{Ohm's Law}: $V=IR$.
        \item \textbf{Kirchhoff's Laws}: Conservation of Charge (KCL) and Energy (KVL).
    \end{itemize}
\end{studybox}

% ============================================
% Section 3: Opponent Profile (Jan Koller)
% ============================================
\section{对手画像 / Opponent Profile: Jan Koller}

\begin{warnbox}[核心特征: 务实但挑剔]
Jan Koller 是一个\textbf{「务实但挑剔的学究」(Pragmatic Pedant)}。
\begin{itemize}[leftmargin=*]
    \item \textbf{痛恨形式错误}: ``非典型流程图'' (Atypical flowcharts) 是他的触发点。
    \item \textbf{关注单位}: 必须有物理单位 (Does 2.4 mean Volts or Amps?)。
    \item \textbf{你的评分}: 他给了你 \textbf{B (Very Good)}。只要回答好问题,稳 B 冲 A。
\end{itemize}
\end{warnbox}

% ============================================
% Section 4: THE Question He WILL Ask
% ============================================
\section{必答题: Acc + Gyro}

\begin{thesisbox}[Koller 的真题 (100\% 会问)]
\textbf{Q}: "Can you explain why a combination of acceleration and angular velocity is used in fall detection?"

\textbf{A (Concept)}:
\textbf{[CN]}: 单靠加速度计会有误报(如跳跃)。陀螺仪测量姿态变化。真正的跌倒 = 剧烈姿态变化 (Gyro) + 剧烈撞击 (Acc) + 静止。
\tcblower
\textbf{[EN]}: \textbf{Sensor Fusion}. Acceleration alone creates false positives (e.g., Jumping). Gyroscope measures orientation change. A Fall = Rotation (Gyro) + Impact (Acc) + Inactivity.
\end{thesisbox}
