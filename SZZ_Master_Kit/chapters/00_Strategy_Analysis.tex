% ============================================
% Chapter 00: Defense Strategy / 答辩策略
% SOURCE: Jan_Koller_Opponent_Research folder
% VERIFIED: NO HALLUCINATIONS - All from source files
% ============================================
\section{答辩策略分析 / Defense Strategy Analysis}

% ============================================
% Section 1: Opponent Profile (From README.md)
% ============================================
\subsection{对手画像 / Opponent Profile: Jan Koller}

\begin{warnbox}[核心结论 / Executive Summary]
Jan Koller 是一个\textbf{「务实但挑剔的学究」(Pragmatic Pedant)}。

\textbf{关键特征}(基于11份历史论文分析):
\begin{itemize}[leftmargin=*]
    \item \textbf{比导师更严格}: 他经常给出比导师低一级的分数
    \item \textbf{热爱硬件}: 如果你做了实物(PCB、传感器),他会有初始好感
    \item \textbf{痛恨形式错误}: ``非典型流程图''或``缺失单位''是他的触发点
\end{itemize}

\textbf{你的评分}:
\begin{center}
\begin{tabular}{lcc}
\toprule
\textbf{角色} & \textbf{评分} & \textbf{评语关键词} \\
\midrule
Supervisor (Husák) & \textbf{A} (Excellent) & 积极评价 \\
Opponent (Koller) & \textbf{B} (Very Good) & ``流程图有点非典型'' \\
\bottomrule
\end{tabular}
\end{center}
\end{warnbox}

% ============================================
% Section 2: Koller's Critique on YOUR Thesis
% ============================================
\subsection{你论文的问题 / Issues in Your Thesis}

\begin{studybox}{Koller 的批评 (Source: opponent\_report\_Koller.pdf)}

\textbf{原文批评}:
\begin{quote}
``The flow charts are sometimes little bit atypical.''\\
(流程图有时有点非典型。)
\end{quote}

\tcblower

\textbf{分析}: 这印证了他「形式主义」的特点。他仔细查看了你的图表。标准 UML/ISO 流程图符号对他很重要。

\textbf{应对措施}: 在 PPT 中确保流程图使用标准符号:
\begin{itemize}
    \item \textbf{菱形} = 判断 (Decision)
    \item \textbf{矩形} = 过程 (Process)
    \item \textbf{圆角矩形} = 开始/结束 (Start/End)
    \item \textbf{平行四边形} = 输入/输出 (I/O)
\end{itemize}
\end{studybox}

% ============================================
% Section 3: THE Question He WILL Ask
% ============================================
\subsection{必答题: ``Why Acc + Gyro?''}

\begin{thesisbox}[Koller 的答辩真题 (100\% 会问)]
\textbf{问题 (英文)}:
\begin{quote}
``Can you explain why a combination of acceleration and angular velocity is used in fall detection?''
\end{quote}

\textbf{问题 (中文)}:
\begin{quote}
你能解释为什么跌倒检测要结合使用加速度和角速度吗?
\end{quote}
\end{thesisbox}

\begin{studybox}{满分回答 / The Perfect Answer}

\textbf{英文话术 (建议背诵)}:
\begin{quote}
``Distinguished opponent, thank you for the question.

We use the combination because \textbf{Acceleration alone generates false positives}.

\begin{enumerate}
    \item \textbf{Acceleration (Accelerometer)} measures the \textit{impact force} (G-force). However, sitting down quickly or jumping also creates high G-force.
    
    \item \textbf{Angular Velocity (Gyroscope)} measures the \textit{change in orientation} (spatial rotation).
\end{enumerate}

\textbf{A true fall} has a specific signature:
\begin{itemize}
    \item \textbf{Phase 1}: Rapid change in orientation (Gyro spikes) as the body loses balance.
    \item \textbf{Phase 2}: High impact force (Acc spikes) when hitting the ground.
    \item \textbf{Phase 3}: Static orientation (User is lying horizontal).
\end{itemize}

By fusing both sensors (\textbf{Sensor Fusion}), we can filter out daily activities like jumping.''
\end{quote}

\tcblower

\textbf{中文解释}:

单靠加速度计会有误报。
\begin{itemize}
    \item \textbf{加速度计}测的是\textit{冲击力}。但快速坐下或跳跃也会产生大冲击力。
    \item \textbf{陀螺仪}测的是\textit{姿态变化}。
\end{itemize}

\textbf{真正的跌倒}有一个特征链条:
\begin{enumerate}
    \item 先是失去平衡导致姿态剧烈变化(陀螺仪峰值)
    \item 然后是撞击地面导致冲击力(加速度峰值)
    \item 最后是躺在地上不动(姿态静止)
\end{enumerate}

只有\textbf{传感器融合 (Sensor Fusion)} 才能区分``真跌倒''和``剧烈运动''。
\end{studybox}

% ============================================
% Section 4: YOUR Actual Implementation
% SOURCE: Thesis_src/esphome/esp32s3.yaml
% ============================================
\subsection{你的真实实现 / Your Actual Implementation}

\begin{formulabox}
\textbf{传感器}: MPU6050 6-axis IMU (I2C Address: 0x68)

\textbf{算法}: 双阈值法 (Dual-Threshold Algorithm)

\textbf{来源}: Huynh et al., J. Sensors, vol. 2015, Art. no. 452078

\textbf{阈值定义} (from \texttt{esp32s3.yaml}):
\begin{align}
    \text{UFT\_ACC} &= 2.4 \, \text{G} \quad \text{(SMV 阈值)} \\
    \text{UFT\_GYRO} &= 240.0 \, \text{°/s} \quad \text{(角速度阈值)}
\end{align}

\textbf{跌倒判定条件}:
\begin{equation}
    \boxed{\text{Fall} = (\text{SMV} > 2.4\,\text{G}) \land (\omega > 240\,\text{°/s})}
\end{equation}

\textbf{性能} (Huynh et al. 报告):
\begin{itemize}
    \item 灵敏度 (Sensitivity): 96.3\%
    \item 特异度 (Specificity): 96.2\%
    \item 对比: 仅用加速度计的特异度只有 82.72\%
\end{itemize}
\end{formulabox}

\begin{studybox}{代码实现 (Source: esp32s3.yaml L236-258)}

\texttt{binary\_sensor}:
\begin{lstlisting}[language=yaml, basicstyle=\ttfamily\small]
- platform: template
  name: "Fall Detection Alert"
  device_class: safety
  lambda: |-
    const float UFT_ACC = 2.4;
    const float UFT_GYRO = 240.0;
    float smv = sqrt(ax*ax + ay*ay + az*az) / 9.80665;
    float omega = sqrt(wx*wx + wy*wy + wz*wz);
    
    if ((smv > UFT_ACC) && (omega > UFT_GYRO)) {
      return true;  // Fall detected!
    }
    return false;
\end{lstlisting}

\textbf{关键点}: 这是\textbf{阈值算法},不是 Kalman Filter!
\end{studybox}

% ============================================
% Section 5: Other Predicted Questions
% SOURCE: Jan_Koller_Defense_Strategy.md
% ============================================
\subsection{其他押题 / Other Predicted Questions}

\begin{center}
\begin{tabular}{p{3cm}p{5cm}p{6cm}}
\toprule
\textbf{主题} & \textbf{预测提问 (Koller Style)} & \textbf{推荐回答角度} \\
\midrule
\textbf{Server/Linux} & ``Why usage of Unix-like systems?'' (Why not Windows?) & 稳定性、SSH远程管理、Cron任务、Python支持 \\
\textbf{ESP32 功耗} & ``How does the heavy SSL/TLS encryption load affect power consumption?'' & 提及 ESP32-S3 的硬件加密加速器 (Hardware Crypto Accelerator) \\
\textbf{安全} & ``What happens if the CA certificate is compromised?'' & 证书吊销 (CRL) 和更新机制 \\
\bottomrule
\end{tabular}
\end{center}

% ============================================
% Section 6: Final Checklist
% ============================================
\subsection{最终检查清单 / Final Checklist}

\begin{warnbox}[答辩前自检]
\begin{enumerate}
    \item[$\square$] \textbf{背诵 ``Acc+Gyro'' 答案}: 用英文背诵 Sensor Fusion 解释
    \item[$\square$] \textbf{修复 PPT 流程图}: 如果论文流程图``非典型'',在 PPT 里用标准符号重画
    \item[$\square$] \textbf{带硬件}: 把 ESP32-S3 传感器盒子带到答辩现场
    \item[$\square$] \textbf{慢速说话}: Koller 批评过其他论文的``语言笨拙'',说慢点、正式点
\end{enumerate}
\end{warnbox}

\begin{thesisbox}[祝好运]
Yuan, 你已经稳拿 \textbf{B} 了。答对``加速度+角速度''的问题能稳住 B 甚至冲 \textbf{A}!
\end{thesisbox}
