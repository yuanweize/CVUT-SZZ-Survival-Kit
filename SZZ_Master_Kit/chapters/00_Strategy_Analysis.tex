% ============================================
% Chapter 00: Defense Strategy / 答辩策略
% SOURCE: Jan_Koller_Opponent_Research folder
% PERSONALIZED: Based on Study_Results_Yuan_Weize.pdf
% ============================================

% ============================================
% MASTER PRIORITY MATRIX (Based on Transcript)
% ============================================
\section{SZZ Scope \& Priority Matrix / 国考范围与优先级总表}

\begin{warnbox}[\textbf{[ANALYSIS]} 个人定制分析 / Personalized Analysis]
\textbf{[CN]} 此表格基于你的成绩单 (Study\_Results\_Yuan\_Weize.pdf) 自动生成。\\
\textbf{[EN]} This table is auto-generated from your official transcript.\\[0.5em]
\textbf{总学分}: 204 | \textbf{加权平均}: 2.11 | \textbf{专业方向}: Power Engineering (EN1/EN2)
\end{warnbox}

\subsection{Compulsory Topics (必考 - 2 Questions) / 必修科目}

\begin{center}
\small
\begin{tabular}{|p{1.2cm}|p{3.8cm}|c|c|p{4.5cm}|}
\hline
\rowcolor{gray!30} \textbf{Topic} & \textbf{Course / 科目} & \textbf{Grade} & \textbf{Priority} & \textbf{Strategy / 策略} \\ \hline
\multicolumn{5}{|c|}{\cellcolor{blue!10}\textbf{MATHEMATICS 数学}} \\ \hline
1 & BE5B01LAL Linear Algebra & \textcolor{red}{\textbf{E}} & \textcolor{red}{\textbf{[CRITICAL]}} & \textbf{SURVIVAL}: 背公式,不推导 \\ \hline
2,3 & BE5B01DEN Numerical+DiffEq & \textcolor{red}{\textbf{E}} & \textcolor{red}{\textbf{[CRITICAL]}} & \textbf{SURVIVAL}: Newton-Raphson 记死 \\ \hline
4,5 & BE5B01DMG Discrete+Graph & \textcolor{red}{\textbf{E}} & \textcolor{red}{\textbf{[CRITICAL]}} & \textbf{SURVIVAL}: 集合论+图遍历 \\ \hline
6 & BE5B01PRS Probability & \textcolor{green}{\textbf{B}} & \textcolor{green}{\textbf{[SHOW OFF]}} & \textbf{ATTACK}: Bayes→传感器融合 \\ \hline
\multicolumn{5}{|c|}{\cellcolor{yellow!10}\textbf{PHYSICS 物理}} \\ \hline
7,8,9 & BE5B02PH1 Physics 1 & \textcolor{orange}{\textbf{D}} & \textcolor{orange}{\textbf{[CAUTION]}} & \textbf{FOCUS}: F=ma→加速度计原理 \\ \hline
10,11 & BE5B02PH2 Physics 2 & \textcolor{orange}{\textbf{D}} & \textcolor{orange}{\textbf{[CAUTION]}} & \textbf{FOCUS}: 热力学→ESP32散热 \\ \hline
\multicolumn{5}{|c|}{\cellcolor{green!10}\textbf{EECS CORE 电子信息核心}} \\ \hline
12 & BE5B31ZEO Circuits & \textcolor{orange}{\textbf{D}} & \textcolor{orange}{\textbf{[CAUTION]}} & \textbf{FOCUS}: KVL/KCL→传感器电路 \\ \hline
13,14 & BE5B34ELP Electron Devices & \textcolor{green}{\textbf{B}} & \textcolor{green}{\textbf{[SHOW OFF]}} & \textbf{ATTACK}: 半导体→你很熟练! \\ \hline
15 & BE5B34MIK Microcontrollers & \textcolor{green}{\textbf{A}} & \textcolor{green}{\textbf{[SHOW OFF]}} & \textbf{ATTACK}: ESP32是你的主场! \\ \hline
16 & BE5B33PRG/PGE Algorithms & \textcolor{blue}{\textbf{C/B}} & \textcolor{blue}{\textbf{[STABLE]}} & \textbf{HOLD}: 基础数据结构即可 \\ \hline
\end{tabular}
\end{center}

\subsection{Specialization Topic (选考 - 1 Question) / 专业选修}

\begin{center}
\small
\begin{tabular}{|p{1.2cm}|p{3.8cm}|c|c|p{4.5cm}|}
\hline
\rowcolor{gray!30} \textbf{Topic} & \textbf{Course / 科目} & \textbf{Grade} & \textbf{Status} & \textbf{Notes / 备注} \\ \hline
\multicolumn{5}{|c|}{\cellcolor{green!20}\textbf{[YOUR SPECIALIZATION] 你的专业方向}} \\ \hline
22 & BE5B15EN1/EN2 Power Eng. & \textcolor{blue}{\textbf{B/C}} & \textcolor{blue}{\textbf{[FOCUS]}} & \textbf{TARGET}: 电网+变压器 \\ \hline
18 & BE5B35LSP Logic Systems & \textcolor{blue}{\textbf{C}} & \textcolor{blue}{\textbf{[BACKUP]}} & 可作为备选专业题 \\ \hline
20 & BE5B14SP1 Elec. Machinery & \textcolor{red}{\textbf{E}} & \textcolor{red}{\textbf{[AVOID]}} & 成绩差,避免选此题 \\ \hline
21 & BE5B13MVE Materials & \textcolor{red}{\textbf{E}} & \textcolor{red}{\textbf{[AVOID]}} & 虽然修了但成绩差,避免 \\ \hline
\multicolumn{5}{|c|}{\cellcolor{red!10}\textbf{[CRITICAL] SURVIVAL MODE 求生模式}} \\ \hline
23 & BE5B33KUI AI/Cybernetics & \textcolor{red}{\textbf{E}} & \textcolor{red}{\textbf{[CRITICAL]}} & \textbf{SURVIVAL}: BFS/DFS+状态空间 \\ \hline
\multicolumn{5}{|c|}{\cellcolor{gray!20}\textbf{[SKIP] OUT OF SCOPE 不在考试范围}} \\ \hline
17 & BE5B17EMT EM Theory & \textcolor{gray}{---} & \textcolor{gray}{SKIP} & 未修课,跳过 \\ \hline
19 & BE5B31TES Signals & \textcolor{gray}{---} & \textcolor{gray}{SKIP} & 未修课,跳过 \\ \hline
24 & BE5B35APO Architecture & \textcolor{gray}{---} & \textcolor{gray}{SKIP} & 未完成考试,跳过 \\ \hline
25 & BE5B35ARI Control & \textcolor{gray}{---} & \textcolor{gray}{SKIP} & 未完成考试,跳过 \\ \hline
\end{tabular}
\end{center}

\subsection{20-Hour Time Allocation / 20小时时间分配}

\begin{formulabox}[\textbf{[TIME]} Recommended Study Plan / 推荐学习计划]
\begin{center}
\begin{tabular}{|l|c|l|}
\hline
\rowcolor{gray!20} \textbf{Category} & \textbf{Hours} & \textbf{Focus} \\ \hline
\textcolor{red}{[CRITICAL]} Math (E grades) & 5h & LAL/DEN/DMG 公式背诵 \\ \hline
\textcolor{red}{[CRITICAL]} KUI (E grade) & 1h & BFS/DFS/A*状态搜索 \\ \hline
\textcolor{orange}{[CAUTION]} Physics (D grades) & 4h & 力学+热力学基础 \\ \hline
\textcolor{orange}{[CAUTION]} Circuits (D grade) & 2h & KVL/KCL 计算练习 \\ \hline
\textcolor{green}{[SHOW OFF]} MIK(A)/ELP(B) & 2h & 论文桥接准备 \\ \hline
\textcolor{blue}{[FOCUS]} Power Eng. (专业) & 4h & EN1/EN2 电网重点 \\ \hline
\textbf{[THESIS]} Thesis Defense & 2h & Koller 问题演练 \\ \hline
\end{tabular}
\end{center}
\end{formulabox}

% ============================================
% Original Strategy Content Below
% ============================================
\section{答辩策略分析 / Defense Strategy Analysis}

% ============================================
% Section 1: The Golden Rule
% ============================================
\begin{warnbox}[黄金法则 / Golden Rule]
    \textbf{老师并不想挂你。} (Teachers don't want to fail you.)
    他们问基础题是为了确认你具备基本的工程师素养。
    如果你遇到难题卡住了,\textbf{立刻}把话题引导到你熟悉的这些"保命题"上。
\end{warnbox}

% ============================================
% Section 2: SZZ Committee FAQ (Life-Saving Questions)
% ============================================
\section{国考委员会常问基础题 / SZZ Committee FAQ}
\textit{Note: These are "Safe Questions" (保命题). Memorize these simple definitions.}

% --------------------------------------------
% 2.1 Mathematics
% --------------------------------------------
\subsection{数学 / Mathematics}
\begin{studybox}{导数与积分 / Derivative \& Integral}
    \textbf{[CN]}: 老师可能会问:“什么是导数?”
    \textbf{回答}: 
    \begin{itemize}[leftmargin=*]
        \item \textbf{导数 (Derivative)}: 是\textbf{变化率} (Rate of Change)。例如:速度是位置的导数 ($v = dx/dt$)。
        \item \textbf{积分 (Integral)}: 是\textbf{累积量} (Accumulation) 或曲线下的面积。例如:距离是速度的积分。
    \end{itemize}
    \tcblower
    \textbf{[EN]}: 
    \begin{itemize}[leftmargin=*]
        \item \textbf{Derivative}: Represents the \textbf{Rate of Change}. (e.g., Velocity is the derivative of Position).
        \item \textbf{Integral}: Represents \textbf{Accumulation} or Area under the curve. (e.g., Distance is the integral of Velocity).
    \end{itemize}
\end{studybox}

\begin{studybox}{线性代数 / Linear Algebra}
    \textbf{[CN]}: “矩阵是什么?”
    \textbf{回答}: 矩阵是一个\textbf{线性变换} (Linear Transformation)。它可以表示旋转、缩放或平移。特征值 (Eigenvalue) 表示变换中方向不变的向量的缩放比例。
    \tcblower
    \textbf{[EN]}: A Matrix represents a \textbf{Linear Transformation} (Rotation, Scaling, Translation). Eigenvalues represent the scaling factor of vectors that do not change direction.
\end{studybox}

% --------------------------------------------
% 2.2 Microcontrollers (MIK)
% --------------------------------------------
\subsection{微控制器 / Microcontrollers}
\begin{studybox}{哈佛 vs 冯诺依曼 / Harvard vs Von Neumann}
    \textbf{[CN]}: 这是 MIK 必问的基础题。
    \textbf{回答}: 
    \begin{itemize}[leftmargin=*]
        \item \textbf{冯诺依曼 (Von Neumann)}: 指令和数据共享\textbf{同一个}存储器和总线。瓶颈在于无法同时读写。(如 PC x86)。
        \item \textbf{哈佛 (Harvard)}: 指令和数据拥有\textbf{物理分离}的存储器和总线。速度更快。ESP32 使用哈佛架构。
    \end{itemize}
    \tcblower
    \textbf{[EN]}: 
    \begin{itemize}[leftmargin=*]
        \item \textbf{Von Neumann}: Instructions and Data share the \textbf{SAME} memory and bus. (Bottleneck: Cannot fetch and write simultaneously).
        \item \textbf{Harvard}: Instructions and Data have \textbf{SEPARATE} memories and buses. (Faster). \textbf{ESP32 uses Harvard architecture}.
    \end{itemize}
\end{studybox}

\begin{studybox}{中断 / Interrupt vs Polling}
    \textbf{[CN]}: 
    \begin{itemize}[leftmargin=*]
        \item \textbf{轮询 (Polling)}: CPU 既然不断检查外设状态(浪费资源)。
        \item \textbf{中断 (Interrupt)}: 外设主动通知 CPU。CPU 暂停当前任务,处理中断服务程序 (ISR),然后返回。
    \end{itemize}
    \tcblower
    \textbf{[EN]}: 
    \begin{itemize}[leftmargin=*]
        \item \textbf{Polling}: CPU constantly checks peripheral status (Wastes cycles).
        \item \textbf{Interrupt}: Peripheral signals the CPU. CPU pauses, executes ISR (Interrupt Service Routine), and resumes.
    \end{itemize}
\end{studybox}

% --------------------------------------------
% 2.3 Physics & Circuits
% --------------------------------------------
\subsection{物理与电路 / Physics \& Circuits}
\begin{studybox}{基本定律 / Laws of Physics}
    \textbf{[CN]}: 
    \begin{itemize}[leftmargin=*]
        \item \textbf{牛顿第二定律}: $F=ma$。力等于质量乘以加速度。这是加速度计的基础。
        \item \textbf{欧姆定律}: $V=IR$。电压等于电流乘以电阻。
        \item \textbf{基尔霍夫定律 (KCL/KVL)}: 电荷守恒 ($\sum I=0$) 和能量守恒 ($\sum V=0$)。
    \end{itemize}
    \tcblower
    \textbf{[EN]}: 
    \begin{itemize}[leftmargin=*]
        \item \textbf{Newton's 2nd Law}: $F=ma$. Force equals Mass times Acceleration. (Basis of Accelerometers).
        \item \textbf{Ohm's Law}: $V=IR$.
        \item \textbf{Kirchhoff's Laws}: Conservation of Charge (KCL) and Energy (KVL).
    \end{itemize}
\end{studybox}

% ============================================
% Section 3: Opponent Profile (Jan Koller)
% ============================================
\section{对手画像 / Opponent Profile: Jan Koller}

\begin{warnbox}[核心特征: 务实但挑剔]
Jan Koller 是一个\textbf{「务实但挑剔的学究」(Pragmatic Pedant)}。
\begin{itemize}[leftmargin=*]
    \item \textbf{痛恨形式错误}: ``非典型流程图'' (Atypical flowcharts) 是他的触发点。
    \item \textbf{关注单位}: 必须有物理单位 (Does 2.4 mean Volts or Amps?)。
    \item \textbf{你的评分}: 他给了你 \textbf{B (Very Good)}。只要回答好问题,稳 B 冲 A。
\end{itemize}
\end{warnbox}

% ============================================
% Section 4: THE Question He WILL Ask
% ============================================
\section{必答题: Acc + Gyro}

\begin{thesisbox}[Koller 的真题 (100\% 会问)]
\textbf{Q}: "Can you explain why a combination of acceleration and angular velocity is used in fall detection?"

\textbf{A (Concept)}:
\textbf{[CN]}: 单靠加速度计会有误报(如跳跃)。陀螺仪测量姿态变化。真正的跌倒 = 剧烈姿态变化 (Gyro) + 剧烈撞击 (Acc) + 静止。
\tcblower
\textbf{[EN]}: \textbf{Sensor Fusion}. Acceleration alone creates false positives (e.g., Jumping). Gyroscope measures orientation change. A Fall = Rotation (Gyro) + Impact (Acc) + Inactivity.
\end{thesisbox}
