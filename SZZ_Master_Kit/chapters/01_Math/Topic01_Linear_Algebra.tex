% ============================================
% Topic 01: Linear Algebra / 线性代数
% Course: BE5B01LAG
% ============================================
\section{线性代数 / Linear Algebra (BE5B01LAG)}

% --------------------------------------------
% 1. Matrices and Systems
% --------------------------------------------
\begin{studybox}{矩阵与方程组 / Matrices \& Linear Systems}
    \textbf{[CN] 定义}: 
    \begin{itemize}[leftmargin=*]
        \item \textbf{线性方程组}: 可以用矩阵方程 $A\mathbf{x}=\mathbf{b}$ 表示。
        \item \textbf{秩 (Rank)}: 矩阵中线性无关的行/列的最大数目。决定方程组解的情况。
        \item \textbf{高斯消元法}: 将矩阵转化为行阶梯形矩阵以求解的方法。
    \end{itemize}
    \tcblower
    \textbf{[EN] Definition}: 
    \begin{itemize}[leftmargin=*]
        \item \textbf{Linear System}: Represented as $A\mathbf{x}=\mathbf{b}$.
        \item \textbf{Rank}: Max number of linearly independent rows/cols. Determines solution existence (Frobenius Theorem).
        \item \textbf{Gaussian Elimination}: Method to convert matrix to Row Echelon Form (REF).
    \end{itemize}
\end{studybox}

% --------------------------------------------
% 2. Eigenvalues
% --------------------------------------------
\begin{studybox}{特征值与特征向量 / Eigenvalues \& Eigenvectors}
    \textbf{[CN] 定义}: 
    对于方阵 $A$,如果存在非零向量 $\mathbf{v}$ 和标量 $\lambda$,使得 $A\mathbf{v} = \lambda \mathbf{v}$,则 $\lambda$ 是特征值,$\mathbf{v}$ 是特征向量。
    几何意义:矩阵变换后方向不变,只改变长度。
    \tcblower
    \textbf{[EN] Definition}: 
    For square matrix $A$, if $A\mathbf{v} = \lambda \mathbf{v}$ ($\mathbf{v} \neq 0$), then $\lambda$ is the Eigenvalue and $\mathbf{v}$ is the Eigenvector.
    Geometric: The vector direction remains invariant under transformation $A$.
\end{studybox}

\begin{formulabox}
\textbf{特征方程 / Characteristic Equation}:
\begin{equation}
    \det(A - \lambda I) = 0
\end{equation}
\textbf{行列式 (Determinant)}: $\det(A) = \prod \lambda_i$ (特征值之积).
\textbf{迹 (Trace)}: $\text{tr}(A) = \sum \lambda_i$ (特征值之和).
\end{formulabox}

% --------------------------------------------
% Thesis Connection
% --------------------------------------------
\begin{thesisbox}[论文关联 / Project Application]
    \textbf{[CN]}: 
    \begin{itemize}
        \item \textbf{传感器融合}: 卡尔曼滤波 (Kalman Filter) 严重依赖矩阵运算 (协方差矩阵 $P$ 的更新和求逆)。虽然你使用的是互补滤波,但理解矩阵是理解 KF 的基础。
        \item \textbf{旋转矩阵}: 描述 IMU 姿态 (Roll/Pitch/Yaw) 时使用 3x3 旋转矩阵或四元数。
        \item \textbf{PCA}: 主成分分析用于数据降维,本质上是求协方差矩阵的特征值。
    \end{itemize}

    \tcblower

    \textbf{[EN]}: 
    \begin{itemize}
        \item \textbf{Sensor Fusion}: Kalman Filtering relies on Matrix Algebra (Covariance Matrix $P$ inversion). 
        \item \textbf{Rotation}: 3x3 Rotation Matrices describe IMU orientation in 3D space.
        \item \textbf{PCA}: Principal Component Analysis (Data reduction) is essentially finding Eigenvalues of the covariance matrix.
    \end{itemize}
\end{thesisbox}
