% ============================================
% Topic 01: Linear Algebra / 线性代数
% ============================================
\section{线性代数 / Linear Algebra}

% --------------------------------------------
% 1. Matrix Basics
% --------------------------------------------
\begin{studybox}{矩阵基础 (Matrix Basics)}

\textbf{概念 (CN)}: 矩阵的定义、运算和性质

\textbf{Term (EN)}: Matrix, Transpose, Inverse, Determinant

\tcblower

\textbf{基本运算}:
\begin{itemize}[leftmargin=*]
    \item \textbf{加法}: $\mathbf{C} = \mathbf{A} + \mathbf{B}$ (同尺寸矩阵对应元素相加)
    \item \textbf{乘法}: $(\mathbf{AB})_{ij} = \sum_k a_{ik} b_{kj}$ (行乘列)
    \item \textbf{转置}: $(\mathbf{A}^T)_{ij} = a_{ji}$ (行列互换)
\end{itemize}

\textbf{Key Insight}: Matrix multiplication is NOT commutative: $\mathbf{AB} \neq \mathbf{BA}$ in general.

\end{studybox}

\begin{formulabox}
\textbf{行列式 (2×2)}:
\begin{equation}
    \det\begin{pmatrix} a & b \\ c & d \end{pmatrix} = ad - bc
\end{equation}

\textbf{行列式 (3×3, 展开)}:
\begin{equation}
    \det(\mathbf{A}) = a_{11}C_{11} + a_{12}C_{12} + a_{13}C_{13}
\end{equation}
其中 $C_{ij}$ 是代数余子式。

\textbf{逆矩阵}:
\begin{equation}
    \mathbf{A}^{-1} = \frac{1}{\det(\mathbf{A})} \text{adj}(\mathbf{A})
\end{equation}

\textbf{逆矩阵存在条件}: $\det(\mathbf{A}) \neq 0$(非奇异矩阵)
\end{formulabox}

% --------------------------------------------
% 2. Systems of Linear Equations
% --------------------------------------------
\begin{studybox}{线性方程组 (Systems of Linear Equations)}

\textbf{概念 (CN)}: 求解 $\mathbf{Ax} = \mathbf{b}$ 的方法

\textbf{Term (EN)}: Gaussian Elimination, Cramer's Rule, LU Decomposition

\tcblower

\textbf{矩阵形式}:
\begin{equation}
    \mathbf{Ax} = \mathbf{b} \quad \Rightarrow \quad \mathbf{x} = \mathbf{A}^{-1}\mathbf{b}
\end{equation}

\textbf{克莱默法则} (Cramer's Rule):
\begin{equation}
    x_i = \frac{\det(\mathbf{A}_i)}{\det(\mathbf{A})}
\end{equation}
其中 $\mathbf{A}_i$ 是将 $\mathbf{A}$ 的第 $i$ 列替换为 $\mathbf{b}$。

\textbf{Key Insight}: Gaussian elimination is more efficient than Cramer's rule for large systems.

\end{studybox}

\begin{formulabox}
\textbf{高斯消元示例}:

原方程组:
\begin{equation}
    \begin{cases}
        2x + y = 5 \\
        4x + 3y = 11
    \end{cases}
\end{equation}

增广矩阵:
\begin{equation}
    \left(\begin{array}{cc|c}
        2 & 1 & 5 \\
        4 & 3 & 11
    \end{array}\right) 
    \xrightarrow{R_2 - 2R_1}
    \left(\begin{array}{cc|c}
        2 & 1 & 5 \\
        0 & 1 & 1
    \end{array}\right)
\end{equation}

回代: $y = 1$, $x = 2$
\end{formulabox}

% --------------------------------------------
% 3. Eigenvalues and Eigenvectors
% --------------------------------------------
\begin{studybox}{特征值与特征向量 (Eigenvalues \& Eigenvectors)}

\textbf{概念 (CN)}: 满足 $\mathbf{Av} = \lambda\mathbf{v}$ 的标量 $\lambda$ 和非零向量 $\mathbf{v}$

\textbf{Term (EN)}: Eigenvalue, Eigenvector, Characteristic Polynomial

\tcblower

\textbf{定义}:
\begin{equation}
    \mathbf{Av} = \lambda\mathbf{v} \quad \Leftrightarrow \quad (\mathbf{A} - \lambda\mathbf{I})\mathbf{v} = \mathbf{0}
\end{equation}

\textbf{求解步骤}:
\begin{enumerate}
    \item 求特征多项式: $\det(\mathbf{A} - \lambda\mathbf{I}) = 0$
    \item 解出特征值 $\lambda_1, \lambda_2, \ldots$
    \item 对每个 $\lambda_i$,解 $(\mathbf{A} - \lambda_i\mathbf{I})\mathbf{v} = \mathbf{0}$ 得特征向量
\end{enumerate}

\textbf{Key Insight}: Eigenvalues reveal fundamental properties of a matrix (stability, principal directions).

\end{studybox}

\begin{formulabox}
\textbf{2×2 矩阵特征值}:

对于 $\mathbf{A} = \begin{pmatrix} a & b \\ c & d \end{pmatrix}$:

\textbf{特征多项式}:
\begin{equation}
    \lambda^2 - (a+d)\lambda + (ad-bc) = 0
\end{equation}

\textbf{简化}:
\begin{equation}
    \lambda^2 - \text{tr}(\mathbf{A})\lambda + \det(\mathbf{A}) = 0
\end{equation}

\textbf{解}:
\begin{equation}
    \lambda = \frac{\text{tr}(\mathbf{A}) \pm \sqrt{\text{tr}(\mathbf{A})^2 - 4\det(\mathbf{A})}}{2}
\end{equation}
\end{formulabox}

\begin{thmbox}[特征值性质]
\begin{itemize}
    \item $\sum \lambda_i = \text{tr}(\mathbf{A})$(特征值之和 = 迹)
    \item $\prod \lambda_i = \det(\mathbf{A})$(特征值之积 = 行列式)
    \item 对称矩阵的特征值都是实数
    \item 正定矩阵的特征值都 > 0
\end{itemize}
\end{thmbox}

% --------------------------------------------
% 4. Vector Spaces
% --------------------------------------------
\begin{studybox}{向量空间 (Vector Spaces)}

\textbf{概念 (CN)}: 满足向量加法和标量乘法封闭性的集合

\textbf{Term (EN)}: Basis, Dimension, Span, Linear Independence

\tcblower

\textbf{关键概念}:
\begin{itemize}[leftmargin=*]
    \item \textbf{线性无关}: 没有向量可以表示为其他向量的线性组合
    \item \textbf{基 (Basis)}: 线性无关且张成整个空间的向量组
    \item \textbf{维度 (Dimension)}: 基中向量的个数
    \item \textbf{秩 (Rank)}: 矩阵列空间的维度
\end{itemize}

\textbf{Key Insight}: Rank tells you the "effective" number of independent equations/constraints.

\end{studybox}

\begin{formulabox}
\textbf{秩-零化度定理}:
\begin{equation}
    \text{rank}(\mathbf{A}) + \text{nullity}(\mathbf{A}) = n
\end{equation}
其中 $n$ 是列数。

\textbf{齐次方程组解的结构}:
\begin{itemize}[leftmargin=*]
    \item $\text{rank}(\mathbf{A}) = n$: 唯一解 ($\mathbf{x} = \mathbf{0}$)
    \item $\text{rank}(\mathbf{A}) < n$: 无穷多解(解空间维度 = $n - \text{rank}$)
\end{itemize}
\end{formulabox}

% --------------------------------------------
% Thesis Connection
% --------------------------------------------
\begin{thesisbox}
\textbf{线性代数与传感器融合}:

你的论文使用 \textbf{MPU-6050 (6轴传感器)} 进行跌倒检测。

\textbf{传感器融合 (Kalman Filter) 用到的线性代数}:
\begin{itemize}
    \item 状态向量: $\mathbf{x} = [\theta, \dot{\theta}, \omega_{bias}]^T$
    \item 状态转移矩阵: $\mathbf{A} = \begin{pmatrix} 1 & dt & 0 \\ 0 & 1 & -dt \\ 0 & 0 & 1 \end{pmatrix}$
    \item 协方差矩阵更新: $\mathbf{P} = \mathbf{APA}^T + \mathbf{Q}$
\end{itemize}

\textbf{Jan Koller 问题}: "How do you combine Accelerometer and Gyroscope data?"

\textbf{答案}: 使用卡尔曼滤波器。加速度计提供绝对姿态(有噪声),陀螺仪提供角速度积分(有漂移)。卡尔曼滤波通过最小化协方差,融合两者得到最优估计。核心是矩阵运算(状态预测、协方差更新、卡尔曼增益计算)。
\end{thesisbox}

% --------------------------------------------
% Exam Strategy
% --------------------------------------------
\begin{warnbox}[[!] 考试陷阱 / Exam Pitfalls]
\begin{enumerate}
    \item \textbf{逆矩阵不存在}: $\det(\mathbf{A}) = 0$ 时矩阵奇异,无逆!
    \item \textbf{特征向量方向}: 特征向量乘以任意非零常数仍是特征向量。
    \item \textbf{矩阵乘法顺序}: $\mathbf{AB}$ 的尺寸是 $(m \times p)$ 当 $\mathbf{A}$ 是 $m \times n$,$\mathbf{B}$ 是 $n \times p$。
\end{enumerate}
\end{warnbox}
