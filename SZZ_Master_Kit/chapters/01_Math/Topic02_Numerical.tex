% ============================================
% Topic 02: Numerical Analysis / 数值分析
% Course: BE5B01DEN
% GRADE: E | PRIORITY: [CRITICAL]
% ============================================
\section{Numerical Analysis / 数值分析 (BE5B01DEN)}

\begin{warnbox}[\textbf{[WARN]} GRADE E DETECTED: SURVIVAL MODE / 成绩E:求生模式]
\textbf{[CN]} 你的数值分析成绩是 \textbf{E}。\textbf{只背这两个}:牛顿法公式 + 求根条件!\\
\textbf{[EN]} Your Numerical grade is \textbf{E}. \textbf{ONLY memorize}: Newton-Raphson + Convergence condition!\\
\textbf{Formula}: $x_{n+1} = x_n - \frac{f(x_n)}{f'(x_n)}$ | 收敛条件:$|f(x_0) \cdot f''(x_0)| < |f'(x_0)|^2$
\end{warnbox}

% --------------------------------------------
% 1. Roots and Optimization
% --------------------------------------------
\begin{studybox}{求根与优化 / Root Finding \& Optimization}
    \textbf{[CN] 定义}: 
    \begin{itemize}[leftmargin=*]
        \item \textbf{二分法 (Bisection)}: 简单稳健,收敛慢。基于介值定理。
        \item \textbf{牛顿法 (Newton's Method)}: 利用导数迭代 $x_{n+1} = x_n - f(x_n)/f'(x_n)$。收敛快 (二次收敛),但可能发散。
    \end{itemize}
    \tcblower
    \textbf{[EN] Definition}: 
    \begin{itemize}[leftmargin=*]
        \item \textbf{Bisection}: Robust but slow. Interval halving.
        \item \textbf{Newton's Method}: Iterative method using derivatives. Fast convergence (Quadratic), but requires good initial guess.
    \end{itemize}
\end{studybox}

% --------------------------------------------
% 2. Integration
% --------------------------------------------
\begin{studybox}{数值积分 / Numerical Integration}
    \textbf{[CN] 定义}: 
    用求和近似积分 $\int f(x) dx$。
    \begin{itemize}[leftmargin=*]
        \item \textbf{梯形法则 (Trapezoidal Rule)}: 用直线段连接点。
        \item \textbf{辛普森法则 (Simpson's Rule)}: 用抛物线拟合点。精度更高。
    \end{itemize}
    \tcblower
    \textbf{[EN] Definition}: 
    Approximating definite integrals.
    \begin{itemize}[leftmargin=*]
        \item \textbf{Trapezoidal}: Approximates area with trapezoids.
        \item \textbf{Simpson's}: Approximates with parabolas (Higher accuracy).
    \end{itemize}
\end{studybox}

\begin{formulabox}
\textbf{有限差分 / Finite Difference} (用于求导):
\begin{equation}
    f'(x) \approx \frac{f(x+h) - f(x)}{h} \quad (\text{前向差分})
\end{equation}
\end{formulabox}

% --------------------------------------------
% Thesis Connection
% --------------------------------------------
\begin{thesisbox}[论文关联 / Project Application]
    \textbf{[CN]}: 
    你的 ESP32 固件在处理传感器数据时使用数值方法:
    \begin{itemize}
        \item \textbf{[CN]} 积分 (Integration): 加速度计 $\to$ 速度 $\to$ 位移。这是黎曼和 (Riemann Sum) 的离散实现 ($v += a \cdot dt$)\\
              \textbf{[EN]} Integration: Accel $\to$ Vel $\to$ Pos. Discrete Riemann sum ($v += a \cdot dt$)
        \item \textbf{[CN]} 微分 (Differentiation): 陀螞仪角度变化率计算\\
              \textbf{[EN]} Differentiation: Gyroscope angular rate calculation
        \item \textbf{[CN]} 浮点数精度: IEEE 754 标准 (float/double) 的舍入误差会随时间累积 (Drift)\\
              \textbf{[EN]} Floating point precision: IEEE 754 rounding errors accumulate over time (Drift)
    \end{itemize}

    \tcblower

    \textbf{[EN] Full Mirror}: 
    Your ESP32 firmware uses numerical methods for sensor processing:
    \begin{itemize}
        \item \textbf{Integration}: Accelerometer $\to$ Velocity $\to$ Position via discrete accumulation
        \item \textbf{Differentiation}: Angular rate from gyroscope
        \item \textbf{Error Accumulation}: Floating-point rounding causes ``Drift'' in IMU navigation, requiring sensor fusion corrections
    \end{itemize}
\end{thesisbox}
