% ============================================
% Topic 02: Numerical Methods / 数值方法
% Course: BE5B01DEN
% ============================================
\section{数值方法 / Numerical Methods (BE5B01DEN)}

% --------------------------------------------
% 1. Error Analysis
% --------------------------------------------
\begin{studybox}{误差分析 (Error Analysis)}

\textbf{概念 (CN)}: 数值计算中的误差来源和传播

\textbf{Term (EN)}: Absolute Error, Relative Error, Round-off Error, Truncation Error

\tcblower

\textbf{误差类型}:
\begin{itemize}[leftmargin=*]
    \item \textbf{截断误差 (Truncation Error)}: 用有限项近似无限级数(如泰勒展开)
    \item \textbf{舍入误差 (Round-off Error)}: 计算机有限精度表示
    \item \textbf{传播误差 (Propagation Error)}: 初始误差在计算中放大
\end{itemize}

\textbf{Key Insight}: In ESP32, floating-point operations use 32-bit precision, causing round-off errors in sensor calculations.

\end{studybox}

\begin{formulabox}
\textbf{绝对误差 (Absolute Error)}:
\begin{equation}
    E_{abs} = |x_{true} - x_{approx}|
\end{equation}

\textbf{相对误差 (Relative Error)}:
\begin{equation}
    E_{rel} = \frac{|x_{true} - x_{approx}|}{|x_{true}|}
\end{equation}

\textbf{条件数 (Condition Number)}:
\begin{equation}
    \kappa(A) = \|A\| \cdot \|A^{-1}\|
\end{equation}
高条件数 $\Rightarrow$ 问题对误差敏感(ill-conditioned)
\end{formulabox}

% --------------------------------------------
% 2. Root Finding - Newton's Method
% --------------------------------------------
\begin{studybox}{求根方法 - 牛顿法 (Newton's Method)}

\textbf{概念 (CN)}: 迭代求解方程 $f(x) = 0$ 的根

\textbf{Term (EN)}: Newton-Raphson Method, Root Finding, Iteration

\tcblower

\textbf{核心思想}: 用切线近似曲线,切线与 x 轴交点作为新的近似值

\textbf{收敛性}:
\begin{itemize}[leftmargin=*]
    \item 二次收敛(Quadratic Convergence):误差平方减少
    \item 需要良好的初始值,否则可能发散
    \item 在根附近 $f'(x) \approx 0$ 时收敛慢
\end{itemize}

\textbf{Key Insight}: Newton's method is the foundation of many optimization algorithms in machine learning.

\end{studybox}

\begin{formulabox}
\textbf{牛顿迭代公式}:
\begin{equation}
    \boxed{x_{n+1} = x_n - \frac{f(x_n)}{f'(x_n)}}
\end{equation}

\textbf{示例}: 求 $\sqrt{2}$ (即 $f(x) = x^2 - 2 = 0$)
\begin{align}
    f'(x) &= 2x \\
    x_{n+1} &= x_n - \frac{x_n^2 - 2}{2x_n} = \frac{x_n + 2/x_n}{2}
\end{align}

\textbf{其他方法对比}:
\begin{center}
\begin{tabular}{lcc}
\toprule
\textbf{方法} & \textbf{收敛阶} & \textbf{需要导数?} \\
\midrule
二分法 (Bisection) & 1 (线性) & 否 \\
牛顿法 (Newton) & 2 (二次) & 是 \\
割线法 (Secant) & 1.618 & 否 \\
\bottomrule
\end{tabular}
\end{center}
\end{formulabox}

% --------------------------------------------
% 3. Numerical Integration
% --------------------------------------------
\begin{studybox}{数值积分 (Numerical Integration)}

\textbf{概念 (CN)}: 用离散求和近似定积分

\textbf{Term (EN)}: Quadrature, Trapezoidal Rule, Simpson's Rule

\tcblower

\textbf{基本思想}: 将积分区间分成小段,用简单函数近似

\textbf{常用方法}:
\begin{itemize}[leftmargin=*]
    \item \textbf{矩形法}: 用矩形面积近似
    \item \textbf{梯形法}: 用梯形面积近似(更精确)
    \item \textbf{Simpson法}: 用抛物线近似(最精确)
\end{itemize}

\textbf{Key Insight}: Higher-order methods need fewer points for the same accuracy.

\end{studybox}

\begin{formulabox}
\textbf{梯形公式 (Trapezoidal Rule)}:
\begin{equation}
    \int_a^b f(x)\,dx \approx \frac{h}{2}\left[f(a) + 2\sum_{i=1}^{n-1}f(x_i) + f(b)\right]
\end{equation}
其中 $h = \frac{b-a}{n}$,误差 $O(h^2)$

\textbf{Simpson 公式 (Simpson's Rule)}:
\begin{equation}
    \int_a^b f(x)\,dx \approx \frac{h}{3}\left[f(a) + 4\sum_{odd} f(x_i) + 2\sum_{even} f(x_i) + f(b)\right]
\end{equation}
误差 $O(h^4)$,需要偶数个区间
\end{formulabox}

% --------------------------------------------
% 4. Numerical ODE - Euler & Runge-Kutta
% --------------------------------------------
\begin{studybox}{常微分方程数值解 (Numerical ODE)}

\textbf{概念 (CN)}: 用离散步进求解 ODE 初值问题

\textbf{Term (EN)}: Euler Method, Runge-Kutta, Initial Value Problem (IVP)

\tcblower

\textbf{问题形式}: 给定 $\frac{dy}{dt} = f(t, y)$, $y(t_0) = y_0$,求 $y(t)$

\textbf{欧拉法 (Euler Method)}:
\begin{itemize}[leftmargin=*]
    \item 最简单的方法,一阶精度
    \item 误差随步长线性增长
\end{itemize}

\textbf{龙格-库塔法 (Runge-Kutta)}:
\begin{itemize}[leftmargin=*]
    \item RK4 是最常用的四阶方法
    \item 精度高,稳定性好
\end{itemize}

\textbf{Key Insight}: ESP32 uses Euler-like integration for sensor fusion (complementary filter).

\end{studybox}

\begin{formulabox}
\textbf{前向欧拉法 (Forward Euler)}:
\begin{equation}
    \boxed{y_{n+1} = y_n + h \cdot f(t_n, y_n)}
\end{equation}

\textbf{四阶龙格-库塔 (RK4)}:
\begin{align}
    k_1 &= f(t_n, y_n) \\
    k_2 &= f(t_n + \frac{h}{2}, y_n + \frac{h}{2}k_1) \\
    k_3 &= f(t_n + \frac{h}{2}, y_n + \frac{h}{2}k_2) \\
    k_4 &= f(t_n + h, y_n + h \cdot k_3)
\end{align}
\begin{equation}
    \boxed{y_{n+1} = y_n + \frac{h}{6}(k_1 + 2k_2 + 2k_3 + k_4)}
\end{equation}
\end{formulabox}

% --------------------------------------------
% 5. Linear Systems - Gaussian Elimination
% --------------------------------------------
\begin{studybox}{线性方程组数值解 (Solving Linear Systems)}

\textbf{概念 (CN)}: 数值求解 $\mathbf{Ax} = \mathbf{b}$

\textbf{Term (EN)}: Gaussian Elimination, LU Decomposition, Pivoting

\tcblower

\textbf{高斯消元法}:
\begin{enumerate}
    \item 前向消元:将矩阵变为上三角形式
    \item 回代:从最后一行开始求解
\end{enumerate}

\textbf{主元选取 (Pivoting)}:
\begin{itemize}[leftmargin=*]
    \item 避免除以零或小数
    \item 部分主元:选列中最大元素
    \item 完全主元:选整个矩阵最大元素
\end{itemize}

\textbf{Key Insight}: Condition number determines how errors propagate in linear systems.

\end{studybox}

% --------------------------------------------
% Thesis Connection
% --------------------------------------------
\begin{thesisbox}
\textbf{数值方法与智能家居传感器}:

你的论文使用多种数值方法处理传感器数据:

\textbf{移动平均滤波 (Moving Average)}:
\begin{equation}
    y_n = \frac{1}{N}\sum_{i=0}^{N-1} x_{n-i}
\end{equation}
这是一种简单的数值滤波方法,用于平滑温湿度读数。

\textbf{互补滤波器 (Complementary Filter)}:
\begin{equation}
    \theta = \alpha \cdot (\theta_{prev} + \omega \cdot dt) + (1-\alpha) \cdot \theta_{acc}
\end{equation}
用于融合加速度计和陀螺仪数据,本质上是数值积分 + 加权平均。

\textbf{Jan Koller 问题}: "How do you handle sensor noise?"

\textbf{答案}: 使用移动平均滤波(一种数值方法)平滑读数,减少高频噪声。对于姿态估计,使用互补滤波器结合数值积分。
\end{thesisbox}

% --------------------------------------------
% Exam Strategy
% --------------------------------------------
\begin{warnbox}[[!] 考试陷阱 / Exam Pitfalls]
\begin{enumerate}
    \item \textbf{牛顿法发散}: 初始值选得不好,或 $f'(x_0) \approx 0$。
    \item \textbf{Simpson 需要偶数区间}: $n$ 必须是偶数!
    \item \textbf{条件数}: 高条件数矩阵 $\Rightarrow$ 结果不可靠。
    \item \textbf{步长选择}: 太大 $\Rightarrow$ 不精确,太小 $\Rightarrow$ 舍入误差累积。
\end{enumerate}
\end{warnbox}
