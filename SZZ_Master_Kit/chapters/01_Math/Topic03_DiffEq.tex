% ============================================
% Topic 03: Differential Equations / 微分方程
% Course: BE5B01DEN
% ============================================
\section{微分方程 / Differential Equations (BE5B01DEN)}

% --------------------------------------------
% 1. First Order ODE
% --------------------------------------------
\begin{studybox}{一阶常微分方程 (First Order ODE)}

\textbf{概念 (CN)}: 含有一阶导数的方程

\textbf{Term (EN)}: First Order ODE, Separable, Homogeneous

\tcblower

\textbf{标准形式}:
\begin{equation}
    \frac{dy}{dx} = f(x, y)
\end{equation}

\textbf{主要类型}:
\begin{itemize}[leftmargin=*]
    \item \textbf{可分离变量}: $\frac{dy}{dx} = g(x)h(y)$
    \item \textbf{一阶线性}: $\frac{dy}{dx} + P(x)y = Q(x)$
    \item \textbf{齐次方程}: $\frac{dy}{dx} = f(\frac{y}{x})$
\end{itemize}

\textbf{Key Insight}: RC circuit charging is described by a first-order linear ODE.

\end{studybox}

\begin{formulabox}
\textbf{可分离变量方程解法}:
\begin{equation}
    \frac{dy}{dx} = g(x)h(y) \quad \Rightarrow \quad \int \frac{1}{h(y)}dy = \int g(x)dx + C
\end{equation}

\textbf{一阶线性方程解法} (积分因子法):
\begin{equation}
    \frac{dy}{dx} + P(x)y = Q(x)
\end{equation}
\textbf{积分因子}: $\mu(x) = e^{\int P(x)dx}$

\textbf{通解}:
\begin{equation}
    \boxed{y = \frac{1}{\mu(x)}\left[\int \mu(x)Q(x)dx + C\right]}
\end{equation}
\end{formulabox}

% --------------------------------------------
% 2. Higher Order Linear ODE
% --------------------------------------------
\begin{studybox}{高阶线性常微分方程 (Higher Order Linear ODE)}

\textbf{概念 (CN)}: 含有二阶及以上导数的线性方程

\textbf{Term (EN)}: Second Order ODE, Characteristic Equation, Homogeneous/Non-homogeneous

\tcblower

\textbf{二阶齐次方程}:
\begin{equation}
    y'' + p(x)y' + q(x)y = 0
\end{equation}

\textbf{常系数情况}: $ay'' + by' + cy = 0$

\textbf{解法}: 用特征方程 $ar^2 + br + c = 0$

\textbf{Key Insight}: RLC circuit oscillation is described by a second-order ODE.

\end{studybox}

\begin{formulabox}
\textbf{特征方程} $ar^2 + br + c = 0$:

\textbf{判别式}: $\Delta = b^2 - 4ac$

\begin{center}
\begin{tabular}{lc}
\toprule
\textbf{情况} & \textbf{通解} \\
\midrule
$\Delta > 0$ (两个实根 $r_1, r_2$) & $y = C_1 e^{r_1 x} + C_2 e^{r_2 x}$ \\
$\Delta = 0$ (重根 $r$) & $y = (C_1 + C_2 x)e^{rx}$ \\
$\Delta < 0$ (共轭复根 $\alpha \pm \beta i$) & $y = e^{\alpha x}(C_1\cos\beta x + C_2\sin\beta x)$ \\
\bottomrule
\end{tabular}
\end{center}

\textbf{非齐次方程}: $y = y_h + y_p$ (齐次解 + 特解)
\end{formulabox}

% --------------------------------------------
% 3. Systems of Linear ODE
% --------------------------------------------
\begin{studybox}{线性微分方程组 (Systems of Linear ODE)}

\textbf{概念 (CN)}: 多个未知函数的微分方程组成的系统

\textbf{Term (EN)}: System of ODEs, Matrix Exponential, Phase Portrait

\tcblower

\textbf{矩阵形式}:
\begin{equation}
    \mathbf{x}' = \mathbf{A}\mathbf{x}
\end{equation}

\textbf{解法}:
\begin{enumerate}
    \item 求矩阵 $\mathbf{A}$ 的特征值 $\lambda_1, \lambda_2, \ldots$
    \item 求对应的特征向量 $\mathbf{v}_1, \mathbf{v}_2, \ldots$
    \item 通解: $\mathbf{x}(t) = C_1 e^{\lambda_1 t}\mathbf{v}_1 + C_2 e^{\lambda_2 t}\mathbf{v}_2 + \ldots$
\end{enumerate}

\textbf{Key Insight}: Stability of linear systems depends on eigenvalues: $\text{Re}(\lambda) < 0 \Rightarrow$ stable.

\end{studybox}

\begin{formulabox}
\textbf{二维系统稳定性}:

对于 $\mathbf{x}' = \mathbf{A}\mathbf{x}$,令 $\lambda_1, \lambda_2$ 为特征值:

\begin{center}
\begin{tabular}{lc}
\toprule
\textbf{特征值类型} & \textbf{稳定性} \\
\midrule
$\lambda_1, \lambda_2 < 0$ & 稳定结点 (Stable Node) \\
$\lambda_1, \lambda_2 > 0$ & 不稳定结点 (Unstable Node) \\
$\lambda_1 < 0 < \lambda_2$ & 鞍点 (Saddle) - 不稳定 \\
$\lambda = \alpha \pm \beta i$, $\alpha < 0$ & 稳定焦点 (Stable Focus) \\
$\lambda = \pm \beta i$ & 中心 (Center) - 边界稳定 \\
\bottomrule
\end{tabular}
\end{center}
\end{formulabox}

% --------------------------------------------
% 4. Laplace Transform
% --------------------------------------------
\begin{studybox}{拉普拉斯变换 (Laplace Transform)}

\textbf{概念 (CN)}: 将时域函数变换到复频域的积分变换

\textbf{Term (EN)}: Laplace Transform, Transfer Function, Inverse Transform

\tcblower

\textbf{定义}:
\begin{equation}
    \mathcal{L}\{f(t)\} = F(s) = \int_0^{\infty} f(t)e^{-st}dt
\end{equation}

\textbf{ODE 求解步骤}:
\begin{enumerate}
    \item 对方程两边做 Laplace 变换
    \item 在 s 域求解代数方程
    \item 做逆变换回时域
\end{enumerate}

\textbf{Key Insight}: Laplace transform converts differential equations into algebraic equations!

\end{studybox}

\begin{formulabox}
\textbf{常用变换表}:

\begin{center}
\begin{tabular}{lc}
\toprule
$f(t)$ & $F(s) = \mathcal{L}\{f(t)\}$ \\
\midrule
$1$ & $\frac{1}{s}$ \\
$t$ & $\frac{1}{s^2}$ \\
$e^{at}$ & $\frac{1}{s-a}$ \\
$\sin(\omega t)$ & $\frac{\omega}{s^2 + \omega^2}$ \\
$\cos(\omega t)$ & $\frac{s}{s^2 + \omega^2}$ \\
$f'(t)$ & $sF(s) - f(0)$ \\
$f''(t)$ & $s^2 F(s) - sf(0) - f'(0)$ \\
\bottomrule
\end{tabular}
\end{center}
\end{formulabox}

% --------------------------------------------
% Thesis Connection
% --------------------------------------------
\begin{thesisbox}
\textbf{微分方程与传感器系统}:

你的论文中的传感器系统可以用微分方程建模:

\textbf{RC 低通滤波器} (一阶 ODE):
\begin{equation}
    RC\frac{dV_{out}}{dt} + V_{out} = V_{in}
\end{equation}

\textbf{陀螺仪角度积分} (初值问题):
\begin{equation}
    \theta(t) = \theta_0 + \int_0^t \omega(\tau)d\tau
\end{equation}

\textbf{PID 控制} (微分方程描述):
\begin{equation}
    u(t) = K_p e(t) + K_i \int_0^t e(\tau)d\tau + K_d \frac{de(t)}{dt}
\end{equation}

\textbf{Jan Koller 问题}: "How does your system respond to step inputs?"

\textbf{答案}: 系统响应由一阶 ODE 决定。时间常数 $\tau = RC$ 决定响应速度。63.2\% 的稳态值在 $t = \tau$ 时达到。
\end{thesisbox}

% --------------------------------------------
% Exam Strategy
% --------------------------------------------
\begin{warnbox}[[!] 考试陷阱 / Exam Pitfalls]
\begin{enumerate}
    \item \textbf{特征方程}: 不要忘记负号!$y'' - y = 0$ 的特征方程是 $r^2 - 1 = 0$。
    \item \textbf{初值条件}: 通解中的常数由初值条件确定。
    \item \textbf{稳定性判据}: 所有特征值实部 < 0 才稳定!
    \item \textbf{Laplace 导数}: $\mathcal{L}\{f'\} = sF(s) - f(0)$,不要忘记初值项。
\end{enumerate}
\end{warnbox}
