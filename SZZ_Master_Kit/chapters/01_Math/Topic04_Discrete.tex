% ============================================
% Topic 03: Discrete Math / 离散数学
% Course: BE5B01DMA
% ============================================
\section{离散数学 / Discrete Math (BE5B01DMA)}

% --------------------------------------------
% 1. Sets and Logic
% --------------------------------------------
\begin{studybox}{集合与逻辑 / Sets \& Logic}
    \textbf{[CN] 定义}: 
    \begin{itemize}[leftmargin=*]
        \item \textbf{集合 (Set)}: 唯一对象的汇集。$\cup$ (并), $\cap$ (交), $\setminus$ (不包含).
        \item \textbf{命题逻辑}: AND ($\land$), OR ($\lor$), NOT ($\neg$), IMPLIES ($\Rightarrow$).
        \item \textbf{关系 (Relation)}: 等价关系 (自反、对称、传递) 和偏序关系。
    \end{itemize}
    \tcblower
    \textbf{[EN] Definition}: 
    \begin{itemize}[leftmargin=*]
        \item \textbf{Set}: Collection of unique objects. Union, Intersection, Difference.
        \item \textbf{Logic}: Boolean algebra ($\land, \lor, \neg$).
        \item \textbf{Relation}: Equivalence (Reflexive, Symmetric, Transitive) and Partial Order.
    \end{itemize}
\end{studybox}

% --------------------------------------------
% 2. Combinatorics
% --------------------------------------------
\begin{formulabox}
\textbf{排列组合 / Combinatorics}:
\begin{itemize}
    \item \textbf{排列 (Permutation)}: 有序. $P(n,k) = \frac{n!}{(n-k)!}$.
    \item \textbf{组合 (Combination)}: 无序. $C(n,k) = \binom{n}{k} = \frac{n!}{k!(n-k)!}$.
\end{itemize}
\end{formulabox}

% --------------------------------------------
% Thesis Connection
% --------------------------------------------
\begin{thesisbox}[论文关联 / Project Application]
    \textbf{[CN]}: 
    \begin{itemize}
        \item \textbf{布尔逻辑}: 代码中的 \texttt{if (Condition A \&\& Condition B)} 是命题逻辑的直接应用。
        \item \textbf{状态机}: 有限状态机 (FSM) 用于管理 ESP32 的连接状态 (Disconnected $\to$ Connecting $\to$ Connected)。这是离散数学中的图论应用。
        \item \textbf{位运算}: 掩码 (Masking) 操作对应集合的交集运算。
    \end{itemize}

    \tcblower

    \textbf{[EN]}: 
    \begin{itemize}
        \item \textbf{Boolean Logic}: `if` statements relate directly to Propositional Logic.
        \item \textbf{FSM}: Finite State Machines manage connection states, rooted in Graph Theory.
        \item \textbf{Bitwise Ops}: Masking corresponds to Set Intersection.
    \end{itemize}
\end{thesisbox}
