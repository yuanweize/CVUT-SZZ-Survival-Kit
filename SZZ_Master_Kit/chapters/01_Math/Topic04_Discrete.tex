% ============================================
% Topic 04: Discrete Mathematics / 离散数学
% Course: BE5B01DMG
% ============================================
\section{离散数学 / Discrete Mathematics (BE5B01DMG)}

% --------------------------------------------
% 1. Logic
% --------------------------------------------
\begin{studybox}{命题逻辑 (Propositional Logic)}

\textbf{概念 (CN)}: 研究命题之间的逻辑关系

\textbf{Term (EN)}: Proposition, Logical Connectives, Truth Table, Tautology

\tcblower

\textbf{逻辑连接词}:
\begin{itemize}[leftmargin=*]
    \item $\neg p$ (NOT): 非
    \item $p \land q$ (AND): 合取
    \item $p \lor q$ (OR): 析取
    \item $p \to q$ (IMPLIES): 蕴含
    \item $p \leftrightarrow q$ (IFF): 等价
\end{itemize}

\textbf{重要等价式}:
\begin{itemize}[leftmargin=*]
    \item $p \to q \equiv \neg p \lor q$
    \item $\neg(p \land q) \equiv \neg p \lor \neg q$ (德摩根定律)
    \item $\neg(p \lor q) \equiv \neg p \land \neg q$ (德摩根定律)
\end{itemize}

\textbf{Key Insight}: Digital logic gates (AND, OR, NOT) implement these operations in hardware.

\end{studybox}

% --------------------------------------------
% 2. Set Theory
% --------------------------------------------
\begin{studybox}{集合论 (Set Theory)}

\textbf{概念 (CN)}: 研究集合及其运算

\textbf{Term (EN)}: Set, Subset, Union, Intersection, Cardinality

\tcblower

\textbf{基本运算}:
\begin{itemize}[leftmargin=*]
    \item $A \cup B$: 并集 (Union)
    \item $A \cap B$: 交集 (Intersection)
    \item $A \setminus B$: 差集 (Difference)
    \item $A^c$: 补集 (Complement)
    \item $A \times B$: 笛卡尔积 (Cartesian Product)
\end{itemize}

\textbf{幂集}: $\mathcal{P}(A)$ 包含 $A$ 的所有子集,$|\mathcal{P}(A)| = 2^{|A|}$

\textbf{Key Insight}: Set operations correspond to logical operations via indicator functions.

\end{studybox}

\begin{formulabox}
\textbf{容斥原理 (Inclusion-Exclusion)}:
\begin{equation}
    |A \cup B| = |A| + |B| - |A \cap B|
\end{equation}

\textbf{三个集合}:
\begin{equation}
    |A \cup B \cup C| = |A| + |B| + |C| - |A \cap B| - |A \cap C| - |B \cap C| + |A \cap B \cap C|
\end{equation}

\textbf{集合基数}:
\begin{itemize}[leftmargin=*]
    \item $|\mathbb{N}| = \aleph_0$ (可数无穷)
    \item $|\mathbb{R}| = \mathfrak{c}$ (连续统)
    \item $|\mathfrak{c}| > |\aleph_0|$ (康托定理)
\end{itemize}
\end{formulabox}

% --------------------------------------------
% 3. Number Theory
% --------------------------------------------
\begin{studybox}{数论 (Number Theory)}

\textbf{概念 (CN)}: 研究整数的性质

\textbf{Term (EN)}: Divisibility, Modular Arithmetic, GCD, Prime

\tcblower

\textbf{模运算}:
\begin{equation}
    a \equiv b \pmod{n} \quad \Leftrightarrow \quad n \mid (a-b)
\end{equation}

\textbf{模运算性质}:
\begin{itemize}[leftmargin=*]
    \item $(a + b) \mod n = ((a \mod n) + (b \mod n)) \mod n$
    \item $(a \cdot b) \mod n = ((a \mod n) \cdot (b \mod n)) \mod n$
\end{itemize}

\textbf{最大公约数 (GCD)}: 用欧几里得算法求解

\textbf{Key Insight}: Cryptography (RSA) relies heavily on modular arithmetic.

\end{studybox}

\begin{formulabox}
\textbf{欧几里得算法 (Euclidean Algorithm)}:
\begin{equation}
    \gcd(a, b) = \gcd(b, a \mod b), \quad \gcd(a, 0) = a
\end{equation}

\textbf{扩展欧几里得}: 求 $ax + by = \gcd(a, b)$ 的整数解

\textbf{费马小定理}:
\begin{equation}
    a^{p-1} \equiv 1 \pmod{p} \quad \text{(if } p \text{ is prime, } \gcd(a, p) = 1\text{)}
\end{equation}

\textbf{欧拉定理}:
\begin{equation}
    a^{\phi(n)} \equiv 1 \pmod{n} \quad \text{(if } \gcd(a, n) = 1\text{)}
\end{equation}
\end{formulabox}

% --------------------------------------------
% 4. Combinatorics
% --------------------------------------------
\begin{studybox}{组合数学 (Combinatorics)}

\textbf{概念 (CN)}: 研究有限集合的计数和排列

\textbf{Term (EN)}: Permutation, Combination, Binomial Coefficient

\tcblower

\textbf{排列 (Permutation)}: 有序选择
\begin{equation}
    P(n, k) = \frac{n!}{(n-k)!}
\end{equation}

\textbf{组合 (Combination)}: 无序选择
\begin{equation}
    C(n, k) = \binom{n}{k} = \frac{n!}{k!(n-k)!}
\end{equation}

\textbf{Key Insight}: Combinations count subsets; permutations count arrangements.

\end{studybox}

\begin{formulabox}
\textbf{二项式定理}:
\begin{equation}
    (x + y)^n = \sum_{k=0}^{n} \binom{n}{k} x^k y^{n-k}
\end{equation}

\textbf{帕斯卡恒等式}:
\begin{equation}
    \binom{n}{k} = \binom{n-1}{k-1} + \binom{n-1}{k}
\end{equation}

\textbf{鸽巢原理 (Pigeonhole Principle)}:
\begin{equation}
    \text{If } n+1 \text{ objects are placed in } n \text{ boxes, at least one box contains } \geq 2 \text{ objects.}
\end{equation}
\end{formulabox}

% --------------------------------------------
% 5. Graph Theory
% --------------------------------------------
\begin{studybox}{图论 (Graph Theory)}

\textbf{概念 (CN)}: 研究由顶点和边组成的结构

\textbf{Term (EN)}: Vertex, Edge, Adjacency, Path, Cycle, Connected, Tree

\tcblower

\textbf{基本定义}:
\begin{itemize}[leftmargin=*]
    \item \textbf{图 (Graph)}: $G = (V, E)$,顶点集和边集
    \item \textbf{度 (Degree)}: 顶点连接的边数
    \item \textbf{路径 (Path)}: 不重复顶点的边序列
    \item \textbf{环 (Cycle)}: 起点=终点的路径
    \item \textbf{树 (Tree)}: 连通无环图
\end{itemize}

\textbf{Key Insight}: Network topology (Star, Mesh) is a graph theory problem.

\end{studybox}

\begin{formulabox}
\textbf{握手定理 (Handshaking Lemma)}:
\begin{equation}
    \sum_{v \in V} \deg(v) = 2|E|
\end{equation}

\textbf{树的性质}:
\begin{itemize}[leftmargin=*]
    \item $|E| = |V| - 1$
    \item 任意两点之间有唯一路径
    \item 移除任意边使图不连通
\end{itemize}

\textbf{欧拉路径/回路}:
\begin{itemize}[leftmargin=*]
    \item 欧拉路径存在 $\Leftrightarrow$ 恰有 0 或 2 个奇度顶点
    \item 欧拉回路存在 $\Leftrightarrow$ 所有顶点度数为偶数
\end{itemize}
\end{formulabox}

% --------------------------------------------
% 6. Graph Algorithms
% --------------------------------------------
\begin{studybox}{图算法 (Graph Algorithms)}

\textbf{概念 (CN)}: 在图上进行搜索和优化

\textbf{Term (EN)}: BFS, DFS, Dijkstra, Shortest Path

\tcblower

\textbf{广度优先搜索 (BFS)}:
\begin{itemize}[leftmargin=*]
    \item 使用队列 (Queue)
    \item 按层遍历
    \item 找到的是最短路径(无权图)
\end{itemize}

\textbf{深度优先搜索 (DFS)}:
\begin{itemize}[leftmargin=*]
    \item 使用栈 (Stack) 或递归
    \item 尽可能深入
    \item 用于检测环、拓扑排序
\end{itemize}

\textbf{Dijkstra 算法}:
\begin{itemize}[leftmargin=*]
    \item 求加权图的最短路径
    \item 时间复杂度 $O(|V|^2)$ 或 $O(|E| \log |V|)$ (使用堆)
\end{itemize}

\textbf{Key Insight}: Routing protocols (OSPF) use Dijkstra's algorithm.

\end{studybox}

% --------------------------------------------
% Thesis Connection
% --------------------------------------------
\begin{thesisbox}
\textbf{离散数学与智能家居网络}:

你的论文使用图论概念描述网络拓扑:

\textbf{MQTT 网络拓扑}:
\begin{itemize}
    \item \textbf{星型拓扑 (Star)}: 所有设备连接到中心 Broker(你的设计)
    \item 图表示: $G = (V, E)$, $|V| = n+1$, $|E| = n$
    \item 这是一棵树!满足 $|E| = |V| - 1$
\end{itemize}

\textbf{模运算应用}:
\begin{itemize}
    \item CRC 校验使用模 2 运算
    \item 时间戳循环使用模 $2^{32}$ 运算
\end{itemize}

\textbf{Jan Koller 问题}: "What is the network topology of your system?"

\textbf{答案}: 星型拓扑 (Star Topology)。所有 ESP32 设备作为叶节点连接到中心 MQTT Broker。这是一棵树结构,优点是简单、易管理,缺点是单点故障。
\end{thesisbox}

% --------------------------------------------
% Exam Strategy
% --------------------------------------------
\begin{warnbox}[[!] 考试陷阱 / Exam Pitfalls]
\begin{enumerate}
    \item \textbf{模运算}: $(a - b) \mod n \neq (a \mod n) - (b \mod n)$,可能为负!
    \item \textbf{组合 vs 排列}: 注意题目是否要求顺序。
    \item \textbf{欧拉路径 vs 回路}: 0 个奇度顶点是回路,2 个奇度顶点是路径。
    \item \textbf{BFS vs DFS}: BFS 找最短路径(无权),DFS 不一定。
\end{enumerate}
\end{warnbox}
