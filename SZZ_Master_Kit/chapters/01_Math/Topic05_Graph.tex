% ============================================
% Topic 05: Graph Theory / 图论
% Course: BE5B01DMA
% ============================================
\section{图论 / Graph Theory (BE5B01DMA)}

% --------------------------------------------
% 1. Basic Concepts
% --------------------------------------------
\begin{studybox}{图的概念 / Graph Concepts}
    \textbf{[CN] 定义}: 
    图 $G = (V, E)$ 由顶点集合 $V$ 和边集合 $E$ 组成。
    \begin{itemize}[leftmargin=*]
        \item \textbf{度 (Degree)}: 连接到一个顶点的边的数量。
        \item \textbf{路径 (Path)}: 顶点序列。
        \item \textbf{连通图 (Connected)}: 任意两点间都存在路径。
    \end{itemize}
    \tcblower
    \textbf{[EN] Definition}: 
    Graph $G = (V, E)$ consists of Vertices $V$ and Edges $E$.
    \begin{itemize}[leftmargin=*]
        \item \textbf{Degree}: Number of edges incident to a vertex.
        \item \textbf{Path}: Sequence of adjacent vertices.
        \item \textbf{Connected}: Path exists between any pair of vertices.
    \end{itemize}
\end{studybox}

\begin{formulabox}
\textbf{树 (Tree)}:
无环连通图 (Connected Acyclic Graph)。
性质:
\begin{itemize}
    \item 任意两点间只有\textbf{唯一}路径。
    \item $|E| = |V| - 1$ (边数 = 顶点数 - 1)。
\end{itemize}

\textbf{欧拉路径 (Euler Path)}: 经过每条边恰好一次。条件: 奇度数顶点为 0 或 2 个。
\end{formulabox}

% --------------------------------------------
% Thesis Connection
% --------------------------------------------
\begin{thesisbox}[论文关联 / Project Application]
    \textbf{[CN]}: 
    \begin{itemize}
        \item \textbf{网络拓扑}: 你的智能家居采用\textbf{星型拓扑} (Star Topology)。WiFi 路由器是中心节点 (Hub),所有 ESP32 设备是叶子节点 (Leaves)。
        \item \textbf{MQTT 主题}: MQTT 的 Topic 层级结构 (e.g., `home/livingroom/temp`) 实际上构成了一棵\textbf{树} (Topic Tree)。
        \item \textbf{状态机}: 状态转移图 (State Transition Graph) 是有向图 (Directed Graph)。
    \end{itemize}

    \tcblower

    \textbf{[EN]}: 
    \begin{itemize}
        \item \textbf{Network Topology}: Star Topology. Router is the Hub; ESP32s are Leaves.
        \item \textbf{MQTT Topics}: Form a Topic Tree hierarchy.
        \item \textbf{FSM}: Represented as a Directed Graph where nodes are states and edges are transitions.
    \end{itemize}
\end{thesisbox}
