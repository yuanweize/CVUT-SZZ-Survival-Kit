% ============================================
% Topic 06: Probability / 概率论
% Course: BE5B01PST
% ============================================
\section{概率统计 / Probability \& Statistics (BE5B01PST)}

% --------------------------------------------
% 1. Random Variables
% --------------------------------------------
\begin{studybox}{随机变量 / Random Variables}
    \textbf{[CN] 定义}: 
    \begin{itemize}[leftmargin=*]
        \item \textbf{随机变量 (RV)}: 随机实验结果的数值表示。
        \item \textbf{概率密度函数 (PDF)}: 描述连续 RV 在某点附近取值的概率密度。
        \item \textbf{累积分布函数 (CDF)}: $F(x) = P(X \le x)$。
    \end{itemize}
    \tcblower
    \textbf{[EN] Definition}: 
    \begin{itemize}[leftmargin=*]
        \item \textbf{Random Variable (RV)}: Numerical outcome of a random phenomenon.
        \item \textbf{PDF}: Integration gives probability.
        \item \textbf{CDF}: Cumulative Distribution Function.
    \end{itemize}
\end{studybox}

\begin{formulabox}
\textbf{期望与方差}:
\begin{itemize}
    \item \textbf{期望 (Expectation)}: $E[X] = \mu$ (平均值)
    \item \textbf{方差 (Variance)}: $\text{Var}(X) = E[(X-\mu)^2] = E[X^2] - (E[X])^2$
\end{itemize}

\textbf{正态分布 (Normal Distribution)}:
\begin{equation}
    f(x) = \frac{1}{\sigma\sqrt{2\pi}} e^{-\frac{(x-\mu)^2}{2\sigma^2}}
\end{equation}
\end{formulabox}

% --------------------------------------------
% 2. Statistics
% --------------------------------------------
\begin{studybox}{假设检验 / Hypothesis Testing}
    \textbf{[CN] 定义}: 
    \begin{itemize}[leftmargin=*]
        \item \textbf{零假设 ($H_0$)}: 默认假设 (如"无关联")。
        \item \textbf{P值 (p-value)}: 在 $H_0$ 为真时,观察到当前结果的概率。若 $p < \alpha$,拒绝 $H_0$。
        \item \textbf{第一类错误 (Type I)}: 假阳性 (False Positive)。
        \item \textbf{第二类错误 (Type II)}: 假阴性 (False Negative)。
    \end{itemize}
    \tcblower
    \textbf{[EN] Definition}: 
    \begin{itemize}[leftmargin=*]
        \item \textbf{Null Hypothesis ($H_0$)}: Default assumption.
        \item \textbf{p-value}: Prob. of observing results given $H_0$ is true. Reject if $p < 0.05$.
        \item \textbf{Type I Error}: False Positive (Reject true $H_0$).
        \item \textbf{Type II Error}: False Negative (Accept false $H_0$).
    \end{itemize}
\end{studybox}

% --------------------------------------------
% Thesis Connection
% --------------------------------------------
\begin{thesisbox}[论文关联 / Project Application]
    \textbf{[CN]}: 
    \begin{itemize}
        \item \textbf{传感器噪声}: 假设 MEMS 加速度计的噪声服从\textbf{高斯分布} (Gaussian White Noise)。这是卡尔曼滤波和移动平均滤波的基本假设。
        \item \textbf{跌倒检测性能}: 评估算法时,你需要计算灵敏度 (Sensitivity) 和特异度 (Specificity)。这直接对应于统计学中的 True Positives 和 True Negatives。
    \end{itemize}

    \tcblower

    \textbf{[EN]}: 
    \begin{itemize}
        \item \textbf{Sensor Noise}: Assumed to be Gaussian (Normal) White Noise. Filter design relies on signal-to-noise ratio (SNR).
        \item \textbf{Performance}: Sensitivity (Recall) and Specificity metrics quantify the fall detection algorithm's accuracy (checking for Type I/II errors).
    \end{itemize}
\end{thesisbox}
