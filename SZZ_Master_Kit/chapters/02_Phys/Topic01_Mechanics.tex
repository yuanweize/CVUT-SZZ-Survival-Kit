% ============================================
% Topic 01: Mechanics / 力学
% Course: BE5B02PH1
% ============================================
\section{力学 / Mechanics (BE5B02PH1)}

% --------------------------------------------
% 1. Kinematics
% --------------------------------------------
\begin{studybox}{运动学 / Kinematics}
    \textbf{[CN] 定义}: 
    运动学描述物体的运动而不考虑其成因 (力)。
    \begin{itemize}[leftmargin=*]
        \item \textbf{基本量}: 
        位移 (Displacement) $\vec{r}$, 
        速度 (Velocity) $\vec{v} = d\vec{r}/dt$, 
        加速度 (Acceleration) $\vec{a} = d\vec{v}/dt$.
        \item \textbf{圆周运动}: 线速度 $v=\omega r$, 向心加速度 $a_c = v^2/r$.
    \end{itemize}
    \tcblower
    \textbf{[EN] Definition}: 
    Kinematics describes motion without considering the forces that cause it.
    \begin{itemize}[leftmargin=*]
        \item \textbf{Basic Quantities}: 
        Displacement $\vec{r}$, 
        Velocity $\vec{v} = \dot{\vec{r}}$, 
        Acceleration $\vec{a} = \dot{\vec{v}}$.
        \item \textbf{Circular Motion}: Linear vel $v=\omega r$, Centripetal acc $a_c = v^2/r$.
    \end{itemize}
\end{studybox}

\begin{formulabox}
\textbf{匀加速直线运动 / Uniformly Accelerated Motion}:
\begin{align}
    v &= v_0 + at \\
    x &= x_0 + v_0 t + \frac{1}{2}at^2 \\
    v^2 &= v_0^2 + 2a(x - x_0)
\end{align}
\textbf{应用}: 计算自由落体距离 (Free fall distance).
\end{formulabox}

% --------------------------------------------
% 2. Newton's Laws
% --------------------------------------------
\begin{studybox}{牛顿定律 / Newton's Laws}
    \textbf{[CN] 定义}: 
    动力学的基础。
    \begin{enumerate}
        \item \textbf{第一定律 (惯性)}: 物体保持静止或匀速直线运动,除非有力作用。
        \item \textbf{第二定律}: 力等于质量乘以加速度 ($\vec{F}=m\vec{a}$)。这是整个力学的核心方程。
        \item \textbf{第三定律}: 作用力与反作用力大小相等、方向相反。
    \end{enumerate}
    \tcblower
    \textbf{[EN] Definition}: 
    The foundation of dynamics.
    \begin{enumerate}
        \item \textbf{1st (Inertia)}: An object remains at rest or uniform motion unless acted upon by a force.
        \item \textbf{2nd}: Force equals mass times acceleration ($\vec{F}=m\vec{a}$). The core equation.
        \item \textbf{3rd}: Action and reaction are equal in magnitude and opposite in direction.
    \end{enumerate}
\end{studybox}

% --------------------------------------------
% 3. Energy
% --------------------------------------------
\begin{studybox}{功与能 / Work and Energy}
    \textbf{[CN] 定义}: 
    \begin{itemize}[leftmargin=*]
        \item \textbf{功 (Work)}: 力在位移方向上的累积 ($W = \int \vec{F} \cdot d\vec{r}$).
        \item \textbf{动能 (Kinetic Energy)}: 运动物体具有的能量 ($E_k = \frac{1}{2}mv^2$).
        \item \textbf{势能 (Potential Energy)}: 位置决定的储能 (重力 $mgh$, 弹簧 $\frac{1}{2}kx^2$).
    \end{itemize}
    \textbf{守恒律}: 在只有保守力做功时,机械能守恒。
    \tcblower
    \textbf{[EN] Definition}: 
    \begin{itemize}[leftmargin=*]
        \item \textbf{Work}: Integration of force over distance ($W = \vec{F} \cdot \vec{d}$).
        \item \textbf{Kinetic Energy}: Energy of motion ($E_k = \frac{1}{2}mv^2$).
        \item \textbf{Potential Energy}: Stored energy by position (Gravity $mgh$, Spring $\frac{1}{2}kx^2$).
    \end{itemize}
    \textbf{Conservation}: Total mechanical energy is constant if only conservative forces act.
\end{studybox}

% --------------------------------------------
% 4. Rigid Body
% --------------------------------------------
\begin{studybox}{刚体力学 / Rigid Body Mechanics}
    \textbf{[CN] 定义}: 
    描述具有形状大小物体的转动。
    \begin{itemize}[leftmargin=*]
        \item \textbf{力矩 (Torque)}: 使物体转动的"力" ($\tau = r \times F$).
        \item \textbf{转动惯量 (Moment of Inertia)}: 转动中的"质量" ($I = \sum mr^2$).
        \item \textbf{转动定律}: $\tau = I \alpha$ (类比 $F=ma$).
    \end{itemize}
    \tcblower
    \textbf{[EN] Definition}: 
    Describing rotation of extended bodies.
    \begin{itemize}[leftmargin=*]
        \item \textbf{Torque ($\tau$)}: The rotational equivalent of force ($\tau = r \times F$).
        \item \textbf{Moment of Inertia ($I$)}: Rotational mass, resistance to angular acceleration.
        \item \textbf{Newton's 2nd Law for Rotation}: $\tau = I \alpha$.
    \end{itemize}
\end{studybox}

% --------------------------------------------
% Thesis Connection
% --------------------------------------------
\begin{thesisbox}[论文关联 / Project Application]
    \textbf{[CN]}: 
    你的跌倒检测算法直接基于力学原理:
    \begin{itemize}
        \item \textbf{MPU6050 加速度计}: 测量 $\vec{F}/m$。静止时测量重力加速度 $g$。自由落体时读数为 0 (失重)。
        \item \textbf{冲击检测}: 跌倒撞击地面时,系统经历巨大的负加速度 (减速),导致读数飙升 ($a \gg g$)。
        \item \textbf{陀螺仪}: 测量角速度 $\omega$,辅助判断人体姿态变化。
    \end{itemize}

    \tcblower

    \textbf{[EN]}: 
    Your fall detection relies on mechanics:
    \begin{itemize}
        \item \textbf{Accelerometer}: Measures proper acceleration. Reads $1g$ when static. Reads $\approx 0g$ during free fall (weightlessness).
        \item \textbf{Impact}: Upon impact, the body decelerates rapidly, causing a spike in accelerometer reading ($a \gg g$).
        \item \textbf{Gyroscope}: Measures angular velocity $\omega$ to detect orientation changes.
    \end{itemize}
\end{thesisbox}

\begin{warnbox}[考试陷阱 / Exam Pitfalls]
    \begin{itemize}
        \item \textbf{Weight vs Mass}: 质量 ($m$) 是不变的,重量 ($mg$) 随重力场变化。
        \item \textbf{Centripetal Force}: 向心力不是一种独立的力,而是重力、张力等的合力。
        \item \textbf{Force Pairs}: 牛顿第三定律的作用力和反作用力作用在\textbf{不同}的物体上,永远不会互相抵消!
    \end{itemize}
\end{warnbox}
