% ============================================
% Topic 01: Mechanics / 力学
% Course: BE5B02PH1
% ============================================
\section{力学 / Mechanics (BE5B02PH1)}

% --------------------------------------------
% 1. Kinematics
% --------------------------------------------
\begin{studybox}{运动学 (Kinematics)}

\textbf{概念 (CN)}: 描述物体运动,不考虑力

\textbf{Term (EN)}: Position, Velocity, Acceleration, Trajectory

\tcblower

\textbf{基本量}:
\begin{itemize}[leftmargin=*]
    \item \textbf{位移 (Displacement)}: $\vec{r}(t)$ - 矢量
    \item \textbf{速度 (Velocity)}: $\vec{v} = \frac{d\vec{r}}{dt}$ - 位移的导数
    \item \textbf{加速度 (Acceleration)}: $\vec{a} = \frac{d\vec{v}}{dt}$ - 速度的导数
\end{itemize}

\textbf{匀加速运动}:
\begin{itemize}[leftmargin=*]
    \item 直线运动:$a = \text{const}$
    \item 抛体运动:$\vec{a} = -g\hat{j}$
    \item 圆周运动:向心加速度 $a_c = \frac{v^2}{r}$
\end{itemize}

\textbf{Key Insight}: Accelerometer measures acceleration; integrating twice gives position.

\end{studybox}

\begin{formulabox}
\textbf{匀加速运动公式}:
\begin{align}
    v &= v_0 + at \\
    x &= x_0 + v_0 t + \frac{1}{2}at^2 \\
    v^2 &= v_0^2 + 2a(x - x_0)
\end{align}

\textbf{圆周运动}:
\begin{align}
    v &= \omega r \quad \text{(线速度)} \\
    a_c &= \omega^2 r = \frac{v^2}{r} \quad \text{(向心加速度)}
\end{align}

\textbf{角加速度}: $\alpha = \frac{d\omega}{dt}$
\end{formulabox}

% --------------------------------------------
% 2. Newton's Laws
% --------------------------------------------
\begin{studybox}{牛顿运动定律 (Newton's Laws)}

\textbf{概念 (CN)}: 力与运动的基本关系

\textbf{Term (EN)}: Force, Mass, Inertia, Momentum

\tcblower

\textbf{三大定律}:

\textbf{第一定律 (惯性定律)}:
物体在没有外力作用时保持静止或匀速直线运动。

\textbf{第二定律 (运动定律)}:
\begin{equation}
    \boxed{\vec{F} = m\vec{a}}
\end{equation}

\textbf{第三定律 (作用-反作用)}:
作用力与反作用力大小相等、方向相反。
\begin{equation}
    \vec{F}_{12} = -\vec{F}_{21}
\end{equation}

\textbf{Key Insight}: $F = ma$ is the foundation of accelerometer-based motion detection.

\end{studybox}

\begin{formulabox}
\textbf{常见力}:

\textbf{重力}: $\vec{F}_g = m\vec{g}$, $g \approx 9.81 \, \text{m/s}^2$

\textbf{摩擦力}: $f = \mu N$ (静摩擦/动摩擦)

\textbf{弹簧力 (Hooke's Law)}: $F = -kx$

\textbf{向心力}: $F_c = m\frac{v^2}{r} = m\omega^2 r$

\textbf{动量}: $\vec{p} = m\vec{v}$

\textbf{动量定理}: $\vec{F} = \frac{d\vec{p}}{dt}$
\end{formulabox}

% --------------------------------------------
% 3. Energy and Work
% --------------------------------------------
\begin{studybox}{功与能 (Work and Energy)}

\textbf{概念 (CN)}: 能量是做功的能力

\textbf{Term (EN)}: Work, Kinetic Energy, Potential Energy, Conservation

\tcblower

\textbf{功 (Work)}:
\begin{equation}
    W = \int \vec{F} \cdot d\vec{r} = F \cdot d \cdot \cos\theta
\end{equation}

\textbf{动能 (Kinetic Energy)}:
\begin{equation}
    E_k = \frac{1}{2}mv^2
\end{equation}

\textbf{势能 (Potential Energy)}:
\begin{itemize}[leftmargin=*]
    \item 重力势能: $E_p = mgh$
    \item 弹性势能: $E_p = \frac{1}{2}kx^2$
\end{itemize}

\textbf{Key Insight}: Energy conservation helps analyze fall detection thresholds.

\end{studybox}

\begin{formulabox}
\textbf{功-能定理}: $W_{net} = \Delta E_k$

\textbf{机械能守恒} (仅保守力做功):
\begin{equation}
    E_k + E_p = \text{const}
\end{equation}

\textbf{功率 (Power)}:
\begin{equation}
    P = \frac{dW}{dt} = \vec{F} \cdot \vec{v}
\end{equation}

\textbf{自由落体能量}:
\begin{equation}
    mgh = \frac{1}{2}mv^2 \quad \Rightarrow \quad v = \sqrt{2gh}
\end{equation}
\end{formulabox}

% --------------------------------------------
% 4. Rigid Body Mechanics
% --------------------------------------------
\begin{studybox}{刚体力学 (Rigid Body Mechanics)}

\textbf{概念 (CN)}: 有体积和形状的物体的运动

\textbf{Term (EN)}: Torque, Moment of Inertia, Angular Momentum

\tcblower

\textbf{力矩 (Torque)}:
\begin{equation}
    \vec{\tau} = \vec{r} \times \vec{F}
\end{equation}

\textbf{转动惯量 (Moment of Inertia)}: 
\begin{equation}
    I = \sum m_i r_i^2 \quad \text{or} \quad I = \int r^2 dm
\end{equation}

\textbf{转动定律}:
\begin{equation}
    \vec{\tau} = I\vec{\alpha}
\end{equation}

\textbf{Key Insight}: Gyroscope measures angular velocity $\omega$ of rotation.

\end{studybox}

\begin{formulabox}
\textbf{常见转动惯量}:

\begin{center}
\begin{tabular}{lc}
\toprule
\textbf{物体} & \textbf{转动惯量 (绕中心轴)} \\
\midrule
细棒 (Rod) & $\frac{1}{12}ML^2$ \\
圆盘 (Disk) & $\frac{1}{2}MR^2$ \\
圆环 (Ring) & $MR^2$ \\
实心球 (Solid Sphere) & $\frac{2}{5}MR^2$ \\
\bottomrule
\end{tabular}
\end{center}

\textbf{平行轴定理}: $I = I_{cm} + Md^2$

\textbf{角动量}: $L = I\omega$

\textbf{角动量守恒}: $\tau_{ext} = 0 \Rightarrow L = \text{const}$
\end{formulabox}

% --------------------------------------------
% Thesis Connection
% --------------------------------------------
\begin{thesisbox}
\textbf{力学与可穿戴传感器}:

你的论文使用加速度计检测跌倒,直接应用了牛顿第二定律:

\textbf{加速度计原理}:
\begin{equation}
    F = ma \quad \Rightarrow \quad a = \frac{F}{m}
\end{equation}
MEMS 加速度计测量作用在悬浮质量块上的惯性力。

\textbf{跌倒检测}:
\begin{itemize}
    \item 自由落体阶段: $a \approx 0$ (失重)
    \item 撞击阶段: $a >> g$ (冲击)
\end{itemize}

\textbf{陀螺仪原理}:
测量角速度 $\omega$,基于科里奥利力:
\begin{equation}
    F_{Coriolis} = -2m(\vec{\omega} \times \vec{v})
\end{equation}

\textbf{Jan Koller 问题}: "How does your accelerometer work?"

\textbf{答案}: MEMS 加速度计使用悬浮质量块。当设备加速时,质量块由于惯性 ($F = ma$) 产生相对位移。这个位移改变电容,转换为电信号。它测量的是「真实加速度 + 重力」的叠加。
\end{thesisbox}

% --------------------------------------------
% Exam Strategy
% --------------------------------------------
\begin{warnbox}[[!] 考试陷阱 / Exam Pitfalls]
\begin{enumerate}
    \item \textbf{矢量方向}: 力、速度、加速度都是矢量,注意正负号!
    \item \textbf{圆周运动}: 向心力 $\neq$ 实际力,是合力的效果。
    \item \textbf{能量守恒}: 只有保守力(重力、弹力)做功时适用。
    \item \textbf{转动 vs 平动}: 转动用 $\tau = I\alpha$,平动用 $F = ma$。
\end{enumerate}
\end{warnbox}
