% ============================================
% Topic 01: Mechanics / 力学
% ============================================
\section{力学 / Mechanics}

% --------------------------------------------
% 1. Newton's Laws
% --------------------------------------------
\begin{studybox}{牛顿运动定律 (Newton's Laws of Motion)}

\textbf{概念 (CN)}: 经典力学的三大基本定律

\textbf{Term (EN)}: Newton's First, Second, and Third Laws

\tcblower

\textbf{第一定律 (惯性定律)}:
物体在不受外力时,保持静止或匀速直线运动。

\textbf{第二定律 (动力学基本方程)}:
物体的加速度与合外力成正比,与质量成反比。

\textbf{第三定律 (作用与反作用)}:
两物体间的作用力与反作用力大小相等、方向相反。

\textbf{Key Insight}: Newton's 2nd Law ($F=ma$) is the core equation for all dynamics. It links force, mass, and acceleration.

\end{studybox}

\begin{formulabox}
\textbf{牛顿第二定律}:
\begin{equation}
    \vec{F} = m\vec{a} = m\frac{d\vec{v}}{dt} = \frac{d\vec{p}}{dt}
\end{equation}
其中 $\vec{p} = m\vec{v}$ 是动量。

\textbf{冲量-动量定理}:
\begin{equation}
    \vec{J} = \int_{t_1}^{t_2} \vec{F} \, dt = \Delta\vec{p} = m\vec{v}_2 - m\vec{v}_1
\end{equation}

\textbf{动量守恒}:
\begin{equation}
    \sum \vec{p}_{before} = \sum \vec{p}_{after} \quad (\text{无外力时})
\end{equation}
\end{formulabox}

% --------------------------------------------
% 2. Work, Energy, Power
% --------------------------------------------
\begin{studybox}{功、能、功率 (Work, Energy, Power)}

\textbf{概念 (CN)}: 功是力沿位移方向的作用效果;能量是做功的能力;功率是做功的速率。

\textbf{Term (EN)}: Work, Kinetic Energy, Potential Energy, Power

\tcblower

\textbf{功 (Work)}:
\begin{itemize}[leftmargin=*]
    \item 恒力做功: $W = \vec{F} \cdot \vec{s} = Fs\cos\theta$
    \item 变力做功: $W = \int \vec{F} \cdot d\vec{s}$
\end{itemize}

\textbf{能量守恒}:
机械能 = 动能 + 势能(保守力场中守恒)

\textbf{Key Insight}: Energy is ALWAYS conserved; it transforms between forms (kinetic, potential, thermal).

\end{studybox}

\begin{formulabox}
\textbf{动能}:
\begin{equation}
    E_k = \frac{1}{2}mv^2
\end{equation}

\textbf{重力势能}:
\begin{equation}
    E_p = mgh
\end{equation}

\textbf{弹性势能}:
\begin{equation}
    E_p = \frac{1}{2}kx^2
\end{equation}

\textbf{功率}:
\begin{equation}
    P = \frac{dW}{dt} = \vec{F} \cdot \vec{v}
\end{equation}

\textbf{动能定理}:
\begin{equation}
    W_{net} = \Delta E_k = \frac{1}{2}mv_2^2 - \frac{1}{2}mv_1^2
\end{equation}
\end{formulabox}

% --------------------------------------------
% 3. Rotational Mechanics
% --------------------------------------------
\begin{studybox}{转动力学 (Rotational Mechanics)}

\textbf{概念 (CN)}: 刚体绕固定轴转动的运动学与动力学

\textbf{Term (EN)}: Moment of Inertia, Torque, Angular Momentum

\tcblower

\textbf{类比关系}:
\begin{center}
\begin{tabular}{cc}
\toprule
\textbf{平动} & \textbf{转动} \\
\midrule
质量 $m$ & 转动惯量 $I$ \\
力 $\vec{F}$ & 力矩 $\vec{\tau}$ \\
速度 $\vec{v}$ & 角速度 $\vec{\omega}$ \\
动量 $\vec{p}$ & 角动量 $\vec{L}$ \\
\bottomrule
\end{tabular}
\end{center}

\textbf{Key Insight}: Rotational mechanics mirrors linear mechanics—just replace mass with moment of inertia.

\end{studybox}

\begin{formulabox}
\textbf{转动惯量}:
\begin{equation}
    I = \sum_i m_i r_i^2 = \int r^2 \, dm
\end{equation}

\textbf{力矩与角加速度}:
\begin{equation}
    \vec{\tau} = I\vec{\alpha} = \vec{r} \times \vec{F}
\end{equation}

\textbf{角动量}:
\begin{equation}
    \vec{L} = I\vec{\omega} = \vec{r} \times \vec{p}
\end{equation}

\textbf{转动动能}:
\begin{equation}
    E_k = \frac{1}{2}I\omega^2
\end{equation}

\textbf{常见转动惯量}:
\begin{itemize}[leftmargin=*]
    \item 细杆(绕端点): $I = \frac{1}{3}mL^2$
    \item 圆盘(绕轴): $I = \frac{1}{2}mR^2$
    \item 球体(绕直径): $I = \frac{2}{5}mR^2$
\end{itemize}
\end{formulabox}

% --------------------------------------------
% 4. Simple Harmonic Motion
% --------------------------------------------
\begin{studybox}{简谐运动 (Simple Harmonic Motion)}

\textbf{概念 (CN)}: 物体在回复力作用下,沿直线往复运动的周期性运动

\textbf{Term (EN)}: SHM, Angular Frequency, Phase, Amplitude

\tcblower

\textbf{特征}:
\begin{itemize}[leftmargin=*]
    \item 回复力: $F = -kx$(与位移成正比,方向相反)
    \item 运动方程: $x(t) = A\cos(\omega t + \phi)$
    \item 等时性: 周期与振幅无关
\end{itemize}

\textbf{Key Insight}: SHM is everywhere: pendulums, springs, crystal oscillators (in your ESP32!).

\end{studybox}

\begin{formulabox}
\textbf{运动方程}:
\begin{equation}
    x(t) = A\cos(\omega t + \phi)
\end{equation}

\textbf{速度}:
\begin{equation}
    v(t) = -A\omega\sin(\omega t + \phi)
\end{equation}

\textbf{加速度}:
\begin{equation}
    a(t) = -A\omega^2\cos(\omega t + \phi) = -\omega^2 x
\end{equation}

\textbf{角频率}:
\begin{equation}
    \omega = \sqrt{\frac{k}{m}}, \quad T = 2\pi\sqrt{\frac{m}{k}}
\end{equation}

\textbf{机械能}:
\begin{equation}
    E = \frac{1}{2}kA^2 = \frac{1}{2}m\omega^2A^2
\end{equation}
\end{formulabox}

% --------------------------------------------
% Thesis Connection
% --------------------------------------------
\begin{thesisbox}
\textbf{跌倒检测与力学}:

你的论文使用 \textbf{MPU-6050 加速度计/陀螺仪} 进行跌倒检测。

\textbf{牛顿第二定律应用}:
加速度计测量的是 $a = F/m$,即施加在传感器上的力。

\textbf{跌倒特征}:
\begin{enumerate}
    \item \textbf{失重阶段}: 自由落体时加速度接近 0g(失重)
    \item \textbf{碰撞阶段}: 撞击地面时加速度峰值可达 3-5g
    \item \textbf{静止阶段}: 角速度 $\omega \approx 0$(躺在地上不动)
\end{enumerate}

\textbf{Jan Koller 问题}: "Why use Acc + Gyro?"

\textbf{答案}: 加速度检测冲击力,陀螺仪检测姿态变化。两者融合 (Sensor Fusion) 才能区分"跌倒"和"快速坐下"。
\end{thesisbox}

% --------------------------------------------
% Exam Strategy
% --------------------------------------------
\begin{warnbox}[🔴 考试陷阱 / Exam Pitfalls]
\begin{enumerate}
    \item \textbf{向量方向}: $\vec{F}$、$\vec{a}$、$\vec{\tau}$ 都是向量,注意方向!
    \item \textbf{能量符号}: 动能永远 $\geq 0$,势能可正可负(取决于参考点)。
    \item \textbf{SHM 相位}: $\phi = 0$ 表示从最大位移开始;$\phi = \pi/2$ 表示从平衡点开始。
\end{enumerate}
\end{warnbox}
