% ============================================================================
% Topic 01: Mechanics / 力学
% Blackboard + Mnemonic Style for SZZ Exam
% GRADE: D (retake) | PRIORITY: [CAUTION]
% ============================================================================

\section{Mechanics / 力学}
\label{sec:mechanics}

\begin{warnbox}[\textcolor{orange}{\textbf{[CAUTION]}} GRADE D: FOCUS MODE / 成绩D:专注模式]
\textbf{[CN]} 物理1补考通过 (D)。重点掌握 $F=ma$ 和SHM公式,\textbf{关联论文}:加速度计原理!\\
\textbf{[EN]} Physics 1 passed on retake (D). Focus on $F=ma$ and SHM. \textbf{Bridge to Thesis}: Accelerometer principle!\\
\textbf{Thesis Link}: MPU6050 测量 $a$,用于跌倒检测中的冲击识别。
\end{warnbox}

% ----------------------------------------------------------------------------
% 1. NEWTON'S SECOND LAW / 牛顿第二定律
% ----------------------------------------------------------------------------

\begin{defbox}{Newton's Second Law / 牛顿第二定律}
\textbf{公式 / Formula}:
\[
\mathbf{F} = m\mathbf{a} \quad \text{或 / or} \quad \mathbf{F} = \frac{d\mathbf{p}}{dt}
\]

\textbf{变量说明 / Variables}:
\begin{itemize}
    \item \textbf{[CN]} $\mathbf{F}$ --- 合外力 [N] / \textbf{[EN]} Net Force [N]
    \item \textbf{[CN]} $m$ --- 质量 [kg] / \textbf{[EN]} Mass [kg]
    \item \textbf{[CN]} $\mathbf{a}$ --- 加速度 [m/s$^2$] / \textbf{[EN]} Acceleration [m/s$^2$]
    \item \textbf{[CN]} $\mathbf{p} = m\mathbf{v}$ --- 动量 [kg$\cdot$m/s] / \textbf{[EN]} Momentum [kg$\cdot$m/s]
\end{itemize}

\textbf{Mnemonic / 记忆口诀}: 
\textbf{[CN]} 「力是加速度的老板」--- 力越大,加速度越大;质量越大,越难推动。\\
\textbf{[EN]} ``Force is the boss of acceleration'' --- more force means more acceleration; more mass means harder to push.

\tcblower
\textbf{[EN] Full Mirror}: Newton's Second Law relates force to the rate of change of momentum. For constant mass: $F = ma$. Force causes acceleration inversely proportional to mass. Key insight: mass resists acceleration (inertia).
\end{defbox}

\begin{studybox}{Blackboard Challenge: Newton's Second Law / 牛顿第二定律计算}
\textbf{[CN] 场景}: 「一个 $m = 2$ kg 的物体在光滑水平面上,受到 $F = 10$ N 的水平推力。求加速度和 $t = 3$ s 后的速度(初速度为零)。」

\textbf{Solution / 解答}:

\textbf{Step 1 / 步骤1}: 
\textbf{[CN]} 应用牛顿第二定律求加速度\\
\textbf{[EN]} Apply Newton's Second Law to find acceleration:
\[
F = ma \quad \Rightarrow \quad a = \frac{F}{m} = \frac{10}{2} = 5 \text{ m/s}^2
\]

\textbf{Step 2 / 步骤2}: 
\textbf{[CN]} 使用运动学公式求速度\\
\textbf{[EN]} Use kinematics equation to find velocity:
\[
v = v_0 + at = 0 + 5 \times 3 = 15 \text{ m/s}
\]

\textbf{Step 3 / 步骤3 (验算/Verify)}: 
\textbf{[CN]} 计算位移验证\\
\textbf{[EN]} Calculate displacement to verify:
\[
s = v_0 t + \frac{1}{2}at^2 = 0 + \frac{1}{2} \times 5 \times 9 = 22.5 \text{ m}
\]

\textbf{Final Answer / 最终答案}:
\[
\boxed{a = 5 \text{ m/s}^2, \quad v(3\text{s}) = 15 \text{ m/s}}
\]

\textbf{Mnemonic / 记忆口诀}: 
\textbf{[CN]} 「F除m得a,v等v0加at」--- 牛二定律和运动学公式联合使用。\\
\textbf{[EN]} ``F divided by m gives a, v equals v0 plus at'' --- Newton's Law + kinematics combined.

\tcblower
\textbf{[EN] Full Mirror}:
\begin{itemize}
    \item \textbf{Step 1}: Apply $F = ma$ to solve for $a = F/m$
    \item \textbf{Step 2}: Use $v = v_0 + at$ to find final velocity
    \item \textbf{Step 3}: Verify with $s = v_0 t + \frac{1}{2}at^2$
\end{itemize}
\textbf{Key insight}: Kinematics + dynamics work together in mechanics problems.
\end{studybox}

% ----------------------------------------------------------------------------
% 2. SIMPLE HARMONIC MOTION / 简谐运动
% ----------------------------------------------------------------------------

\begin{defbox}{Simple Harmonic Motion (SHM) / 简谐运动}
\textbf{[CN]} 弹簧-质量系统的运动方程:\\
\textbf{[EN]} Equation of motion for spring-mass system:
\[
m\frac{d^2x}{dt^2} = -kx \quad \Rightarrow \quad \frac{d^2x}{dt^2} + \omega^2 x = 0
\]

\textbf{角频率 / Angular Frequency}:
\[
\omega = \sqrt{\frac{k}{m}}
\]

\textbf{周期 / Period}:
\[
T = \frac{2\pi}{\omega} = 2\pi\sqrt{\frac{m}{k}}
\]

\textbf{通解 / General Solution}:
\[
x(t) = A\cos(\omega t + \phi)
\]

\textbf{Mnemonic / 记忆口诀}: 
\textbf{[CN]} 「k大弹得快,m大晃得慢」--- 弹簧越硬($k$大)频率越高,质量越大频率越低。\\
\textbf{[EN]} ``Stiffer spring = faster, heavier mass = slower'' --- $k$ up means $\omega$ up; $m$ up means $\omega$ down.

\tcblower
\textbf{[EN] Full Mirror}: SHM occurs when restoring force is proportional to displacement ($F = -kx$). Angular frequency $\omega = \sqrt{k/m}$ determines how fast oscillation occurs. Period $T = 2\pi/\omega$ is time for one complete cycle.
\end{defbox}

\begin{formulabox}{SHM Key Relationships / 简谐运动关键公式}
\textbf{位移、速度、加速度 / Displacement, Velocity, Acceleration}:
\begin{align*}
x(t) &= A\cos(\omega t + \phi) \\
v(t) &= -A\omega\sin(\omega t + \phi) \\
a(t) &= -A\omega^2\cos(\omega t + \phi) = -\omega^2 x
\end{align*}

\textbf{最大值 / Maximum Values}:
\begin{itemize}
    \item \textbf{[CN]} 最大位移:$x_{max} = A$(振幅)/ \textbf{[EN]} Max displacement: $x_{max} = A$ (amplitude)
    \item \textbf{[CN]} 最大速度:$v_{max} = A\omega$(平衡点处)/ \textbf{[EN]} Max velocity: $v_{max} = A\omega$ (at equilibrium)
    \item \textbf{[CN]} 最大加速度:$a_{max} = A\omega^2$(端点处)/ \textbf{[EN]} Max acceleration: $a_{max} = A\omega^2$ (at extremes)
\end{itemize}

\textbf{Mnemonic / 记忆口诀}: 
\textbf{[CN]} 「位余速正加反位」--- 位移用cos,速度用sin(差90度),加速度和位移反相。\\
\textbf{[EN]} ``x is cos, v is sin, a is opposite to x'' --- velocity leads by 90°, acceleration opposes displacement.

\tcblower
\textbf{[EN] Full Mirror}: Velocity leads displacement by 90° (phase). Acceleration is 180° out of phase with displacement (always points toward equilibrium). Max velocity occurs at equilibrium; max acceleration occurs at turning points.
\end{formulabox}

\begin{studybox}{Blackboard Challenge: Spring-Mass SHM / 弹簧振子}
\textbf{[CN] 场景}: 「一个 $m = 0.5$ kg 的物体连接在 $k = 200$ N/m 的弹簧上。求角频率、周期和频率。若振幅 $A = 0.1$ m,求最大速度和最大加速度。」

\textbf{Solution / 解答}:

\textbf{Step 1 / 步骤1}: 
\textbf{[CN]} 计算角频率\\
\textbf{[EN]} Calculate angular frequency:
\[
\omega = \sqrt{\frac{k}{m}} = \sqrt{\frac{200}{0.5}} = \sqrt{400} = 20 \text{ rad/s}
\]

\textbf{Step 2 / 步骤2}: 
\textbf{[CN]} 计算周期和频率\\
\textbf{[EN]} Calculate period and frequency:
\[
T = \frac{2\pi}{\omega} = \frac{2\pi}{20} = \frac{\pi}{10} \approx 0.314 \text{ s}
\]
\[
f = \frac{1}{T} = \frac{10}{\pi} \approx 3.18 \text{ Hz}
\]

\textbf{Step 3 / 步骤3}: 
\textbf{[CN]} 计算最大速度\\
\textbf{[EN]} Calculate maximum velocity:
\[
v_{max} = A\omega = 0.1 \times 20 = 2 \text{ m/s}
\]

\textbf{Step 4 / 步骤4}: 
\textbf{[CN]} 计算最大加速度\\
\textbf{[EN]} Calculate maximum acceleration:
\[
a_{max} = A\omega^2 = 0.1 \times 400 = 40 \text{ m/s}^2
\]

\textbf{Final Answer / 最终答案}:
\[
\boxed{\omega = 20 \text{ rad/s}, \quad T \approx 0.314 \text{ s}, \quad v_{max} = 2 \text{ m/s}, \quad a_{max} = 40 \text{ m/s}^2}
\]

\textbf{Mnemonic / 记忆口诀}: 「根号k除m」--- 角频率公式简单直接,周期是角频率的倒数乘 $2\pi$。

\tcblower
\textbf{[EN]}: $\omega = \sqrt{k/m}$, then $T = 2\pi/\omega$, $v_{max} = A\omega$, $a_{max} = A\omega^2$. These are the four key quantities for SHM.
\end{studybox}

% ----------------------------------------------------------------------------
% 3. ENERGY CONSERVATION / 能量守恒
% ----------------------------------------------------------------------------

\begin{defbox}{Energy Conservation / 能量守恒}
\textbf{机械能守恒条件}:只有保守力做功时,机械能守恒。

\[
E_{total} = KE + PE = \text{constant}
\]

\textbf{动能 / Kinetic Energy}:
\[
KE = \frac{1}{2}mv^2
\]

\textbf{势能 / Potential Energy}:
\begin{itemize}
    \item 重力势能:$PE_g = mgh$
    \item 弹性势能:$PE_s = \frac{1}{2}kx^2$
\end{itemize}

\textbf{Mnemonic / 记忆口诀}: 「动势互换总不变」--- 动能和势能相互转化,总能量不变。

\tcblower
\textbf{[EN]}: When only conservative forces act, mechanical energy is conserved: $KE + PE = $ constant. Energy transforms between kinetic and potential forms.
\end{defbox}

\begin{studybox}{Blackboard Challenge: Energy Conservation / 能量守恒}
\textbf{[CN] 场景}: 「一个 $m = 1$ kg 的球从 $h = 5$ m 高处自由落下。求落地时的速度(忽略空气阻力,$g = 10$ m/s$^2$)。」

\textbf{Solution / 解答}:

\textbf{Step 1}: 写出初态能量(顶部,$v_0 = 0$)
\[
E_i = KE_i + PE_i = 0 + mgh = 1 \times 10 \times 5 = 50 \text{ J}
\]

\textbf{Step 2}: 写出末态能量(底部,$h = 0$)
\[
E_f = KE_f + PE_f = \frac{1}{2}mv^2 + 0
\]

\textbf{Step 3}: 应用能量守恒
\[
E_i = E_f \quad \Rightarrow \quad mgh = \frac{1}{2}mv^2
\]

\textbf{Step 4}: 解出速度
\[
gh = \frac{1}{2}v^2 \quad \Rightarrow \quad v = \sqrt{2gh} = \sqrt{2 \times 10 \times 5} = \sqrt{100} = 10 \text{ m/s}
\]

\textbf{Final Answer / 最终答案}:
\[
\boxed{v = 10 \text{ m/s}}
\]

\textbf{Mnemonic / 记忆口诀}: 「根号2gh」--- 自由落体末速度公式,能量守恒直接得出。

\tcblower
\textbf{[EN]}: Energy conservation: $mgh = \frac{1}{2}mv^2$, so $v = \sqrt{2gh}$. Mass cancels out, only height and gravity matter.
\end{studybox}

\begin{studybox}{Blackboard Challenge: Spring Energy / 弹簧能量}
\textbf{[CN] 场景}: 「弹簧振子在 SHM 中,证明总能量恒定,并求任意位置的动能和势能比例。」

\textbf{Solution / 解答}:

\textbf{Step 1}: 写出总能量(设 $x = A\cos(\omega t)$)
\[
E = KE + PE = \frac{1}{2}mv^2 + \frac{1}{2}kx^2
\]

\textbf{Step 2}: 代入 $v = -A\omega\sin(\omega t)$ 和 $\omega^2 = k/m$
\begin{align*}
KE &= \frac{1}{2}m(A\omega)^2\sin^2(\omega t) = \frac{1}{2}kA^2\sin^2(\omega t) \\
PE &= \frac{1}{2}kA^2\cos^2(\omega t)
\end{align*}

\textbf{Step 3}: 求和
\[
E = \frac{1}{2}kA^2(\sin^2 + \cos^2) = \frac{1}{2}kA^2 = \text{constant}
\]

\textbf{Step 4}: 在 $x = A/2$ 处的比例
\[
PE = \frac{1}{2}k\left(\frac{A}{2}\right)^2 = \frac{1}{8}kA^2 = \frac{E}{4}
\]
\[
KE = E - PE = \frac{1}{2}kA^2 - \frac{1}{8}kA^2 = \frac{3}{8}kA^2 = \frac{3E}{4}
\]

\textbf{Final Answer / 最终答案}:
\[
\boxed{E_{total} = \frac{1}{2}kA^2, \quad \text{At } x = A/2: \quad KE:PE = 3:1}
\]

\textbf{Mnemonic / 记忆口诀}: 「半振幅三比一」--- 位移为振幅一半时,动能是势能的三倍。

\tcblower
\textbf{[EN]}: Total SHM energy is $\frac{1}{2}kA^2$. At half amplitude, $KE:PE = 3:1$ because PE scales as $x^2$.
\end{studybox}

\begin{warnbox}{Common Mistakes / 常见错误}
\begin{enumerate}
    \item \textbf{忘记平方}:动能 $\frac{1}{2}mv^2$,弹性势能 $\frac{1}{2}kx^2$,都有平方
    \item \textbf{单位混乱}:确保用 SI 单位(kg, m, s, N)
    \item \textbf{参考点选择}:势能需要选定参考点($h=0$ 或 $x=0$)
    \item \textbf{非保守力}:有摩擦时机械能不守恒,能量损失为热
\end{enumerate}

\tcblower
\textbf{[EN]}: Remember the squares in energy formulas. Always use SI units. Choose a reference point for PE. Mechanical energy is NOT conserved with friction.
\end{warnbox}

% ----------------------------------------------------------------------------
% 4. THESIS APPLICATION / 论文应用
% ----------------------------------------------------------------------------

\begin{thesisbox}{Thesis Application: Accelerometer and Fall Detection / 论文应用:加速度计与跌倒检测}
\textbf{[CN] 与论文的联系}:

力学原理在 IoT 跌倒检测系统中的核心应用:

\textbf{1. 加速度计原理 (MEMS Accelerometer)}

加速度计内部是一个微型弹簧-质量系统:
\[
F = ma = kx \quad \Rightarrow \quad a = \frac{k}{m}x
\]
通过测量位移 $x$ 即可得到加速度 $a$。

\textbf{2. 跌倒检测算法}

基于加速度特征的检测:
\begin{itemize}
    \item \textbf{自由落体相}:$|\mathbf{a}| \approx 0$(失重)
    \item \textbf{撞击相}:$|\mathbf{a}| > 3g$(大冲击)
    \item \textbf{静止相}:$|\mathbf{a}| \approx g$(躺在地上)
\end{itemize}

\textbf{加速度矢量幅值}:
\[
|\mathbf{a}| = \sqrt{a_x^2 + a_y^2 + a_z^2}
\]

\textbf{3. 能量分析}

跌倒过程的能量转化:
\[
mgh \rightarrow \frac{1}{2}mv^2 \rightarrow \text{Impact Energy}
\]
撞击力估算:$F_{impact} = \frac{\Delta p}{\Delta t} = \frac{mv}{\Delta t}$

\textbf{4. ESP32 实现}
\begin{itemize}
    \item MPU6050/ADXL345 加速度计采样率:100-500 Hz
    \item 实时计算加速度幅值和变化率
    \item 阈值判断 + 状态机检测跌倒
\end{itemize}

\textbf{Mnemonic / 记忆口诀}: 「零-大-稳」--- 跌倒三相:自由落体零加速度,撞击大加速度,躺下稳定加速度。

\tcblower
\textbf{[EN]}: MEMS accelerometers use spring-mass principles. Fall detection analyzes acceleration magnitude through three phases: free-fall ($|a| \approx 0$), impact ($|a| > 3g$), and rest ($|a| \approx g$). Understanding mechanics is essential for designing reliable fall detection algorithms on ESP32.
\end{thesisbox}

% ----------------------------------------------------------------------------
% 5. CENTRIPETAL FORCE / 向心力
% ----------------------------------------------------------------------------

\begin{defbox}{Centripetal Force / 向心力}
物体做圆周运动时,指向圆心的力:
\[
F_c = \frac{mv^2}{r} = m\omega^2 r = m \cdot a_c
\]

其中:
\begin{itemize}
    \item $v$ --- 线速度 (Linear velocity) [m/s]
    \item $r$ --- 圆周半径 (Radius) [m]
    \item $\omega$ --- 角速度 (Angular velocity) [rad/s]
    \item $a_c = v^2/r = \omega^2 r$ --- 向心加速度
\end{itemize}

\textbf{Mnemonic / 记忆口诀}: 「向心力是拉绳的手」--- 没有向心力,物体会沿切线飞出。

\tcblower
\textbf{[EN]}: Centripetal force keeps an object moving in a circle. Without it, the object would fly off tangentially (Newton's 1st Law).
\end{defbox}

\begin{studybox}{Blackboard Challenge: Centripetal Acceleration / 向心加速度}
\textbf{[CN] 场景}: 「人转身时手臂长$r=0.6m$,角速度$\omega=3$ rad/s。求手腕处的向心加速度。」

\textbf{Solution / 解答}:
\begin{align}
    a_c &= \omega^2 r = (3)^2 \times 0.6 = 9 \times 0.6 = \boxed{5.4 \text{ m/s}^2}
\end{align}

\textbf{比较}: $a_c = 5.4$ m/s$^2 \approx 0.55g$

这个加速度会被手腕上的加速度计检测到!

\tcblower
\textbf{[EN]}: Centripetal acceleration = $\omega^2 r$. This contributes to accelerometer reading during rotation.
\end{studybox}

\begin{thesisbox}{Thesis Bridge: Gyroscope + Rotation / 陀螺仪与旋转}
\textbf{[CN]}: 这在跌倒检测中很重要...

\textbf{Script}: ``During a fall, the body rotates. My MPU6050 measures:
\begin{itemize}
    \item \textbf{Accelerometer}: Linear acceleration + Centripetal acceleration + Gravity
    \item \textbf{Gyroscope}: Angular velocity $\omega$ in rad/s
\end{itemize}

\textbf{Problem}: Accelerometer reading is contaminated by rotation:
\[
\mathbf{a}_{measured} = \mathbf{a}_{linear} + \boldsymbol{\omega} \times (\boldsymbol{\omega} \times \mathbf{r}) + \mathbf{g}
\]

\textbf{Solution}: Use gyroscope to calculate and subtract centripetal component:
\[
a_{centripetal} = \omega^2 r
\]

For typical human fall: $\omega \approx 2-5$ rad/s, $r \approx 0.5m$ $\Rightarrow$ $a_c \approx 2-12$ m/s$^2$

This is significant and must be accounted for!''
\end{thesisbox}

% ----------------------------------------------------------------------------
% 6. IMPULSE AND MOMENTUM / 冲量与动量
% ----------------------------------------------------------------------------

\begin{defbox}{Impulse-Momentum Theorem / 动量定理}
\[
\mathbf{J} = \int \mathbf{F} \, dt = \Delta \mathbf{p} = m\mathbf{v}_f - m\mathbf{v}_i
\]

对于恒力:
\[
\mathbf{F} \cdot \Delta t = m \Delta \mathbf{v}
\]

\textbf{Mnemonic / 记忆口诀}: 「力乘时间改变动量」--- 越大的力作用越久,动量改变越大。

\tcblower
\textbf{[EN]}: Impulse equals change in momentum. Key for understanding collisions and impacts.
\end{defbox}

\begin{studybox}{Blackboard Challenge: Fall Impact Force / 跌倒撞击力}
\textbf{[CN] 场景}: 「一个70kg的人从1m高度跌倒,撞击地面时间0.1s。估算平均撞击力。」

\textbf{Solution / 解答}:

\textbf{Step 1}: 计算落地速度(能量守恒)
\[
v = \sqrt{2gh} = \sqrt{2 \times 10 \times 1} = \sqrt{20} \approx 4.47 \text{ m/s}
\]

\textbf{Step 2}: 计算动量变化
\[
\Delta p = mv - 0 = 70 \times 4.47 = 313 \text{ kg}\cdot\text{m/s}
\]

\textbf{Step 3}: 计算平均撞击力
\[
F_{avg} = \frac{\Delta p}{\Delta t} = \frac{313}{0.1} = \boxed{3130 \text{ N}}
\]

\textbf{加速度}: $a = F/m = 3130/70 = 44.7$ m/s$^2 \approx \boxed{4.5g}$

\tcblower
\textbf{[EN]}: Impact force depends on stopping time. Shorter $\Delta t$ = higher force. This is why padding helps!
\end{studybox}

\begin{thesisbox}{Thesis Bridge: Fall Detection Threshold / 跌倒检测阈值}
\textbf{[CN]}: 这是我论文算法的核心参数...

\textbf{Script}: ``Based on physics calculation:
\begin{itemize}
    \item Typical fall impact: 3-6g
    \item Sitting down hard: 1-2g
    \item Normal walking: < 1g variation
\end{itemize}

\textbf{My detection thresholds}:
\begin{enumerate}
    \item \textbf{Free-fall phase}: $|a| < 0.4g$ for > 200ms
    \item \textbf{Impact phase}: $|a| > 2.5g$ peak
    \item \textbf{Rest phase}: $|a| \approx 1g$ and lying orientation
\end{enumerate}

\textbf{Why 2.5g not 4g?}
\begin{itemize}
    \item Elderly falls are often slower, less violent
    \item Want sensitivity (catch more falls) over specificity
    \item False alarms handled by additional checks (orientation, time)
\end{itemize}''
\end{thesisbox}

% ----------------------------------------------------------------------------
% 7. MEMS SENSOR DETAILED PHYSICS / MEMS传感器物理详解
% ----------------------------------------------------------------------------

\begin{studybox}{Blackboard Challenge: MEMS Accelerometer Design / MEMS加速度计设计}
\textbf{[CN] 场景}: 「MEMS加速度计的proof mass为10$\mu$g,弹簧常数$k$设计为测量$\pm$16g范围。求$k$值和最大位移。」

\textbf{Solution / 解答}:

\textbf{Step 1}: 单位转换
\begin{itemize}
    \item $m = 10\mu g = 10 \times 10^{-9} kg = 10^{-8} kg$
    \item $a_{max} = 16g = 16 \times 9.8 = 156.8$ m/s$^2$
\end{itemize}

\textbf{Step 2}: 计算最大力
\[
F_{max} = m \cdot a_{max} = 10^{-8} \times 156.8 = 1.57 \times 10^{-6} \text{ N} = 1.57 \mu N
\]

\textbf{Step 3}: 设计位移(典型MEMS位移约1$\mu$m满量程)
\[
x_{max} = 1 \mu m = 10^{-6} m
\]

\textbf{Step 4}: 计算弹簧常数
\[
k = \frac{F_{max}}{x_{max}} = \frac{1.57 \times 10^{-6}}{10^{-6}} = \boxed{1.57 \text{ N/m}}
\]

\textbf{Step 5}: 验证谐振频率
\[
f_0 = \frac{1}{2\pi}\sqrt{\frac{k}{m}} = \frac{1}{2\pi}\sqrt{\frac{1.57}{10^{-8}}} = \frac{1}{2\pi} \times 12530 \approx \boxed{2 \text{ kHz}}
\]

\tcblower
\textbf{[EN]}: MEMS design uses F=ma and Hooke's Law. Resonant frequency should be >> signal bandwidth (human motion < 20Hz).
\end{studybox}

\begin{thesisbox}{Thesis Bridge: Why MEMS Works for Fall Detection / MEMS适用于跌倒检测}
\textbf{[CN]}: 这解释了我的传感器选择...

\textbf{Script}: ``The MPU6050 MEMS accelerometer is ideal because:
\begin{enumerate}
    \item \textbf{Bandwidth}: 2kHz resonance >> 20Hz human motion (no aliasing)
    \item \textbf{Range}: $\pm$16g covers worst-case fall impact
    \item \textbf{Resolution}: 16-bit ADC = 0.5mg per LSB (sensitive enough)
    \item \textbf{Power}: <1mW (battery-powered wearable)
    \item \textbf{Cost}: <\$5 (mass deployment feasible)
\end{enumerate}

\textbf{Physics insight}:
\[
\text{Signal BW} \ll f_0 \Rightarrow \text{Flat frequency response (no resonance effects)}
\]

If human motion were at 1kHz (it's not!), I would need a different sensor.''
\end{thesisbox}

\begin{formulabox}{Key Formulas Summary / 关键公式总结}
\begin{enumerate}
    \item \textbf{Newton's 2nd Law}: $F = ma$
    \item \textbf{SHM Angular Frequency}: $\omega = \sqrt{k/m}$
    \item \textbf{SHM Period}: $T = 2\pi\sqrt{m/k}$
    \item \textbf{Kinetic Energy}: $KE = \frac{1}{2}mv^2$
    \item \textbf{Gravitational PE}: $PE = mgh$
    \item \textbf{Elastic PE}: $PE = \frac{1}{2}kx^2$
    \item \textbf{Free Fall Velocity}: $v = \sqrt{2gh}$
    \item \textbf{SHM Total Energy}: $E = \frac{1}{2}kA^2$
\end{enumerate}

\tcblower
\textbf{[EN]}: Master these eight formulas. They cover force, motion, and energy --- the three pillars of mechanics.
\end{formulabox}
