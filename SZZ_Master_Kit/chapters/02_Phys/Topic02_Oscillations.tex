% ============================================
% Topic 02: Oscillations / 振动
% Course: BE5B02PH1
% ============================================
\section{振动 / Oscillations (BE5B02PH1)}

% --------------------------------------------
% 1. Simple Harmonic Motion
% --------------------------------------------
\begin{studybox}{简谐运动 (Simple Harmonic Motion)}

\textbf{概念 (CN)}: 回复力与位移成正比的周期运动

\textbf{Term (EN)}: SHM, Angular Frequency, Amplitude, Phase

\tcblower

\textbf{定义}: 物体在回复力 $F = -kx$ 作用下的运动

\textbf{运动方程}:
\begin{equation}
    m\frac{d^2x}{dt^2} = -kx \quad \Rightarrow \quad \frac{d^2x}{dt^2} + \omega^2 x = 0
\end{equation}

\textbf{解}:
\begin{equation}
    x(t) = A\cos(\omega t + \phi)
\end{equation}

其中:
\begin{itemize}[leftmargin=*]
    \item $A$: 振幅 (Amplitude)
    \item $\omega = \sqrt{k/m}$: 角频率 (Angular Frequency)
    \item $\phi$: 初相位 (Initial Phase)
\end{itemize}

\textbf{Key Insight}: Crystal oscillator in ESP32 uses mechanical resonance for timing.

\end{studybox}

\begin{formulabox}
\textbf{简谐运动公式}:

\textbf{位移}: $x = A\cos(\omega t + \phi)$

\textbf{速度}: $v = -A\omega\sin(\omega t + \phi)$

\textbf{加速度}: $a = -A\omega^2\cos(\omega t + \phi) = -\omega^2 x$

\textbf{周期}: $T = \frac{2\pi}{\omega} = 2\pi\sqrt{\frac{m}{k}}$

\textbf{频率}: $f = \frac{1}{T} = \frac{\omega}{2\pi}$

\textbf{能量}: $E = \frac{1}{2}kA^2 = \frac{1}{2}m\omega^2 A^2$
\end{formulabox}

% --------------------------------------------
% 2. Damped Oscillations
% --------------------------------------------
\begin{studybox}{阻尼振动 (Damped Oscillations)}

\textbf{概念 (CN)}: 有阻力作用的振动,振幅逐渐减小

\textbf{Term (EN)}: Damping, Damping Ratio, Overdamped, Underdamped, Critical Damping

\tcblower

\textbf{运动方程}:
\begin{equation}
    m\frac{d^2x}{dt^2} + b\frac{dx}{dt} + kx = 0
\end{equation}

\textbf{阻尼系数}: $\gamma = \frac{b}{2m}$

\textbf{阻尼比}: $\zeta = \frac{\gamma}{\omega_0}$, $\omega_0 = \sqrt{k/m}$

\textbf{三种情况}:
\begin{itemize}[leftmargin=*]
    \item $\zeta < 1$: 欠阻尼 (Underdamped) - 振荡衰减
    \item $\zeta = 1$: 临界阻尼 (Critical) - 最快回平衡
    \item $\zeta > 1$: 过阻尼 (Overdamped) - 缓慢回平衡
\end{itemize}

\textbf{Key Insight}: Control systems aim for critical damping for fastest response.

\end{studybox}

\begin{formulabox}
\textbf{欠阻尼解} ($\zeta < 1$):
\begin{equation}
    x(t) = Ae^{-\gamma t}\cos(\omega_d t + \phi)
\end{equation}
其中 $\omega_d = \omega_0\sqrt{1 - \zeta^2}$ 是阻尼角频率

\textbf{品质因数 (Q Factor)}:
\begin{equation}
    Q = \frac{\omega_0}{2\gamma} = \frac{1}{2\zeta}
\end{equation}
高 Q $\Rightarrow$ 低阻尼,振动持续更久

\textbf{能量衰减}:
\begin{equation}
    E(t) = E_0 e^{-2\gamma t}
\end{equation}
\end{formulabox}

% --------------------------------------------
% 3. Forced Oscillations and Resonance
% --------------------------------------------
\begin{studybox}{受迫振动与共振 (Forced Oscillations \& Resonance)}

\textbf{概念 (CN)}: 在周期性外力作用下的振动

\textbf{Term (EN)}: Driven Oscillator, Resonance, Resonance Frequency

\tcblower

\textbf{运动方程}:
\begin{equation}
    m\frac{d^2x}{dt^2} + b\frac{dx}{dt} + kx = F_0\cos(\omega_{drive} t)
\end{equation}

\textbf{稳态解}:
\begin{equation}
    x(t) = A(\omega_{drive})\cos(\omega_{drive} t - \psi)
\end{equation}

\textbf{共振}: 当 $\omega_{drive} \approx \omega_0$ 时,振幅最大!

\textbf{Key Insight}: Resonance can cause structural failures (Tacoma Bridge).

\end{studybox}

\begin{formulabox}
\textbf{振幅响应}:
\begin{equation}
    A(\omega) = \frac{F_0/m}{\sqrt{(\omega_0^2 - \omega^2)^2 + (2\gamma\omega)^2}}
\end{equation}

\textbf{共振频率}:
\begin{equation}
    \omega_{res} = \omega_0\sqrt{1 - 2\zeta^2} \approx \omega_0 \quad (\text{低阻尼})
\end{equation}

\textbf{共振时振幅}:
\begin{equation}
    A_{max} = \frac{F_0}{2m\gamma\omega_0} = \frac{QF_0}{k}
\end{equation}
\end{formulabox}

% --------------------------------------------
% Thesis Connection
% --------------------------------------------
\begin{thesisbox}
\textbf{振动与传感器系统}:

\textbf{晶振 (Crystal Oscillator)}:
ESP32 使用 40MHz 晶振作为时钟源。晶体的机械共振提供精确的频率参考:
\begin{equation}
    f = \frac{1}{2\pi}\sqrt{\frac{k}{m}}
\end{equation}

\textbf{MEMS 陀螺仪原理}:
陀螺仪使用振动质量块。当设备旋转时,科里奥利力改变振动模式。

\textbf{滤波器设计}:
RLC 电路是电子谐振器:
\begin{equation}
    f_0 = \frac{1}{2\pi\sqrt{LC}}
\end{equation}

\textbf{Jan Koller 问题}: "Why does your system need a precise clock?"

\textbf{答案}: 晶振利用石英晶体的机械共振提供稳定的频率参考。这对于 UART (115200 baud)、I2C 时序、WiFi 调制等都至关重要。晶体的高 Q 值确保频率稳定性。
\end{thesisbox}

% --------------------------------------------
% Exam Strategy
% --------------------------------------------
\begin{warnbox}[[!] 考试陷阱 / Exam Pitfalls]
\begin{enumerate}
    \item \textbf{角频率 vs 频率}: $\omega = 2\pi f$,不要混淆!
    \item \textbf{简谐运动能量}: 总能量 $E = \frac{1}{2}kA^2$ 是常数,动能和势能在转换。
    \item \textbf{阻尼比}: $\zeta = 1$ 是临界阻尼,不是无阻尼。
    \item \textbf{共振}: 共振频率略低于固有频率(当阻尼存在时)。
\end{enumerate}
\end{warnbox}
