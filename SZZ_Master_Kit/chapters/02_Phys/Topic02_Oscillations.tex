% ============================================
% Topic 02: Oscillations / 振动与波
% Course: BE5B02PH1
% ============================================
\section{振动与波 / Oscillations \& Waves (BE5B02PH1)}

% --------------------------------------------
% 1. Simple Harmonic Motion (SHM)
% --------------------------------------------
\begin{studybox}{简谐振动 / Simple Harmonic Motion (SHM)}
    \textbf{[CN] 定义}: 
    物体受到的回复力与位移成正比且方向相反的运动 ($F = -kx$)。
    \begin{itemize}[leftmargin=*]
        \item \textbf{方程}: $x(t) = A \cos(\omega t + \phi)$.
        \item \textbf{周期 (Period)}: $T = 2\pi / \omega$.
        \item \textbf{频率 (Frequency)}: $f = 1/T$.
    \end{itemize}
    \tcblower
    \textbf{[EN] Definition}: 
    Motion where the restoring force is directly proportional to the displacement and acts in the direction opposite to that of displacement ($F = -kx$).
    \begin{itemize}[leftmargin=*]
        \item \textbf{Equation}: $x(t) = A \cos(\omega t + \phi)$.
        \item \textbf{Angular Frequency}: $\omega = \sqrt{k/m}$.
    \end{itemize}
\end{studybox}

\begin{formulabox}
\textbf{单摆 / Simple Pendulum}:
\begin{equation}
    T = 2\pi \sqrt{\frac{L}{g}}, \quad \omega = \sqrt{\frac{g}{L}}
\end{equation}
(仅适用于小角度 $\theta \ll 1$ rad)

\textbf{弹簧振子 / Spring Mass System}:
\begin{equation}
    T = 2\pi \sqrt{\frac{m}{k}}, \quad \omega = \sqrt{\frac{k}{m}}
\end{equation}
\end{formulabox}

% --------------------------------------------
% 2. Damped and Driven Oscillations
% --------------------------------------------
\begin{studybox}{阻尼与驱动 / Damped \& Driven Oscillations}
    \textbf{[CN] 定义}: 
    \begin{itemize}[leftmargin=*]
        \item \textbf{阻尼 (Damping)}: 由于摩擦力导致能量耗散,振幅随时间指数衰减 ($A(t) = A_0 e^{-\beta t}$)。
        \item \textbf{驱动 (Driven)}: 受到周期性外力作用。当驱动频率接近固有频率时发生\textbf{共振 (Resonance)}。
    \end{itemize}
    \tcblower
    \textbf{[EN] Definition}: 
    \begin{itemize}[leftmargin=*]
        \item \textbf{Damped}: Amplitude decreases over time due to energy loss (friction/drag).
        \item \textbf{Driven}: External periodic force is applied. \textbf{Resonance} occurs when driving frequency matches natural frequency ($\omega \approx \omega_0$), causing maximum amplitude.
    \end{itemize}
\end{studybox}

% --------------------------------------------
% 3. Waves
% --------------------------------------------
\begin{studybox}{机械波 / Mechanical Waves}
    \textbf{[CN] 定义}: 
    振动在介质中的传播。
    \begin{itemize}[leftmargin=*]
        \item \textbf{横波 (Transverse)}: 质点振动垂直于传播方向 (如弦波、光波)。
        \item \textbf{纵波 (Longitudinal)}: 质点振动平行于传播方向 (如声波)。
    \end{itemize}
    \tcblower
    \textbf{[EN] Definition}: 
    Propagation of disturbances through a medium.
    \begin{itemize}[leftmargin=*]
        \item \textbf{Transverse}: Particle oscillation is perpendicular to wave propagation (e.g., Light, String).
        \item \textbf{Longitudinal}: Particle oscillation is parallel to wave propagation (e.g., Sound).
    \end{itemize}
\end{studybox}

\begin{formulabox}
\textbf{波函数 / Wave Equation}:
\begin{equation}
    y(x,t) = A \cos(kx - \omega t)
\end{equation}
$k = 2\pi/\lambda$ (波数 / Wave Number)

\textbf{波速 / Wave Speed}:
\begin{equation}
    v = \lambda f = \frac{\omega}{k}
\end{equation}
\end{formulabox}

% --------------------------------------------
% Thesis Connection
% --------------------------------------------
\begin{thesisbox}[论文关联 / Project Application]
    \textbf{[CN]}: 
    \begin{itemize}
        \item \textbf{MEMS 谐振器}: MPU6050 内部使用微小的振动质量块 (Vibrating Mass) 和电容检测。它们工作在谐振状态。Coriolis 力会改变振动模式。
        \item \textbf{信号滤波}: 我们的"移动平均滤波器"本质上是一个低通滤波器,用于去除高频振动 (噪音),只保留低频运动分量。
    \end{itemize}

    \tcblower

    \textbf{[EN]}: 
    \begin{itemize}
        \item \textbf{MEMS Resonator}: Inside MPU6050, vibrating masses operate at resonance. Coriolis forces shift this vibration, detected capacitively.
        \item \textbf{Filtering}: The moving average filter acts as a Low-Pass Filter, attenuating high-frequency oscillations (noise) while passing low-frequency motion.
    \end{itemize}
\end{thesisbox}

\begin{warnbox}[考试陷阱 / Exam Pitfalls]
    \begin{itemize}
        \item \textbf{Period vs Frequency}: $T = 1/f$. Don't mix them up.
        \item \textbf{Resonance}: 共振只发生在驱动频率 $\omega_d = \omega_0$ 时,此时能量传输最大,若无阻尼可能导致破坏。
        \item \textbf{Sound in Vacuum}: 声波是机械波,不能在真空中传播!(Light can).
    \end{itemize}
\end{warnbox}
