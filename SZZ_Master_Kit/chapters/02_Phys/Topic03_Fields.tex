% ============================================
% Topic 03: Physical Fields / 物理场
% Course: BE5B02PH1
% ============================================
\section{物理场 / Physical Fields (BE5B02PH1)}

% --------------------------------------------
% 1. Gravitational Field
% --------------------------------------------
\begin{studybox}{引力场 (Gravitational Field)}

\textbf{概念 (CN)}: 质量周围存在的场,对其他质量产生引力

\textbf{Term (EN)}: Gravitational Field, Newton's Law of Gravitation

\tcblower

\textbf{万有引力定律}:
\begin{equation}
    F = G\frac{m_1 m_2}{r^2}
\end{equation}
其中 $G = 6.674 \times 10^{-11} \, \text{N}\cdot\text{m}^2/\text{kg}^2$

\textbf{引力场强度}:
\begin{equation}
    \vec{g} = -G\frac{M}{r^2}\hat{r}
\end{equation}

\textbf{地表重力加速度}: $g = 9.81 \, \text{m/s}^2$

\textbf{Key Insight}: Accelerometer measures $g$ when stationary.

\end{studybox}

% --------------------------------------------
% 2. Electric Field
% --------------------------------------------
\begin{studybox}{电场 (Electric Field)}

\textbf{概念 (CN)}: 电荷周围存在的场,对其他电荷产生力

\textbf{Term (EN)}: Electric Field, Coulomb's Law, Superposition

\tcblower

\textbf{库仑定律}:
\begin{equation}
    F = k_e\frac{q_1 q_2}{r^2}, \quad k_e = \frac{1}{4\pi\varepsilon_0} \approx 8.99 \times 10^9 \, \text{N}\cdot\text{m}^2/\text{C}^2
\end{equation}

\textbf{电场强度}:
\begin{equation}
    \vec{E} = \frac{\vec{F}}{q} = k_e\frac{Q}{r^2}\hat{r}
\end{equation}

\textbf{电场叠加原理}: 多个电荷的电场矢量相加

\textbf{Key Insight}: Capacitive touch sensors detect changes in electric field.

\end{studybox}

\begin{formulabox}
\textbf{电势 (Electric Potential)}:
\begin{equation}
    V = \frac{U}{q} = k_e\frac{Q}{r}
\end{equation}

\textbf{电场与电势关系}:
\begin{equation}
    \vec{E} = -\nabla V = -\frac{dV}{dx}\hat{x}
\end{equation}

\textbf{点电荷势能}:
\begin{equation}
    U = k_e\frac{q_1 q_2}{r}
\end{equation}

\textbf{均匀电场中}:
\begin{equation}
    V = -\int \vec{E} \cdot d\vec{l} = Ed
\end{equation}
\end{formulabox}

% --------------------------------------------
% 3. Magnetic Field
% --------------------------------------------
\begin{studybox}{磁场 (Magnetic Field)}

\textbf{概念 (CN)}: 运动电荷或电流周围存在的场

\textbf{Term (EN)}: Magnetic Field, Lorentz Force, Biot-Savart Law

\tcblower

\textbf{洛伦兹力}:
\begin{equation}
    \vec{F} = q\vec{v} \times \vec{B}
\end{equation}

\textbf{电流受力}:
\begin{equation}
    \vec{F} = I\vec{L} \times \vec{B}
\end{equation}

\textbf{毕奥-萨伐尔定律}:
\begin{equation}
    d\vec{B} = \frac{\mu_0}{4\pi}\frac{I d\vec{l} \times \hat{r}}{r^2}
\end{equation}

\textbf{Key Insight}: Hall effect sensors measure magnetic fields for position sensing.

\end{studybox}

\begin{formulabox}
\textbf{常见磁场}:

\textbf{长直导线}: $B = \frac{\mu_0 I}{2\pi r}$

\textbf{螺线管中心}: $B = \mu_0 n I$ (n = 匝数/长度)

\textbf{圆环中心}: $B = \frac{\mu_0 I}{2R}$

\textbf{磁通量 (Magnetic Flux)}:
\begin{equation}
    \Phi_B = \int \vec{B} \cdot d\vec{A}
\end{equation}

\textbf{安培环路定律}:
\begin{equation}
    \oint \vec{B} \cdot d\vec{l} = \mu_0 I_{enc}
\end{equation}
\end{formulabox}

% --------------------------------------------
% 4. Electromagnetic Induction
% --------------------------------------------
\begin{studybox}{电磁感应 (Electromagnetic Induction)}

\textbf{概念 (CN)}: 变化的磁场产生电动势

\textbf{Term (EN)}: Faraday's Law, Lenz's Law, Induced EMF

\tcblower

\textbf{法拉第定律}:
\begin{equation}
    \mathcal{E} = -\frac{d\Phi_B}{dt}
\end{equation}

\textbf{楞次定律}: 感应电流的方向总是阻止产生它的磁通量变化

\textbf{自感 (Self-Inductance)}:
\begin{equation}
    \mathcal{E} = -L\frac{dI}{dt}
\end{equation}

\textbf{Key Insight}: Wireless charging uses electromagnetic induction.

\end{studybox}

% --------------------------------------------
% Thesis Connection
% --------------------------------------------
\begin{thesisbox}
\textbf{物理场与传感器}:

\textbf{电容式传感器 (Capacitive Sensors)}:
电场原理用于湿度传感器和触摸屏:
\begin{equation}
    C = \varepsilon_0 \varepsilon_r \frac{A}{d}
\end{equation}

\textbf{霍尔效应传感器 (Hall Effect)}:
检测磁场强度和方向:
\begin{equation}
    V_H = \frac{I B}{n e t}
\end{equation}

\textbf{电磁干扰 (EMI)}:
变化的磁场可能在导线中感应噪声电压(法拉第定律)。

\textbf{Jan Koller 问题}: "How do you minimize EMI in PCB design?"

\textbf{答案}: 
1. 使用差分信号(共模抑制)
2. 地平面提供磁场屏蔽
3. 减小环路面积减少磁通量耦合
4. 走线远离开关电源
\end{thesisbox}

% --------------------------------------------
% Exam Strategy
% --------------------------------------------
\begin{warnbox}[[!] 考试陷阱 / Exam Pitfalls]
\begin{enumerate}
    \item \textbf{符号}: 电场力 $F = qE$,磁场力 $F = qvB\sin\theta$。
    \item \textbf{磁力不做功}: 洛伦兹力垂直于速度,不改变动能。
    \item \textbf{楞次定律}: 感应电流的磁场方向反抗磁通量变化。
    \item \textbf{法拉第定律负号}: 表示楞次定律方向。
\end{enumerate}
\end{warnbox}
