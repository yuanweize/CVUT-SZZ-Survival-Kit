% ============================================
% Topic 03: Fields / 电磁场
% Course: BE5B02PH2
% ============================================
\section{电场与磁场 / Electric \& Magnetic Fields (BE5B02PH2)}

% --------------------------------------------
% 1. Electric Field
% --------------------------------------------
\begin{studybox}{静电场 / Electrostatics}
    \textbf{[CN] 定义}: 
    电荷周围存在的场,对其他电荷产生力。
    \begin{itemize}[leftmargin=*]
        \item \textbf{库仑定律 (Coulomb's Law)}: 点电荷间的力 $F = k \frac{q_1 q_2}{r^2}$.
        \item \textbf{电场强度 (E-Field)}: 单位电荷受到的力 $\vec{E} = \vec{F}/q$.
        \item \textbf{电势 (Potential)}: 单位电荷的电势能 $V = E_p/q$.
    \end{itemize}
    \tcblower
    \textbf{[EN] Definition}: 
    Field around charges exerting force on other charges.
    \begin{itemize}[leftmargin=*]
        \item \textbf{Coulomb's Law}: Force between point charges.
        \item \textbf{Electric Field ($\vec{E}$)}: Force per unit charge ($\vec{E} = \vec{F}/q$). Vector.
        \item \textbf{Electric Potential ($V$)}: Potential energy per unit charge. Scalar.
    \end{itemize}
\end{studybox}

\begin{formulabox}
\textbf{高斯定律 / Gauss's Law}:
\begin{equation}
    \oint \vec{E} \cdot d\vec{A} = \frac{Q_{enc}}{\epsilon_0}
\end{equation}
电通量等于包围的净电荷除以介电常数。

\textbf{电容 / Capacitance}:
\begin{equation}
    C = \frac{Q}{V} = \epsilon_0 \frac{A}{d} \quad \text{(平行板)}
\end{equation}
\end{formulabox}

% --------------------------------------------
% 2. Magnetic Field
% --------------------------------------------
\begin{studybox}{静磁场 / Magnetostatics}
    \textbf{[CN] 定义}: 
    由运动电荷 (电流) 产生的场。
    \begin{itemize}[leftmargin=*]
        \item \textbf{洛伦兹力 (Lorentz Force)}: 运动电荷在磁场中受到的力 $\vec{F} = q(\vec{v} \times \vec{B})$.
        \item \textbf{安培定律 (Ampere's Law)}: 电流产生磁场 $\oint \vec{B} \cdot d\vec{l} = \mu_0 I$.
    \end{itemize}
    \tcblower
    \textbf{[EN] Definition}: 
    Field produced by moving charges (currents).
    \begin{itemize}[leftmargin=*]
        \item \textbf{Lorentz Force}: Force on a moving charge ($\vec{F} = q(\vec{v} \times \vec{B})$). Always perpendicular to velocity (does no work!).
        \item \textbf{Ampere's Law}: Current generates magnetic field.
    \end{itemize}
\end{studybox}

% --------------------------------------------
% 3. Electromagnetic Induction
% --------------------------------------------
\begin{studybox}{电磁感应 / EM Induction}
    \textbf{[CN] 定义}: 
    磁通量变化产生感应电动势。
    \begin{itemize}[leftmargin=*]
        \item \textbf{法拉第定律 (Faraday's Law)}: 感应电动势等于磁通量变化率的负值。
        \item \textbf{楞次定律 (Lenz's Law)}: 感应电流的方向总是阻碍磁通量的变化 (那个负号)。
    \end{itemize}
    \tcblower
    \textbf{[EN] Definition}: 
    Changing magnetic flux induces EMF.
    \begin{itemize}[leftmargin=*]
        \item \textbf{Faraday's Law}: $\mathcal{E} = -\frac{d\Phi_B}{dt}$.
        \item \textbf{Lenz's Law}: The induced current flows in a direction that opposes the change in magnetic flux (Conservation of Energy).
    \end{itemize}
\end{studybox}

% --------------------------------------------
% Thesis Connection
% --------------------------------------------
\begin{thesisbox}[论文关联 / Project Application]
    \textbf{[CN]}: 
    \begin{itemize}
        \item \textbf{PCB 设计 (EMC)}: 快速变化的电流 (如 SPI/I2C 时钟) 会产生磁场辐射 (安培定律)。这是电磁干扰 (EMI) 的来源。
        \item \textbf{去耦电容}: 为高频电流提供回路,减小环路面积,从而减小辐射 (EMI reduction)。
        \item \textbf{无线通信}: ESP32 的 Wi-Fi 天线发射电磁波 (EM Waves),即变化的电场产生磁场,变化的磁场产生电场。
    \end{itemize}

    \tcblower

    \textbf{[EN]}: 
    \begin{itemize}
        \item \textbf{PCB EMC}: High-speed currents (clock lines) generate Magnetic Fields (Ampere's Law), causing Electromagnetic Interference (EMI).
        \item \textbf{Decoupling}: Reduces loop area for high-frequency currents, minimizing radiated emissions.
        \item \textbf{Antenna}: ESP32 Wi-Fi works by radiating EM waves (Maxwell's Equations).
    \end{itemize}
\end{thesisbox}

\begin{warnbox}[考试陷阱 / Exam Pitfalls]
    \begin{itemize}
        \item \textbf{Magnetic Work}: 磁力 $\vec{F} \perp \vec{v}$,所以恒定磁场\textbf{不对}电荷做功!(Work = 0).
        \item \textbf{Field Liens}: 电场线起于正止于负;磁场线是闭合曲线 (无磁单极子)。
    \end{itemize}
\end{warnbox}
