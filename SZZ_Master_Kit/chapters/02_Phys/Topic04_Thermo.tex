% ============================================
% Topic 04: Thermodynamics / 热力学
% Course: BE5B02PH2
% ============================================
\section{热力学 / Thermodynamics (BE5B02PH2)}

% --------------------------------------------
% 1. Temperature and Heat
% --------------------------------------------
\begin{studybox}{温度与热量 (Temperature and Heat)}

\textbf{概念 (CN)}: 温度是分子平均动能的度量,热量是能量传递

\textbf{Term (EN)}: Temperature, Heat, Thermal Equilibrium, Specific Heat

\tcblower

\textbf{热力学第零定律}:
如果 A 与 B 热平衡,B 与 C 热平衡,则 A 与 C 热平衡。

\textbf{温度标度}:
\begin{itemize}[leftmargin=*]
    \item 摄氏度: $T_C$ (水冰点 0°C, 沸点 100°C)
    \item 开尔文: $T_K = T_C + 273.15$
    \item 华氏度: $T_F = \frac{9}{5}T_C + 32$
\end{itemize}

\textbf{热量}:
\begin{equation}
    Q = mc\Delta T
\end{equation}
其中 $c$ 是比热容 (Specific Heat Capacity)

\textbf{Key Insight}: DHT22 sensor measures temperature using thermistor principle.

\end{studybox}

\begin{formulabox}
\textbf{热传导 (Fourier's Law)}:
\begin{equation}
    \frac{dQ}{dt} = -kA\frac{dT}{dx}
\end{equation}

\textbf{热辐射 (Stefan-Boltzmann Law)}:
\begin{equation}
    P = \varepsilon\sigma A T^4
\end{equation}
其中 $\sigma = 5.67 \times 10^{-8} \, \text{W/m}^2\text{K}^4$

\textbf{理想气体状态方程}:
\begin{equation}
    \boxed{PV = nRT}
\end{equation}
$R = 8.314 \, \text{J/(mol·K)}$
\end{formulabox}

% --------------------------------------------
% 2. Laws of Thermodynamics
% --------------------------------------------
\begin{studybox}{热力学定律 (Laws of Thermodynamics)}

\textbf{概念 (CN)}: 描述热、功和能量转换的基本定律

\textbf{Term (EN)}: First/Second Law, Entropy, Carnot Cycle

\tcblower

\textbf{第一定律} (能量守恒):
\begin{equation}
    \boxed{\Delta U = Q - W}
\end{equation}
内能变化 = 吸收热量 - 对外做功

\textbf{第二定律}:
\begin{itemize}[leftmargin=*]
    \item \textbf{克劳修斯表述}: 热量不能自发从低温传到高温
    \item \textbf{开尔文表述}: 不可能从单一热源取热全部转化为功
\end{itemize}

\textbf{熵 (Entropy)}:
\begin{equation}
    dS = \frac{dQ_{rev}}{T}
\end{equation}

\textbf{Key Insight}: Any computing system generates heat due to entropy increase.

\end{studybox}

\begin{formulabox}
\textbf{热力学过程}:

\begin{center}
\begin{tabular}{lccc}
\toprule
\textbf{过程} & \textbf{条件} & \textbf{$Q$} & \textbf{$W$} \\
\midrule
等容 (Isochoric) & $V$ = const & $nC_V\Delta T$ & 0 \\
等压 (Isobaric) & $P$ = const & $nC_P\Delta T$ & $P\Delta V$ \\
等温 (Isothermal) & $T$ = const & $nRT\ln(V_2/V_1)$ & $Q$ \\
绝热 (Adiabatic) & $Q$ = 0 & 0 & $-\Delta U$ \\
\bottomrule
\end{tabular}
\end{center}

\textbf{绝热过程}: $PV^\gamma = \text{const}$, $\gamma = C_P/C_V$

\textbf{卡诺效率}: $\eta = 1 - \frac{T_C}{T_H}$
\end{formulabox}

% --------------------------------------------
% Thesis Connection
% --------------------------------------------
\begin{thesisbox}
\textbf{热力学与电子系统}:

\textbf{温度传感器原理}:
DHT22 使用热敏电阻,电阻随温度变化:
\begin{equation}
    R(T) = R_0 e^{\beta(1/T - 1/T_0)}
\end{equation}

\textbf{ESP32 发热}:
CPU 在计算时消耗功率 $P = VI$,转化为热量。需要散热设计。

\textbf{环境监测}:
温湿度影响舒适度指数:
\begin{equation}
    \text{Heat Index} = f(T, RH)
\end{equation}

\textbf{Jan Koller 问题}: "What is the accuracy of your temperature sensor?"

\textbf{答案}: DHT22 精度 $\pm 0.5°C$。原理是 NTC 热敏电阻,电阻随温度指数变化。传感器内置 ADC 和校准数据,通过单线协议输出数字值。
\end{thesisbox}

% --------------------------------------------
% Exam Strategy
% --------------------------------------------
\begin{warnbox}[[!] 考试陷阱 / Exam Pitfalls]
\begin{enumerate}
    \item \textbf{$Q$ 的符号}: $Q > 0$ 系统吸热,$Q < 0$ 系统放热。
    \item \textbf{$W$ 的符号}: $W > 0$ 系统对外做功(膨胀)。
    \item \textbf{绝热 vs 等温}: 绝热过程快,等温过程慢。
    \item \textbf{熵增}: 孤立系统熵只能增加或不变。
\end{enumerate}
\end{warnbox}
