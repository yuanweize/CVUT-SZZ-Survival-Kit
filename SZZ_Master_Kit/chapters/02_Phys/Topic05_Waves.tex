% ============================================
% Topic 05: Waves / 波动
% Course: BE5B02PH2
% ============================================
\section{波动与光学 / Waves \& Optics (BE5B02PH2)}

% --------------------------------------------
% 1. Wave Fundamentals
% --------------------------------------------
\begin{studybox}{波的基本概念 (Wave Fundamentals)}

\textbf{概念 (CN)}: 能量和振动在空间中的传播

\textbf{Term (EN)}: Wave, Wavelength, Frequency, Phase Velocity

\tcblower

\textbf{基本参数}:
\begin{itemize}[leftmargin=*]
    \item \textbf{波长 (Wavelength)}: $\lambda$ - 一个完整波的空间长度
    \item \textbf{频率 (Frequency)}: $f$ - 每秒振动次数
    \item \textbf{周期 (Period)}: $T = 1/f$
    \item \textbf{波速 (Phase Velocity)}: $v = f\lambda$
\end{itemize}

\textbf{波的类型}:
\begin{itemize}[leftmargin=*]
    \item \textbf{横波 (Transverse)}: 振动垂直于传播方向(光、电磁波)
    \item \textbf{纵波 (Longitudinal)}: 振动平行于传播方向(声波)
\end{itemize}

\textbf{Key Insight}: WiFi (2.4GHz) is an electromagnetic wave with $\lambda \approx 12.5$ cm.

\end{studybox}

\begin{formulabox}
\textbf{一维波动方程}:
\begin{equation}
    \frac{\partial^2 y}{\partial x^2} = \frac{1}{v^2}\frac{\partial^2 y}{\partial t^2}
\end{equation}

\textbf{行波解}:
\begin{equation}
    y(x, t) = A\sin(kx - \omega t + \phi)
\end{equation}

其中:
\begin{itemize}[leftmargin=*]
    \item 波数 (Wave Number): $k = \frac{2\pi}{\lambda}$
    \item 角频率: $\omega = 2\pi f$
    \item 色散关系: $v = \frac{\omega}{k} = f\lambda$
\end{itemize}
\end{formulabox}

% --------------------------------------------
% 2. Wave Phenomena
% --------------------------------------------
\begin{studybox}{波的现象 (Wave Phenomena)}

\textbf{概念 (CN)}: 波在传播过程中的各种行为

\textbf{Term (EN)}: Interference, Diffraction, Reflection, Refraction

\tcblower

\textbf{叠加原理}: 多个波在空间中合成
\begin{equation}
    y_{total} = y_1 + y_2 + \ldots
\end{equation}

\textbf{干涉 (Interference)}:
\begin{itemize}[leftmargin=*]
    \item \textbf{相长干涉}: 相位差 $= 2n\pi$ (波峰对波峰)
    \item \textbf{相消干涉}: 相位差 $= (2n+1)\pi$ (波峰对波谷)
\end{itemize}

\textbf{衍射 (Diffraction)}: 波绕过障碍物传播

\textbf{Key Insight}: WiFi multipath causes interference patterns in buildings.

\end{studybox}

\begin{formulabox}
\textbf{反射与折射}:

\textbf{反射定律}: $\theta_i = \theta_r$

\textbf{折射定律 (Snell's Law)}:
\begin{equation}
    n_1\sin\theta_1 = n_2\sin\theta_2
\end{equation}

\textbf{全内反射}: 当 $\theta > \theta_c = \arcsin(n_2/n_1)$

\textbf{双缝干涉}:
\begin{equation}
    d\sin\theta = m\lambda \quad \text{(亮条纹)}
\end{equation}

\textbf{单缝衍射}:
\begin{equation}
    a\sin\theta = m\lambda \quad \text{(暗条纹)}
\end{equation}
\end{formulabox}

% --------------------------------------------
% 3. Electromagnetic Waves
% --------------------------------------------
\begin{studybox}{电磁波 (Electromagnetic Waves)}

\textbf{概念 (CN)}: 电场和磁场的振荡在空间中传播

\textbf{Term (EN)}: EM Spectrum, Radio Waves, Microwave, Light

\tcblower

\textbf{麦克斯韦方程组} 预言了电磁波的存在

\textbf{电磁波性质}:
\begin{itemize}[leftmargin=*]
    \item 横波:$\vec{E} \perp \vec{B} \perp \vec{v}$
    \item 真空速度:$c = \frac{1}{\sqrt{\varepsilon_0\mu_0}} \approx 3 \times 10^8 \, \text{m/s}$
    \item 无需介质传播
\end{itemize}

\textbf{电磁波谱}: 无线电 $\to$ 微波 $\to$ 红外 $\to$ 可见光 $\to$ 紫外 $\to$ X射线 $\to$ 伽马射线

\textbf{Key Insight}: WiFi (2.4 GHz) and Bluetooth (2.402-2.480 GHz) are microwave bands.

\end{studybox}

\begin{formulabox}
\textbf{电磁波频率与波长}:
\begin{equation}
    c = f\lambda \quad \Rightarrow \quad \lambda = \frac{c}{f}
\end{equation}

\textbf{常见电磁波}:
\begin{center}
\begin{tabular}{lcc}
\toprule
\textbf{类型} & \textbf{频率} & \textbf{波长} \\
\midrule
WiFi 2.4GHz & $2.4 \times 10^9$ Hz & 12.5 cm \\
WiFi 5GHz & $5 \times 10^9$ Hz & 6 cm \\
可见光 (红) & $4 \times 10^{14}$ Hz & 700 nm \\
可见光 (紫) & $8 \times 10^{14}$ Hz & 400 nm \\
\bottomrule
\end{tabular}
\end{center}

\textbf{信号衰减 (Free Space Path Loss)}:
\begin{equation}
    FSPL = \left(\frac{4\pi d}{\lambda}\right)^2 \propto \frac{1}{r^2}
\end{equation}
\end{formulabox}

% --------------------------------------------
% 4. Sound Waves
% --------------------------------------------
\begin{studybox}{声波 (Sound Waves)}

\textbf{概念 (CN)}: 在介质中传播的纵波

\textbf{Term (EN)}: Sound, Intensity, Decibel, Doppler Effect

\tcblower

\textbf{声速}: 取决于介质
\begin{itemize}[leftmargin=*]
    \item 空气 (20°C): ~343 m/s
    \item 水: ~1500 m/s
    \item 钢: ~5000 m/s
\end{itemize}

\textbf{多普勒效应}:
\begin{equation}
    f' = f\frac{v \pm v_o}{v \mp v_s}
\end{equation}

\textbf{声强级}:
\begin{equation}
    L = 10\log_{10}\frac{I}{I_0} \, \text{dB}, \quad I_0 = 10^{-12} \, \text{W/m}^2
\end{equation}

\textbf{Key Insight}: Ultrasonic sensors use sound wave timing for distance measurement.

\end{studybox}

% --------------------------------------------
% Thesis Connection
% --------------------------------------------
\begin{thesisbox}
\textbf{波动与无线通信}:

你的论文使用 WiFi 进行无线通信:

\textbf{WiFi 信号衰减}:
RSSI (接收信号强度) 遵循反平方定律:
\begin{equation}
    P_r \propto \frac{1}{d^2}
\end{equation}
这意味着距离加倍,信号功率减少为 1/4 (即 -6 dB)。

\textbf{多径干涉}:
室内环境中,直射波和反射波叠加产生干涉:
\begin{itemize}
    \item 相长干涉:信号增强
    \item 相消干涉:信号减弱(死区)
\end{itemize}

\textbf{天线设计}:
PCB 天线长度与波长相关:
\begin{equation}
    L \approx \frac{\lambda}{4} = \frac{c}{4f} \approx 31 \, \text{mm} \quad \text{(for 2.4 GHz)}
\end{equation}

\textbf{Jan Koller 问题}: "Why did you choose 2.4GHz WiFi over 5GHz?"

\textbf{答案}: 2.4 GHz 波长更长 (12.5 cm vs 6 cm),穿透墙壁能力更强,覆盖范围更广。虽然 5 GHz 带宽更大,但对于传感器数据(几百字节)不需要高吞吐量。2.4 GHz 更适合智能家居场景。
\end{thesisbox}

% --------------------------------------------
% Exam Strategy
% --------------------------------------------
\begin{warnbox}[[!] 考试陷阱 / Exam Pitfalls]
\begin{enumerate}
    \item \textbf{波速 vs 粒子速度}: 波速是相位传播速度,粒子在原地振动。
    \item \textbf{折射}: 进入光密介质时,光线向法线偏折。
    \item \textbf{干涉条件}: 两波必须相干(同频、稳定相位差)。
    \item \textbf{dB 计算}: dB 是对数单位,3 dB 对应功率翻倍。
\end{enumerate}
\end{warnbox}
