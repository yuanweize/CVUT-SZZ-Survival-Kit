% ============================================
% Topic 01: Circuit Theory / 电路理论
% ============================================
\section{电路理论 / Circuit Theory}

% --------------------------------------------
% 1. Kirchhoff's Laws
% --------------------------------------------
\begin{studybox}{基尔霍夫定律 (Kirchhoff's Laws)}

\textbf{概念 (CN)}: 基尔霍夫电流定律 (KCL) 与基尔霍夫电压定律 (KVL)

\textbf{Term (EN)}: Kirchhoff's Current Law (KCL) \& Kirchhoff's Voltage Law (KVL)

\tcblower

\textbf{定义}:
\begin{itemize}[leftmargin=*]
    \item \textbf{KCL (电流定律)}: 流入任一节点的电流之和等于流出该节点的电流之和。基于\textbf{电荷守恒}。
    \item \textbf{KVL (电压定律)}: 沿任一闭合回路,所有电压降的代数和为零。基于\textbf{能量守恒}。
\end{itemize}

\textbf{Key Insight (EN)}: KCL ensures charge is conserved at nodes; KVL ensures energy is conserved around loops. These are the foundation of ALL circuit analysis.

\end{studybox}

\begin{formulabox}
\textbf{KCL (节点分析)}:
\begin{equation}
    \sum_{k=1}^{n} I_k = 0
\end{equation}
其中 $I_k$ 是进入节点的电流(流出为负)。

\textbf{KVL (回路分析)}:
\begin{equation}
    \sum_{k=1}^{n} V_k = 0
\end{equation}
其中 $V_k$ 是回路中的电压降。
\end{formulabox}

% --------------------------------------------
% 2. Thevenin & Norton Equivalents
% --------------------------------------------
\begin{studybox}{戴维南与诺顿等效 (Thevenin \& Norton)}

\textbf{概念 (CN)}: 任何线性二端网络可以等效为一个电压源串联电阻(戴维南)或一个电流源并联电阻(诺顿)。

\textbf{Term (EN)}: Thevenin Equivalent \& Norton Equivalent

\tcblower

\textbf{戴维南定理}:
\begin{itemize}[leftmargin=*]
    \item $V_{Th}$ = 开路电压 (Open-circuit voltage)
    \item $R_{Th}$ = 从端口看入的等效电阻(所有独立源置零)
\end{itemize}

\textbf{诺顿定理}:
\begin{itemize}[leftmargin=*]
    \item $I_N$ = 短路电流 (Short-circuit current)
    \item $R_N = R_{Th}$(等效电阻相同)
\end{itemize}

\textbf{Key Insight}: Thevenin/Norton simplifies complex circuits into a simple source + resistor for load analysis.

\end{studybox}

\begin{formulabox}
\textbf{戴维南等效电路}:
\begin{equation}
    V_{load} = V_{Th} - I_{load} \cdot R_{Th}
\end{equation}

\textbf{诺顿等效电路}:
\begin{equation}
    I_N = \frac{V_{Th}}{R_{Th}}
\end{equation}

\textbf{转换关系}:
\begin{equation}
    V_{Th} = I_N \cdot R_N, \quad R_{Th} = R_N
\end{equation}
\end{formulabox}

% --------------------------------------------
% 3. RLC Circuits & Resonance
% --------------------------------------------
\begin{studybox}{RLC 电路与谐振 (RLC Circuits \& Resonance)}

\textbf{概念 (CN)}: RLC 串联/并联电路在特定频率下发生谐振,阻抗最小/最大。

\textbf{Term (EN)}: Series/Parallel Resonance, Quality Factor (Q)

\tcblower

\textbf{串联谐振} (Series Resonance):
\begin{itemize}[leftmargin=*]
    \item 谐振频率: $f_0 = \frac{1}{2\pi\sqrt{LC}}$
    \item 阻抗最小: $Z_{min} = R$
    \item 电流最大
\end{itemize}

\textbf{并联谐振} (Parallel Resonance):
\begin{itemize}[leftmargin=*]
    \item 阻抗最大
    \item 电流最小
\end{itemize}

\textbf{Key Insight}: Resonance is used in oscillators, filters, and tuning circuits. Your ESP32's crystal oscillator relies on piezoelectric resonance.

\end{studybox}

\begin{formulabox}
\textbf{谐振频率}:
\begin{equation}
    \omega_0 = \frac{1}{\sqrt{LC}}, \quad f_0 = \frac{1}{2\pi\sqrt{LC}}
\end{equation}

\textbf{品质因数 (Quality Factor)}:
\begin{equation}
    Q = \frac{\omega_0 L}{R} = \frac{1}{\omega_0 CR} = \frac{1}{R}\sqrt{\frac{L}{C}}
\end{equation}

\textbf{带宽}:
\begin{equation}
    BW = \frac{f_0}{Q}
\end{equation}
\end{formulabox}

% --------------------------------------------
% 4. Power in AC Circuits
% --------------------------------------------
\begin{studybox}{交流电路中的功率 (Power in AC Circuits)}

\textbf{概念 (CN)}: 交流电路中存在有功功率、无功功率和视在功率。

\textbf{Term (EN)}: Real Power (P), Reactive Power (Q), Apparent Power (S), Power Factor

\tcblower

\textbf{三种功率}:
\begin{itemize}[leftmargin=*]
    \item \textbf{有功功率 P}: 实际消耗的功率 [W]
    \item \textbf{无功功率 Q}: 储能元件(L/C)交换的功率 [VAR]
    \item \textbf{视在功率 S}: 电源提供的总功率 [VA]
\end{itemize}

\textbf{Key Insight}: Power factor tells you how efficiently power is being used. Low power factor = wasted energy.

\end{studybox}

\begin{formulabox}
\textbf{功率关系}:
\begin{equation}
    S = \sqrt{P^2 + Q^2} = V_{rms} \cdot I_{rms}
\end{equation}

\textbf{有功功率}:
\begin{equation}
    P = V_{rms} \cdot I_{rms} \cdot \cos\phi = S \cdot \cos\phi
\end{equation}

\textbf{无功功率}:
\begin{equation}
    Q = V_{rms} \cdot I_{rms} \cdot \sin\phi = S \cdot \sin\phi
\end{equation}

\textbf{功率因数}:
\begin{equation}
    \cos\phi = \frac{P}{S}
\end{equation}
\end{formulabox}

% --------------------------------------------
% Thesis Connection
% --------------------------------------------
\begin{thesisbox}
\textbf{ESP32 电源设计}:

你的 ESP32-S3 通过 USB (5V) 供电,经过 LDO 线性稳压器降压到 3.3V。

\textbf{KCL 应用}: 流入 LDO 的电流 = ESP32 消耗电流 + 热损耗。

\textbf{功率应用}: ESP32 峰值功耗约 500mW ($P = U \cdot I = 3.3V \times 150mA$)。

\textbf{Jan Koller 问题}: "How is power dissipated in the LDO?"

\textbf{答案}: $P_{LDO} = (V_{in} - V_{out}) \times I = (5V - 3.3V) \times 0.15A = 0.255W$
\end{thesisbox}

% --------------------------------------------
% Exam Strategy
% --------------------------------------------
\begin{warnbox}[[!] 考试陷阱 / Exam Pitfalls]
\begin{enumerate}
    \item \textbf{忘记电流方向}: KVL/KCL 要求统一的电流/电压参考方向。
    \item \textbf{混淆 RMS 与峰值}: $V_{peak} = \sqrt{2} \cdot V_{rms}$。
    \item \textbf{功率因数符号}: 容性负载功率因数超前,感性滞后。
\end{enumerate}
\end{warnbox}
