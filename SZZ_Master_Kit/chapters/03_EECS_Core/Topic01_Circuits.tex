% ============================================
% Topic 01: Circuit Theory / 电路理论
% Course: BE5B31ZEO
% ============================================
\section{电路理论 / Circuit Theory (BE5B31ZEO)}

% --------------------------------------------
% 1. Basic Laws
% --------------------------------------------
\begin{studybox}{基尔霍夫定律 (Kirchhoff's Laws)}

\textbf{概念 (CN)}: 电路分析的两个基本定律

\textbf{Term (EN)}: KVL, KCL, Node, Loop, Branch

\tcblower

\textbf{基尔霍夫电流定律 (KCL)}:
节点上流入的电流之和等于流出的电流之和。
\begin{equation}
    \boxed{\sum_{k=1}^{n} I_k = 0}
\end{equation}
电流守恒 $\Rightarrow$ 没有电荷在节点积累

\textbf{基尔霍夫电压定律 (KVL)}:
回路中电压降之和等于零。
\begin{equation}
    \boxed{\sum_{k=1}^{n} V_k = 0}
\end{equation}
能量守恒 $\Rightarrow$ 绕回路一圈回到原点

\textbf{Key Insight}: Every circuit analysis problem uses KVL and KCL.

\end{studybox}

\begin{formulabox}
\textbf{KCL 示例}: 节点分析
\begin{center}
节点 A: $I_1 = I_2 + I_3$
\end{center}
流入 = 流出

\textbf{KVL 示例}: 回路分析
\begin{center}
$-V_s + V_{R1} + V_{R2} = 0$

$\Rightarrow V_s = I \cdot R_1 + I \cdot R_2 = I(R_1 + R_2)$
\end{center}

\textbf{分压公式} (串联电阻):
\begin{equation}
    V_2 = V_s \cdot \frac{R_2}{R_1 + R_2}
\end{equation}

\textbf{分流公式} (并联电阻):
\begin{equation}
    I_1 = I_{total} \cdot \frac{R_2}{R_1 + R_2}
\end{equation}
\end{formulabox}

% --------------------------------------------
% 2. Ohm's Law and Resistors
% --------------------------------------------
\begin{studybox}{欧姆定律与电阻 (Ohm's Law and Resistors)}

\textbf{概念 (CN)}: 电压、电流和电阻的基本关系

\textbf{Term (EN)}: Ohm's Law, Resistance, Series, Parallel

\tcblower

\textbf{欧姆定律}:
\begin{equation}
    \boxed{V = IR} \quad \text{或} \quad I = \frac{V}{R} \quad \text{或} \quad R = \frac{V}{I}
\end{equation}

\textbf{功率}:
\begin{equation}
    P = VI = I^2 R = \frac{V^2}{R}
\end{equation}

\textbf{电阻组合}:
\begin{itemize}[leftmargin=*]
    \item 串联: $R_{eq} = R_1 + R_2 + \ldots$
    \item 并联: $\frac{1}{R_{eq}} = \frac{1}{R_1} + \frac{1}{R_2} + \ldots$
\end{itemize}

\textbf{Key Insight}: Voltage divider is fundamental in sensor interfacing.

\end{studybox}

\begin{formulabox}
\textbf{两电阻并联}:
\begin{equation}
    R_{eq} = \frac{R_1 \cdot R_2}{R_1 + R_2}
\end{equation}

\textbf{电导 (Conductance)}:
\begin{equation}
    G = \frac{1}{R}, \quad \text{单位: 西门子 (S)}
\end{equation}

\textbf{电阻率}:
\begin{equation}
    R = \rho \frac{L}{A}
\end{equation}
$\rho$: 电阻率, $L$: 长度, $A$: 截面积

\textbf{温度系数}:
\begin{equation}
    R(T) = R_0[1 + \alpha(T - T_0)]
\end{equation}
\end{formulabox}

% --------------------------------------------
% 3. Capacitors and Inductors
% --------------------------------------------
\begin{studybox}{电容与电感 (Capacitors and Inductors)}

\textbf{概念 (CN)}: 储能元件

\textbf{Term (EN)}: Capacitor, Inductor, Time Constant, Energy Storage

\tcblower

\textbf{电容器}:
\begin{equation}
    C = \frac{Q}{V}, \quad I = C\frac{dV}{dt}
\end{equation}
能量: $E = \frac{1}{2}CV^2$

\textbf{电感器}:
\begin{equation}
    V = L\frac{dI}{dt}
\end{equation}
能量: $E = \frac{1}{2}LI^2$

\textbf{组合规则}:
\begin{itemize}[leftmargin=*]
    \item 电容串联: $\frac{1}{C_{eq}} = \frac{1}{C_1} + \frac{1}{C_2}$ (与电阻相反!)
    \item 电容并联: $C_{eq} = C_1 + C_2$
    \item 电感组合规则与电阻相同
\end{itemize}

\textbf{Key Insight}: Decoupling capacitors filter power supply noise.

\end{studybox}

\begin{formulabox}
\textbf{RC 电路时间常数}:
\begin{equation}
    \tau = RC
\end{equation}

\textbf{RC 充电}:
\begin{equation}
    V_C(t) = V_s(1 - e^{-t/\tau})
\end{equation}

\textbf{RC 放电}:
\begin{equation}
    V_C(t) = V_0 e^{-t/\tau}
\end{equation}

\textbf{RL 电路时间常数}:
\begin{equation}
    \tau = \frac{L}{R}
\end{equation}

\textbf{$5\tau$ 法则}: 约 99.3\% 达到稳态
\end{formulabox}

% --------------------------------------------
% 4. AC Circuit Analysis
% --------------------------------------------
\begin{studybox}{交流电路分析 (AC Circuit Analysis)}

\textbf{概念 (CN)}: 正弦稳态电路的分析方法

\textbf{Term (EN)}: Phasor, Impedance, Reactance, RMS

\tcblower

\textbf{正弦信号}:
\begin{equation}
    v(t) = V_m \cos(\omega t + \phi)
\end{equation}

\textbf{相量 (Phasor)}:
\begin{equation}
    \mathbf{V} = V_m \angle \phi = V_m e^{j\phi}
\end{equation}

\textbf{阻抗 (Impedance)}:
\begin{equation}
    \mathbf{Z} = R + jX
\end{equation}
\begin{itemize}[leftmargin=*]
    \item 电阻: $Z_R = R$
    \item 电容: $Z_C = \frac{1}{j\omega C} = -\frac{j}{\omega C}$
    \item 电感: $Z_L = j\omega L$
\end{itemize}

\textbf{Key Insight}: Complex impedance makes AC analysis like DC analysis with Ohm's law.

\end{studybox}

\begin{formulabox}
\textbf{RMS (有效值)}:
\begin{equation}
    V_{rms} = \frac{V_m}{\sqrt{2}} \approx 0.707 V_m
\end{equation}

\textbf{平均功率}:
\begin{equation}
    P = V_{rms} I_{rms} \cos\phi = I_{rms}^2 R
\end{equation}

\textbf{功率因数}: $\cos\phi$ (相位差的余弦)

\textbf{谐振频率} (RLC 电路):
\begin{equation}
    f_0 = \frac{1}{2\pi\sqrt{LC}}
\end{equation}
\end{formulabox}

% --------------------------------------------
% 5. Thévenin and Norton Equivalents
% --------------------------------------------
\begin{studybox}{等效电路定理 (Equivalent Circuit Theorems)}

\textbf{概念 (CN)}: 将复杂电路简化为等效电路

\textbf{Term (EN)}: Thévenin, Norton, Equivalent Circuit

\tcblower

\textbf{戴维南定理 (Thévenin's Theorem)}:
任何线性两端网络可以用电压源 $V_{th}$ 和串联电阻 $R_{th}$ 等效。

\textbf{诺顿定理 (Norton's Theorem)}:
任何线性两端网络可以用电流源 $I_N$ 和并联电阻 $R_N$ 等效。

\textbf{转换关系}:
\begin{equation}
    V_{th} = I_N \cdot R_N, \quad R_{th} = R_N
\end{equation}

\textbf{Key Insight}: Thévenin equivalent simplifies load analysis.

\end{studybox}

\begin{formulabox}
\textbf{求戴维南等效步骤}:
\begin{enumerate}
    \item 断开负载
    \item $V_{th}$: 计算开路电压
    \item $R_{th}$: 关闭独立源 (电压源短路,电流源开路),求等效电阻
\end{enumerate}

\textbf{最大功率传输定理}:
当 $R_L = R_{th}$ 时,负载获得最大功率:
\begin{equation}
    P_{max} = \frac{V_{th}^2}{4R_{th}}
\end{equation}
\end{formulabox}

% --------------------------------------------
% Thesis Connection
% --------------------------------------------
\begin{thesisbox}
\textbf{电路理论与 ESP32 电源设计}:

\textbf{LDO 稳压器分析 (使用 KVL)}:
AMS1117-3.3 将 5V USB 降压到 3.3V:
\begin{equation}
    V_{USB} = V_{dropout} + V_{out}
\end{equation}
$V_{dropout} \approx 1.1V$, $V_{out} = 3.3V$

\textbf{分压器设计 (使用 KVL)}:
如果你需要将 5V 传感器信号连接到 3.3V GPIO:
\begin{equation}
    V_{out} = 5V \cdot \frac{R_2}{R_1 + R_2} = 3.3V
\end{equation}
选择 $R_1 = 1.8k\Omega$, $R_2 = 3.3k\Omega$

\textbf{退耦电容 (使用 KCL)}:
100nF 电容在芯片 VCC 附近:
\begin{itemize}
    \item 芯片瞬态电流需求 $\to$ 电容提供
    \item KCL: $I_{supply} + I_{cap} = I_{chip}$
\end{itemize}

\textbf{Jan Koller 问题}: "How do you handle power supply noise?"

\textbf{答案}: 使用退耦电容。根据 KCL,当芯片需要瞬态电流时,附近的电容提供。我在每个 IC 的 VCC 引脚使用 100nF 陶瓷电容(高频滤波)和 10µF 电解电容(低频滤波)。这形成一个 RC 低通滤波器。
\end{thesisbox}

% --------------------------------------------
% Exam Strategy
% --------------------------------------------
\begin{warnbox}[[!] 考试陷阱 / Exam Pitfalls]
\begin{enumerate}
    \item \textbf{KVL 符号}: 绕回路时保持一致的方向,注意电压升降。
    \item \textbf{电容串并联}: 与电阻规则相反!串联电容减小,并联增加。
    \item \textbf{相量计算}: $j^2 = -1$,注意复数运算。
    \item \textbf{时间常数}: RC 充电到 $1 - e^{-1} \approx 63.2\%$,不是 50\%!
\end{enumerate}
\end{warnbox}
