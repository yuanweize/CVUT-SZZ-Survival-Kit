% ============================================
% Topic 01: Circuit Theory / 电路理论
% Course: BE5B31ZEO
% ============================================
\section{电路理论 / Circuit Theory (BE5B31ZEO)}

% --------------------------------------------
% 1. Basic Laws
% --------------------------------------------
\begin{studybox}{基尔霍夫定律 / Kirchhoff's Laws}
    \textbf{[CN] 定义}: 基尔霍夫定律是电路分析的基础。
    \begin{itemize}[leftmargin=*]
        \item \textbf{电流定律 (KCL)}: 节点上的电流守恒。流入节点的电流之和等于流出的电流之和(基于电荷守恒)。
        \item \textbf{电压定律 (KVL)}: 回路中的能量守恒。沿着闭合回路一周,电压升与电压降的代数和为零。
    \end{itemize}
    \tcblower
    \textbf{[EN] Definition}: 
    \begin{itemize}[leftmargin=*]
        \item \textbf{KCL}: The algebraic sum of currents entering a node is zero ($\sum I = 0$). It reflects the conservation of charge.
        \item \textbf{KVL}: The algebraic sum of voltage drops around any closed loop is zero ($\sum V = 0$). It reflects the conservation of energy.
    \end{itemize}
\end{studybox}

\begin{formulabox}
\textbf{KCL (Node Analysis)}:
\begin{equation}
    \sum_{k=1}^{n} I_k = 0
\end{equation}

\textbf{KVL (Loop Analysis)}:
\begin{equation}
    \sum_{k=1}^{n} V_k = 0
\end{equation}

\textbf{分压公式 / Voltage Divider}:
\begin{equation}
    V_2 = V_{source} \cdot \frac{R_2}{R_1 + R_2}
\end{equation}
使用场景:将 5V 传感器信号转换为 3.3V (Using 1.8k + 3.3k resistors).
\end{formulabox}

% --------------------------------------------
% 2. Ohm's Law and Resistors
% --------------------------------------------
\begin{studybox}{欧姆定律与电阻 / Ohm's Law and Resistors}
    \textbf{[CN] 定义}: 欧姆定律 (Ohm's Law) 描述了线性元件中电压、电流和电阻的关系。
    电阻 (Resistance) 阻碍电流流动并消耗能量。电阻可以串联 (Series) 或者是并联 (Parallel) 连接。
    \tcblower
    \textbf{[EN] Definition}: Ohm's Law states that the current through a conductor is directly proportional to the voltage across it ($V=IR$). Resistance opposes current flow. Resistors can be connected in Series (additive resistance) or Parallel (additive conductance).
\end{studybox}

\begin{formulabox}
\textbf{欧姆定律}:
\begin{equation}
    V = I \cdot R, \quad P = V \cdot I = I^2 R
\end{equation}

\textbf{串并联 / Series \& Parallel}:
\begin{itemize}
    \item Series: $R_{eq} = R_1 + R_2 + \dots$
    \item Parallel: $\frac{1}{R_{eq}} = \frac{1}{R_1} + \frac{1}{R_2} + \dots$
\end{itemize}
\end{formulabox}

% --------------------------------------------
% 3. Capacitors and Inductors
% --------------------------------------------
\begin{studybox}{动态元件 / Capacitors and Inductors}
    \textbf{[CN] 定义}: 
    \begin{itemize}[leftmargin=*]
        \item \textbf{电容 (Capacitor)}: 以电场形式储存能量,阻碍电压突变。即 $i = C \frac{dv}{dt}$。
        \item \textbf{电感 (Inductor)}: 以磁场形式储存能量,阻碍电流突变。即 $v = L \frac{di}{dt}$。
        \item \textbf{时间常数 (Time Constant)}: 描述电路响应速度的参数 ($\tau = RC$ 或 $\tau = L/R$)。
    \end{itemize}
    \tcblower
    \textbf{[EN] Definition}:
    \begin{itemize}[leftmargin=*]
        \item \textbf{Capacitor}: Stores energy in an electric field and opposes voltage changes ($I = C \dot{V}$).
        \item \textbf{Inductor}: Stores energy in a magnetic field and opposes current changes ($V = L \dot{I}$).
        \item \textbf{Time Constant ($\tau$)}: Characterizes the transient response definition. $1\tau \approx 63.2\%$ of final value.
    \end{itemize}
\end{studybox}

\begin{formulabox}
\textbf{RC 充电公式 / RC Charging}:
\begin{equation}
    V_C(t) = V_S (1 - e^{-t/\tau}), \quad \tau = R \cdot C
\end{equation}
\textbf{应用}: 去耦电容 (Decoupling Capacitor) 为芯片提供瞬态电流。
\end{formulabox}

% --------------------------------------------
% 4. AC Analysis
% --------------------------------------------
\begin{studybox}{交流分析 / AC Circuit Analysis}
    \textbf{[CN] 定义}: 在正弦稳态下,我们使用相量 (Phasor) 和阻抗 (Impedance) 进行分析。这允许我们将微分方程转化为代数方程。阻抗 $\mathbf{Z} = R + jX$ 包含实部电阻和虚部电抗。
    \tcblower
    \textbf{[EN] Definition}: AC analysis uses Phasors (complex numbers) to represent sinusoidal signals. Impedance ($\mathbf{Z}$) generalizes resistance to include phase shifts, turning differential equations into algebraic ones ($V = I \cdot Z$).
\end{studybox}

\begin{formulabox}
\textbf{阻抗公式 / Impedance}:
\begin{itemize}
    \item Resistor: $Z_R = R$
    \item Inductor: $Z_L = j\omega L$
    \item Capacitor: $Z_C = \frac{1}{j\omega C} = -j \frac{1}{\omega C}$
\end{itemize}
\textbf{谐振频率 / Resonance}: $f_0 = \frac{1}{2\pi \sqrt{LC}}$
\end{formulabox}

% --------------------------------------------
% 5. Thevenin Theorem
% --------------------------------------------
\begin{studybox}{戴维南定理 / Thévenin's Theorem}
    \textbf{[CN] 定义}: 任何线性双端网络都可以等效为一个电压源 ($V_{th}$) 和一个串联电阻 ($R_{th}$)。这极大地简化了负载分析。
    \tcblower
    \textbf{[EN] Definition}: Any linear two-terminal circuit can be replaced by an equivalent circuit consisting of a single voltage source ($V_{th}$) in series with a resistor ($R_{th}$). This simplifies load connecting analysis.
\end{studybox}

% --------------------------------------------
% Thesis Connection
% --------------------------------------------
\begin{thesisbox}[论文关联 / Project Application]
    \textbf{[CN]}: 你的论文在电源设计部分大量运用了电路理论:
    \begin{itemize}
        \item \textbf{LDO稳压}: AMS1117-3.3 及其外围电容构成了一个稳压系统。
        \item \textbf{去耦 (Decoupling)}: 根据 KCL,当 ESP32 需要瞬态大电流时,100nF 和 10\textmu F 电容提供电流,防止电压跌落。
        \item \textbf{分压器}: 某些传感器 (5V) 需要分压才能连接到 ESP32 (3.3V GPIO)。
    \end{itemize}

    \tcblower

    \textbf{[EN]}: Your thesis applies circuit theory in the power supply design:
    \begin{itemize}
        \item \textbf{LDO Regulator}: The AMS1117-3.3 provides stable voltage.
        \item \textbf{Decoupling}: Capacitors (100nF + 10\textmu F) provide transient current (KCL) to filter noise for the ESP32.
        \item \textbf{Voltage Divider}: Used to level-shift 5V sensor signals to 3.3V logic levels.
    \end{itemize}
\end{thesisbox}

\begin{warnbox}[考试陷阱 / Exam Pitfalls]
    \begin{itemize}
        \item \textbf{KCL/KVL Signs}: 符号搞反是最常见的错误。定义好参考方向!
        \item \textbf{Capacitor Impedance}: $Z_C$ 是 $\frac{1}{j\omega C}$,即 $-j \frac{1}{\omega C}$。别忘了那个负号!
        \item \textbf{RMS vs Peak}: 交流功率计算用 RMS ($V_{rms} = V_m / \sqrt{2}$)。
    \end{itemize}
\end{warnbox}
