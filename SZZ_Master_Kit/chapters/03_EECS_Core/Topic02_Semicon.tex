% ============================================
% Topic 02: Semiconductors / 半导体
% Course: BE5B34ELP
% PRIORITY: [SHOW OFF]
% ============================================
\section{Semiconductors / 半导体 (ELP)}

\begin{tipbox}[\textcolor{green}{\textbf{[GOOD]}} PRIORITY: CONFIDENT MODE / 优先级:自信模式]
\textbf{[CN]} ELP 是你的强项。 \textbf{You're comfortable with semiconductors!}
\textbf{[EN]} You're comfortable with semiconductors! TCS34725 in your thesis = PN junction application!\\
\textbf{Strategy}: Attack questions about PN junctions! 主动攻击PN结相关问题!
\end{tipbox}

% --------------------------------------------
% 1. PN Junction
% --------------------------------------------
\begin{studybox}{PN 结 / PN Junction}
    \textbf{[CN] 定义}: PN 结是现代电子学的基石,由 P 型半导体 (空穴为主) 和 N 型半导体 (电子为主) 结合而成。
    \begin{itemize}[leftmargin=*]
        \item \textbf{耗尽层 (Depletion Region)}: 接触面上载流子扩散复合,形成无自由载流子的区域和内建电场。
        \item \textbf{正向偏置 (Forward Bias)}: 外接电压对抗内建电场 ($V > V_{th} \approx 0.7V$),导通电流。
        \item \textbf{反向偏置 (Reverse Bias)}: 外接电压增强内建电场,电流截止。
    \end{itemize}
    \tcblower
    \textbf{[EN] Definition}: A PN junction is formed by joining P-type and N-type semiconductors.
    \begin{itemize}[leftmargin=*]
        \item \textbf{Depletion Region}: A region devoid of free carriers at the interface, creating a built-in electric field.
        \item \textbf{Forward Bias}: Voltage applied against the built-in potential allowing current flow ($V > 0.7V$).
        \item \textbf{Reverse Bias}: Voltage applied supporting the built-in potential, blocking current flow.
    \end{itemize}
\end{studybox}

\begin{formulabox}
\textbf{肖克利方程 / Shockley Equation}:
\begin{equation}
    I = I_S \left( e^{\frac{V}{n V_T}} - 1 \right)
\end{equation}
$V_T \approx 26mV$ (热电压 / Thermal Voltage).
\end{formulabox}

% --------------------------------------------
% 2. Transistors (BJT & MOSFET)
% --------------------------------------------
\begin{studybox}{晶体管 / Transistors}
    \textbf{[CN] 定义}:
    \begin{itemize}[leftmargin=*]
        \item \textbf{BJT (双极性结型晶体管)}: 电流控制电流设备 ($I_C = \beta I_B$)。输入阻抗低,线性性能好。
        \item \textbf{MOSFET (场效应管)}: 电压控制电流设备 ($I_D \propto V_{GS}$)。输入阻抗极高 (栅极绝缘)。ESP32 使用 CMOS (互补 MOS) 工艺。
    \end{itemize}
    \tcblower
    \textbf{[EN] Definition}:
    \begin{itemize}[leftmargin=*]
        \item \textbf{BJT}: Current-controlled device. Low input impedance. Good for analog amplification.
        \item \textbf{MOSFET}: Voltage-controlled device. Very high input impedance (Gate Oxide). Used in digital logic (CMOS).
    \end{itemize}
\end{studybox}

\begin{formulabox}
\textbf{MOSFET 工作区 / Regions of Operation}:
\begin{enumerate}
    \item \textbf{截止 (Cutoff)}: $V_{GS} < V_{th}$ (Switch OFF)
    \item \textbf{线性 (Linear/Ohmic)}: $V_{DS} < V_{GS} - V_{th}$ (Voltage controlled Resistor)
    \item \textbf{饱和 (Saturation)}: $V_{DS} > V_{GS} - V_{th}$ (Constant Current / Amplifier)
\end{enumerate}
\end{formulabox}

% --------------------------------------------
% 3. Thesis Connection
% --------------------------------------------
\begin{thesisbox}[论文关联 / Project Application]
    \textbf{[CN]}: 你的项目使用了多种半导体器件:
    \begin{itemize}
        \item \textbf{光电二极管 (Photodiode)}: TCS34725 颜色传感器内部含有光电二极管阵列。光子撞击 PN 结产生电子-空穴对,产生光电流 (Photocurrent)。
        \item \textbf{LED}: KY-037 模块上的指示灯。正向偏置时,电子与空穴复合释放能量 ($E = h\nu$)。
    \end{itemize}

    \tcblower

    \textbf{[EN]}: Your project utilizes various semiconductor devices:
    \begin{itemize}
        \item \textbf{Photodiode}: Inside the TCS34725. Photons striking the PN junction generate electron-hole pairs, creating a photocurrent proportional to light intensity.
        \item \textbf{Light Emitting Diode (LED)}: On the KY-037 module. Recombination of electrons and holes releases energy as light when forward-biased.
    \end{itemize}
\end{thesisbox}

\begin{warnbox}[考试陷阱 / Exam Pitfalls]
    \begin{itemize}
        \item \textbf{Saturation Region}: MOSFET 的"饱和区"对应于恒流源特性(用于放大),而 BJT 的"饱和区"对应于开关导通(开关应用)。这两个术语在 BJT 和 MOSFET 中是反义的!
        \item \textbf{Threshold Voltage}: $V_{th}$ 是开启 MOSFET 的门槛电压,不要和 $V_T$ (热电压, 26mV) 混淆。
    \end{itemize}
\end{warnbox}
