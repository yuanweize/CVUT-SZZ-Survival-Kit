% ============================================
% Topic 02: Semiconductors / 半导体
% SOURCE: Thesis (TCS34725 Photodiodes, LEDs)
% ============================================
\section{半导体物理 / Semiconductors (BE5B34ELP)}

% --------------------------------------------
% 1. PN Junction
% --------------------------------------------
\begin{studybox}{PN 结 (PN Junction)}

\textbf{Concept}: The fundamental building block of modern electronics. joins P-type (holes) and N-type (electrons) silicon.

\tcblower

\textbf{关键特性}:
\begin{itemize}[leftmargin=*]
    \item \textbf{Depletion Region (耗尽层)}: 界面处无自由载流子,形成内建电场。
    \item \textbf{Forward Bias (正向偏置)}: $V > V_{th}$ (Si $\approx 0.7V$),电流导通。
    \item \textbf{Reverse Bias (反向偏置)}: 电流截止,直到击穿 (Breakdown)。
\end{itemize}

\textbf{公式 (Shockley Equation)}:
\begin{equation}
    I = I_S \left( e^{\frac{V}{n V_T}} - 1 \right)
\end{equation}
Where $V_T \approx 26mV$ at room temperature.

\end{studybox}

% --------------------------------------------
% 2. Transistors (BJT & MOSFET)
% --------------------------------------------
\begin{formulabox}
\textbf{晶体管对比 (BJT vs MOSFET)}:

\begin{center}
\begin{tabular}{lll}
\toprule
\textbf{Feature} & \textbf{BJT (e.g., NPN)} & \textbf{MOSFET (e.g., NMOS)} \\
\midrule
Control & Current ($I_B$) & Voltage ($V_{GS}$) \\
Carriers & Bipolar (Holes + Electrons) & Unipolar (Electrons or Holes) \\
Input Impedance & Low & Very High (Gate oxide) \\
App Usage & Amplifiers (Linear) & Switches (Digital Logic) \\
\bottomrule
\end{tabular}
\end{center}

\textbf{MOSFET 工作区}:
\begin{itemize}
    \item \textbf{Cutoff}: $V_{GS} < V_{th}$ (OFF)
    \item \textbf{Linear (Ohmic)}: $V_{DS} < V_{GS} - V_{th}$ (Resistor)
    \item \textbf{Saturation}: $V_{DS} > V_{GS} - V_{th}$ (Amplifier/Current Source)
\end{itemize}
\end{formulabox}

% --------------------------------------------
% 3. Thesis Connection
% --------------------------------------------
\begin{thesisbox}[Project Application]
你的项目中使用了多种半导体器件:

\textbf{1. 光电二极管 (Photodiodes)}:
\begin{itemize}
    \item \textbf{Device}: TCS34725 Color Sensor (Thesis \texttt{esp32.yaml})
    \item \textbf{Principle}: 光子轰击 PN 结产生电子-空穴对 (Photocurrent $\propto$ Light Intensity)。
\end{itemize}

\textbf{2. 发光二极管 (LED)}:
\begin{itemize}
    \item \textbf{Device}: KY-037 模块上的状态指示灯 (LED1/LED2)。
    \item \textbf{Logic}: 正向偏置时光复合释放能量 ($E_g = h\nu$)。
\end{itemize}

\textbf{3. CMOS 工艺}:
\begin{itemize}
    \item \textbf{Device}: ESP32-S3 Chip (Xtensa LX7)。
    \item \textbf{Tech}: 使用互补 MOS (PMOS + NMOS) 构成逻辑门,静态功耗低。
\end{itemize}
\end{thesisbox}

% --------------------------------------------
% 4. Exam Pitfalls
% --------------------------------------------
\begin{warnbox}
\begin{itemize}
    \item \textbf{温度影响}: 温度升高,本征载流子浓度 $n_i$ 增加,二极管反向漏电流增大。
    \item \textbf{MOSFET 符号}: 注意箭头方向。NMOS 箭头向内 (Source),PMOS 箭头向外。
    \item \textbf{单位}: $V_T = kT/q$ 不是阈值电压,是热电压 (Thermal Voltage)!
\end{itemize}
\end{warnbox}
