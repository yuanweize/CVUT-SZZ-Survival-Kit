% ============================================
% Topic 03: Analog Electronics / 模拟电路
% SOURCE: KY-037 Datasheet (OpAmp + Comparator)
% ============================================
\section{模拟与数字电路 / Analog \& Digital Electronics (BE5B34MIE)}

% --------------------------------------------
% 1. Operational Amplifiers (Op-Amps)
% --------------------------------------------
\begin{studybox}{运算放大器 / Operational Amplifier}
    \textbf{[CN] 定义}: 运放 (Op-Amp) 是具有极高增益的电压放大器。
    \textbf{理想特性}:
    \begin{itemize}[leftmargin=*]
        \item 输入阻抗无穷大 ($R_{in} = \infty$):不从信号源吸取电流。
        \item 输出阻抗为零 ($R_{out} = 0$):理想电压源。
        \item 开环增益无穷大 ($A_{OL} = \infty$)。
    \end{itemize}
    \textbf{虚短与虚断}: 在负反馈下,$V_+ = V_-$ 以及 $I_+ = I_- = 0$。

    \tcblower
    \textbf{[EN] Definition}: An Op-Amp is a high-gain differential voltage amplifier.
    \textbf{Ideal Properties}: Infinite $R_{in}$, Zero $R_{out}$, Infinite Gain.
    \textbf{Golden Rules}:
    \begin{itemize}[leftmargin=*]
        \item \textbf{Virtual Short}: $V_+ = V_-$ (with negative feedback).
        \item \textbf{Virtual Open}: No current enters input terminals ($I = 0$).
    \end{itemize}
\end{studybox}

\begin{formulabox}
\textbf{常见配置 / Common Configurations}:
\begin{itemize}
    \item \textbf{反相放大器 (Inverting)}: $V_{out} = -\frac{R_f}{R_{in}} V_{in}$
    \item \textbf{同相放大器 (Non-Inverting)}: $V_{out} = (1 + \frac{R_f}{R_g}) V_{in}$
    \item \textbf{电压跟随器 (Buffer)}: $V_{out} = V_{in}$ ($R_f = 0, R_g = \infty$)
\end{itemize}
\end{formulabox}

% --------------------------------------------
% 2. Comparators
% --------------------------------------------
\begin{studybox}{比较器 / Comparator}
    \textbf{[CN] 定义}: 比较器利用运放的开环高增益特性,将模拟信号与参考电压进行比较,输出数字电平 (高/低)。没有负反馈。
    \tcblower
    \textbf{[EN] Definition}: A comparator uses an Op-Amp in open-loop mode to compare an analog signal $V_{in}$ against a reference $V_{ref}$, outputting a digital logic level (High/Low).
\end{studybox}

% --------------------------------------------
% 3. ADC
% --------------------------------------------
\begin{studybox}{模数转换器 / AD Converter (ADC)}
    \textbf{[CN] 定义}: 将连续的模拟电压信号转换为离散的数字值。
    \textbf{关键指标}:
    \begin{itemize}[leftmargin=*]
        \item \textbf{分辨率 (Resolution)}: 比特数 (e.g., 12-bit $\to$ 0-4095)。
        \item \textbf{采样率 (Sampling Rate)}: 每秒采样次数 (Samples per second)。
        \item \textbf{混叠 (Aliasing)}: 如果采样率 $< 2 f_{max}$,高频信号会伪装成低频噪音。
    \end{itemize}
    \tcblower
    \textbf{[EN] Definition}: Converts continuous analog signals into discrete digital values.
    \textbf{Key Specs}:
    \begin{itemize}[leftmargin=*]
        \item \textbf{Resolution}: Number of bits (e.g. 12-bit). $LSB = V_{ref} / 2^N$.
        \item \textbf{Sampling Rate}: How often the signal is measured.
        \item \textbf{Nyquist Theorem}: Sampling rate must be $> 2 \times f_{max}$ to avoid Aliasing.
    \end{itemize}
\end{studybox}

% --------------------------------------------
% Thesis Connection
% --------------------------------------------
\begin{thesisbox}[论文关联 / Project Application]
    \textbf{[CN]}: 
    \textbf{KY-037 声音传感器} 包含了完整的模拟电路链条:
    \begin{enumerate}
        \item \textbf{换能器}: 驻极体麦克风将声波转换为微弱电压信号 (mV 级)。
        \item \textbf{放大器}: 下方芯片 (LM393) 作为一个运算放大器,放大这个信号到 \texttt{Analog Out}。
        \item \textbf{比较器}: 另一个运放通道作为比较器,当音量超过电位器设定阈值时,驱动 \texttt{Digital Out} 变高 (并点亮 LED2)。
        \item \textbf{ADC}: ESP32 的 12-bit SAR ADC 读取模拟输出 ($0-3.3V \to 0-4095$)。
    \end{enumerate}

    \tcblower

    \textbf{[EN]}:
    The \textbf{KY-037 Sound Module} demonstrates the analog signal chain:
    \begin{enumerate}
        \item \textbf{Transducer}: Electret mic converts sound to weak voltage.
        \item \textbf{Amplifier}: LM393 amplifies the signal for \texttt{Analog Out}.
        \item \textbf{Comparator}: Another Op-Amp compares signal vs potentiometer voltage, triggering \texttt{Digital Out}.
        \item \textbf{ADC}: ESP32's 12-bit SAR ADC digitizes the analog signal.
    \end{enumerate}
\end{thesisbox}

\begin{warnbox}[考试陷阱 / Exam Pitfalls]
    \begin{itemize}
        \item \textbf{Op-Amp Real vs Ideal}: 实际运放输出电压受电源轨 ($V_{CC}/GND$) 限制 (Rail-to-Rail)。
        \item \textbf{ADC Range}: ESP32 的 ADC 是非线性的,且在 0V 和 3.3V 附近有盲区。
        \item \textbf{Comparator Hysteresis}: 实际比较器通常加入正反馈产生迟滞 (Hysteresis/Schmitt Trigger),以防止在阈值附近由于噪声引起的频繁跳变。
    \end{itemize}
\end{warnbox}
