% ============================================
% Topic 04: Microprocessors / 微处理器
% SOURCE: Thesis_src/esphome/esp32s3.yaml
% VERIFIED: ESP32-S3 + MPU6050
% ============================================
\section{微处理器与嵌入式系统 / Microprocessors \& Embedded Systems}

% --------------------------------------------
% 1. CPU Architecture
% --------------------------------------------
\begin{studybox}{处理器架构 (CPU Architecture)}

\textbf{概念 (CN)}: CPU 的基本组成和工作原理

\textbf{Term (EN)}: Von Neumann, Harvard, ALU, Control Unit, Registers

\tcblower

\textbf{冯诺依曼架构 (Von Neumann)}:
\begin{itemize}[leftmargin=*]
    \item 程序和数据共享同一内存
    \item 通过同一总线访问
    \item 优点: 设计简单
    \item 缺点: 冯诺依曼瓶颈 (Bottleneck)
\end{itemize}

\textbf{哈佛架构 (Harvard)}:
\begin{itemize}[leftmargin=*]
    \item 程序和数据使用独立内存
    \item 可同时读取指令和数据
    \item 优点: 更高带宽,常用于 DSP/嵌入式
\end{itemize}

\textbf{Key Insight}: ESP32 使用修改版哈佛架构 (Modified Harvard)。

\end{studybox}

\begin{formulabox}
\textbf{CPU 核心组件}:

\begin{center}
\begin{tabular}{ll}
\toprule
\textbf{组件} & \textbf{功能} \\
\midrule
ALU (算术逻辑单元) & 执行算术和逻辑运算 \\
Control Unit (控制单元) & 指令解码和执行控制 \\
Registers (寄存器) & 快速临时存储 \\
Program Counter (PC) & 下一条指令地址 \\
Stack Pointer (SP) & 栈顶地址 \\
\bottomrule
\end{tabular}
\end{center}

\textbf{指令周期}: Fetch $\to$ Decode $\to$ Execute $\to$ Write-back
\end{formulabox}

% --------------------------------------------
% 2. ESP32-S3 (YOUR Platform)
% SOURCE: Thesis_src/esphome/esp32s3.yaml
% --------------------------------------------
\begin{studybox}{ESP32-S3 平台 (Your Thesis Platform)}

\textbf{概念 (CN)}: 论文使用的实际微控制器

\textbf{Term (EN)}: ESP32-S3, Xtensa LX7, Dual-Core, Wi-Fi, Bluetooth

\tcblower

\textbf{ESP32-S3 规格} (Source: \texttt{esp32s3.yaml}):
\begin{itemize}[leftmargin=*]
    \item \textbf{CPU}: Xtensa LX7 双核 @ 240 MHz
    \item \textbf{内存}: 512 KB SRAM + PSRAM 支持
    \item \textbf{Flash}: 通常 4-16 MB
    \item \textbf{无线}: Wi-Fi 802.11 b/g/n + BLE 5.0
    \item \textbf{特性}: 硬件加密加速器 (AES, SHA, RSA, ECC)
\end{itemize}

\textbf{你的配置} (from \texttt{esp32s3.yaml} L5-18):
\begin{itemize}[leftmargin=*]
    \item Board: \texttt{esp32-s3-devkitc-1}
    \item Framework: ESP-IDF
    \item TLS: ECDSA P-256 + X25519 + TLS 1.3
\end{itemize}

\end{studybox}

% --------------------------------------------
% 3. I2C Protocol (Sensor Communication)
% SOURCE: Thesis_src/esphome/esp32s3.yaml L88-93
% --------------------------------------------
\begin{studybox}{I2C 通信协议 (I2C Protocol)}

\textbf{概念 (CN)}: 两线制同步串行通信

\textbf{Term (EN)}: I2C, SDA, SCL, Master, Slave, Address

\tcblower

\textbf{基本结构}:
\begin{itemize}[leftmargin=*]
    \item \textbf{SDA}: Serial Data (数据线)
    \item \textbf{SCL}: Serial Clock (时钟线)
    \item 主从模式: Master 发起通信,Slave 响应
    \item 7位地址: 最多 128 个设备
\end{itemize}

\textbf{你的配置} (Source: \texttt{esp32s3.yaml} L88-93):
\begin{lstlisting}[language=yaml, basicstyle=\ttfamily\small]
i2c:
  sda: GPIO5
  scl: GPIO4
  scan: true
  id: bus_a
  frequency: 100kHz
\end{lstlisting}

\textbf{设备地址}:
\begin{itemize}
    \item MPU6050: \texttt{0x68}
    \item BMP180: \texttt{0x77}
\end{itemize}

\end{studybox}

\begin{formulabox}
\textbf{I2C 通信时序}:

\begin{enumerate}
    \item \textbf{START}: SDA $\downarrow$ while SCL = HIGH
    \item \textbf{Address}: 7-bit 设备地址 + R/W bit
    \item \textbf{ACK}: Slave 拉低 SDA 表示确认
    \item \textbf{Data}: 8-bit 数据传输
    \item \textbf{STOP}: SDA $\uparrow$ while SCL = HIGH
\end{enumerate}

\textbf{标准速度}:
\begin{itemize}
    \item Standard Mode: 100 kHz (你的配置)
    \item Fast Mode: 400 kHz
    \item Fast Mode Plus: 1 MHz
\end{itemize}
\end{formulabox}

% --------------------------------------------
% 4. Interrupts (Sensor Sampling)
% --------------------------------------------
\begin{studybox}{中断系统 (Interrupt System)}

\textbf{概念 (CN)}: 硬件事件触发 CPU 响应

\textbf{Term (EN)}: Interrupt, ISR, IRQ, Priority, Nesting

\tcblower

\textbf{中断类型}:
\begin{itemize}[leftmargin=*]
    \item \textbf{外部中断}: GPIO 电平变化 (如按钮、传感器)
    \item \textbf{定时器中断}: 周期性定时事件
    \item \textbf{通信中断}: UART/I2C/SPI 数据就绪
    \item \textbf{软件中断}: 程序主动触发
\end{itemize}

\textbf{中断服务例程 (ISR)}:
\begin{itemize}[leftmargin=*]
    \item 必须快速执行
    \item 避免阻塞操作
    \item 使用 volatile 变量通信
\end{itemize}

\textbf{Key Insight}: ESP32 传感器采样使用定时器中断实现周期性读取。

\end{studybox}

% --------------------------------------------
% 5. YOUR Sensor Configuration
% SOURCE: Thesis_src/esphome/esp32s3.yaml L107-155
% --------------------------------------------
\begin{studybox}{你的传感器配置 / Your Sensor Configuration}

\textbf{MPU6050 IMU} (Source: \texttt{esp32s3.yaml} L107-155):

\begin{lstlisting}[language=yaml, basicstyle=\ttfamily\small]
- platform: mpu6050
  i2c_id: bus_a
  address: 0x68
  accel_x:
    name: "Acceleration X"
    filters:
      - offset: -0.877  # Calibration
      - throttle: 0.3s
  gyro_x:
    name: "Angular Velocity X"
    filters:
      - offset: 2.53    # Calibration
      - throttle: 0.3s
  update_interval: 0.1s  # 10 Hz sampling
\end{lstlisting}

\tcblower

\textbf{关键参数}:
\begin{center}
\begin{tabular}{lll}
\toprule
\textbf{传感器} & \textbf{采样率} & \textbf{用途} \\
\midrule
MPU6050 & 10 Hz (0.1s) & 跌倒检测 \\
BMP180 & 0.2 Hz (5s) & 环境温压 \\
KY-037 & 5 Hz (0.2s) & 声音检测 \\
\bottomrule
\end{tabular}
\end{center}

\textbf{校准偏移} (Calibration Offsets):
\begin{itemize}
    \item Acc X: -0.877, Y: -0.145, Z: -0.564 m/s²
    \item Gyro X: +2.53, Y: -2.68, Z: +2.44 °/s
\end{itemize}

\end{studybox}

% --------------------------------------------
% Thesis Connection
% ============================================
\begin{thesisbox}
\textbf{论文实现总结 (VERIFIED from source)}:

\textbf{硬件}:
\begin{itemize}
    \item MCU: ESP32-S3 DevKitC-1 (Xtensa LX7 双核 @ 240 MHz)
    \item IMU: MPU6050 (I2C @ 0x68, 采样 10 Hz)
    \item 环境: BMP180 (温度/气压)
    \item 声音: KY-037 (ADC)
\end{itemize}

\textbf{通信}:
\begin{itemize}
    \item 传感器 $\leftrightarrow$ MCU: I2C (100 kHz)
    \item MCU $\leftrightarrow$ Server: MQTT over TLS 1.3 (mTLS)
    \item 加密: ECDSA P-256 + X25519 (硬件加速)
\end{itemize}

\textbf{算法} (NOTKalman Filter!):
\begin{itemize}
    \item 方法: 双阈值法 (Huynh et al. 2015)
    \item 条件: $\text{SMV} > 2.4G \land \omega > 240°/s$
    \item 性能: 96.3\% 灵敏度, 96.2\% 特异度
\end{itemize}

\textbf{Jan Koller 问题}: ``How does the I2C bus handle multiple sensors?''

\textbf{答案}: I2C 是多主从总线。每个设备有唯一地址(MPU6050: 0x68, BMP180: 0x77)。Master (ESP32) 发送 7-bit 地址选择设备,然后进行读写。我配置了 100 kHz 标准模式,足够处理 10 Hz 传感器采样。
\end{thesisbox}

% --------------------------------------------
% Exam Strategy
% --------------------------------------------
\begin{warnbox}[[!] 考试陷阱 / Exam Pitfalls]
\begin{enumerate}
    \item \textbf{哈佛 vs 冯诺依曼}: ESP32 是修改版哈佛架构(指令/数据缓存分离)
    \item \textbf{I2C 地址}: 7-bit 地址,不是 8-bit!第8位是 R/W 标志
    \item \textbf{中断延迟}: ISR 必须快,不能做 printf 或 delay
    \item \textbf{采样率}: 奈奎斯特定理要求采样率 $\geq$ 2×信号频率
\end{enumerate}
\end{warnbox}
