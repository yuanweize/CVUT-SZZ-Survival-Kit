% ============================================
% Topic 04: Microprocessors / 微处理器
% ============================================
\section{微处理器 / Microprocessors}

% --------------------------------------------
% 1. Von Neumann vs Harvard Architecture
% --------------------------------------------
\begin{studybox}{冯诺依曼 vs 哈佛架构 (Von Neumann vs Harvard)}

\textbf{概念 (CN)}: 计算机体系结构的两种基本模型

\textbf{Term (EN)}: Von Neumann Architecture, Harvard Architecture

\tcblower

\textbf{冯诺依曼架构}:
\begin{itemize}[leftmargin=*]
    \item 程序和数据共享同一存储器和总线
    \item 简单,成本低
    \item 瓶颈:内存带宽有限(冯诺依曼瓶颈)
    \item 例:x86 PC
\end{itemize}

\textbf{哈佛架构}:
\begin{itemize}[leftmargin=*]
    \item 程序和数据分开存储,有独立总线
    \item 可同时取指令和取数据
    \item 例:ARM Cortex-M, ESP32
\end{itemize}

\textbf{Key Insight}: Harvard architecture allows simultaneous instruction fetch and data access = faster execution.

\end{studybox}

% --------------------------------------------
% 2. Interrupts
% --------------------------------------------
\begin{studybox}{中断系统 (Interrupts)}

\textbf{概念 (CN)}: 外部或内部事件触发 CPU 暂停当前任务,转去执行中断服务程序 (ISR)

\textbf{Term (EN)}: Interrupt, ISR (Interrupt Service Routine), IRQ, NVIC

\tcblower

\textbf{中断类型}:
\begin{itemize}[leftmargin=*]
    \item \textbf{外部中断}: GPIO 电平变化、定时器溢出
    \item \textbf{内部中断}: 软件中断、异常(如除零)
    \item \textbf{NMI}: 不可屏蔽中断(最高优先级)
\end{itemize}

\textbf{中断处理流程}:
\begin{enumerate}
    \item 保存现场(寄存器入栈)
    \item 跳转到 ISR
    \item 执行 ISR
    \item 恢复现场(寄存器出栈)
    \item 返回原程序
\end{enumerate}

\textbf{Key Insight}: Interrupts allow real-time response to events without polling.

\end{studybox}

\begin{formulabox}
\textbf{中断延迟}:
\begin{equation}
    T_{latency} = T_{recognition} + T_{context\_save} + T_{ISR\_entry}
\end{equation}

\textbf{ESP32 中断优先级}: 1-7 级(7 最高,NMI)

\textbf{关键寄存器}:
\begin{itemize}[leftmargin=*]
    \item \texttt{IE} (Interrupt Enable): 中断使能
    \item \texttt{IF} (Interrupt Flag): 中断标志
    \item \texttt{NVIC\_IPR}: 优先级寄存器
\end{itemize}
\end{formulabox}

% --------------------------------------------
% 3. GPIO and Peripherals
% --------------------------------------------
\begin{studybox}{GPIO 与外设 (GPIO \& Peripherals)}

\textbf{概念 (CN)}: 通用输入输出引脚及其配置

\textbf{Term (EN)}: GPIO, Pull-up, Pull-down, Open-drain, Push-pull

\tcblower

\textbf{GPIO 模式}:
\begin{itemize}[leftmargin=*]
    \item \textbf{输入模式}: 浮空、上拉、下拉
    \item \textbf{输出模式}: 推挽输出、开漏输出
\end{itemize}

\textbf{推挽 vs 开漏}:
\begin{center}
\begin{tabular}{lcc}
\toprule
\textbf{特性} & \textbf{推挽 (Push-Pull)} & \textbf{开漏 (Open-Drain)} \\
\midrule
高电平驱动 & 有 & 无(需外部上拉) \\
低电平驱动 & 有 & 有 \\
线与功能 & 无 & 有 \\
典型应用 & LED驱动 & I2C总线 \\
\bottomrule
\end{tabular}
\end{center}

\textbf{Key Insight}: Open-drain allows multiple devices to share a bus (like I2C).

\end{studybox}

% --------------------------------------------
% 4. Communication Buses
% --------------------------------------------
\begin{studybox}{通信总线 (Communication Buses)}

\textbf{概念 (CN)}: 微处理器与外设通信的标准接口

\textbf{Term (EN)}: I2C, SPI, UART, CAN

\tcblower

\textbf{比较表}:
\begin{center}
\begin{tabular}{lcccc}
\toprule
\textbf{协议} & \textbf{线数} & \textbf{速度} & \textbf{拓扑} & \textbf{典型应用} \\
\midrule
I2C & 2 (SDA, SCL) & 100k-3.4M & 多主多从 & 传感器 \\
SPI & 4 (MOSI, MISO, SCK, CS) & 10M+ & 一主多从 & Flash, LCD \\
UART & 2 (TX, RX) & 115200 & 点对点 & 调试串口 \\
CAN & 2 (CANH, CANL) & 1M & 总线型 & 汽车电子 \\
\bottomrule
\end{tabular}
\end{center}

\textbf{Key Insight}: I2C uses fewer wires but is slower; SPI is faster but needs more pins.

\end{studybox}

\begin{formulabox}
\textbf{I2C 地址格式}:
\begin{equation}
    \text{7-bit address} + \text{R/W bit} = \text{8-bit frame}
\end{equation}

\textbf{SPI 时钟配置}:
\begin{itemize}[leftmargin=*]
    \item \textbf{CPOL}: 时钟极性(空闲时高/低)
    \item \textbf{CPHA}: 时钟相位(第一/二边沿采样)
    \item 四种模式: Mode 0 (CPOL=0, CPHA=0), Mode 1, Mode 2, Mode 3
\end{itemize}

\textbf{UART 波特率}:
\begin{equation}
    \text{Baud Rate} = \frac{f_{clk}}{16 \times \text{Divisor}}
\end{equation}
\end{formulabox}

% --------------------------------------------
% 5. Memory Types
% --------------------------------------------
\begin{studybox}{存储器类型 (Memory Types)}

\textbf{概念 (CN)}: 不同类型存储器的特性

\textbf{Term (EN)}: SRAM, DRAM, Flash, EEPROM, ROM

\tcblower

\begin{center}
\begin{tabular}{lccc}
\toprule
\textbf{类型} & \textbf{易失性} & \textbf{可写} & \textbf{典型用途} \\
\midrule
SRAM & 易失 & 读写 & Cache, 寄存器文件 \\
DRAM & 易失 & 读写 & 主内存 \\
Flash & 非易失 & 读/块写 & 程序存储, SSD \\
EEPROM & 非易失 & 读/字节写 & 配置存储 \\
ROM & 非易失 & 只读 & Bootloader \\
\bottomrule
\end{tabular}
\end{center}

\textbf{Key Insight}: ESP32 uses Flash for program storage and SRAM for runtime data.

\end{studybox}

% --------------------------------------------
% Thesis Connection
% --------------------------------------------
\begin{thesisbox}
\textbf{ESP32-S3 与微处理器}:

你的论文使用 \textbf{ESP32-S3},它是一个双核 Xtensa LX7 处理器。

\textbf{架构}: 修改版哈佛架构(指令和数据分开,但共享部分总线)

\textbf{通信总线使用}:
\begin{itemize}
    \item \textbf{I2C}: SHT40 温湿度传感器, MPU-6050 加速度计
    \item \textbf{SPI}: 未使用(无外部 Flash)
    \item \textbf{UART}: 调试串口
\end{itemize}

\textbf{中断应用}:
\begin{itemize}
    \item GPIO 中断: PIR 运动传感器触发
    \item 定时器中断: 定期采集传感器数据
\end{itemize}

\textbf{Jan Koller 问题}: "How does ESP32 handle multiple sensors on I2C?"

\textbf{答案}: 每个 I2C 设备有唯一的 7-bit 地址。Master (ESP32) 先发送地址,只有匹配的 Slave 响应。
\end{thesisbox}

% --------------------------------------------
% Exam Strategy
% --------------------------------------------
\begin{warnbox}[🔴 考试陷阱 / Exam Pitfalls]
\begin{enumerate}
    \item \textbf{I2C 地址冲突}: 同一地址的两个设备不能挂在同一总线上。
    \item \textbf{中断嵌套}: 高优先级中断可以打断低优先级 ISR。
    \item \textbf{GPIO 浮空}: 未配置上拉/下拉的输入引脚读取值不确定。
\end{enumerate}
\end{warnbox}
