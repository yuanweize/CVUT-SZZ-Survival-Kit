% ============================================
% Topic 04: Microprocessors / 微处理器
% SOURCE: Thesis Source (ESP32-S3)
% ============================================
\section{微处理器 / Microprocessors (BE5B34MIE)}

% --------------------------------------------
% 1. Architecture
% --------------------------------------------
\begin{studybox}{处理器架构 / CPU Architecture}
    \textbf{[CN] 定义}: 
    \begin{itemize}[leftmargin=*]
        \item \textbf{冯诺依曼 (Von Neumann)}: 指令和数据共享同一内存和总线。瓶颈在于无法同时读写。(如传统 x86)。
        \item \textbf{哈佛 (Harvard)}: 指令和数据拥有独立的物理内存和总线。可以同时读取指令和访问数据。(如 ESP32, AVR)。
    \end{itemize}
    \tcblower
    \textbf{[EN] Definition}: 
    \begin{itemize}[leftmargin=*]
        \item \textbf{Von Neumann}: Shared memory and bus for both instructions and data. Suffers from the "Von Neumann Bottleneck".
        \item \textbf{Harvard}: Separate memory and buses for instructions and data. Allows simultaneous instruction fetching and data access (Common in embedded systems like ESP32).
    \end{itemize}
\end{studybox}

\begin{formulabox}
\textbf{CPU 主要组件 / Core Components}:
\begin{itemize}
    \item \textbf{ALU}: Arithmetic Logic Unit (算术逻辑单元) - 执行加减乘除和逻辑运算。
    \item \textbf{Registers}: 寄存器 - CPU 内部的最快存储单元 (PC, SP, Status Reg)。
    \item \textbf{CU}: Control Unit (控制单元) - 解码指令并指挥数据流。
\end{itemize}
\end{formulabox}

% --------------------------------------------
% 2. Interrupts
% --------------------------------------------
\begin{studybox}{中断系统 / Interrupt System}
    \textbf{[CN] 定义}: 中断 (Interrupt) 是一种允许硬件信号打断 CPU 当前执行流,转而去执行特定处理程序 (ISR) 的机制。这比轮询 (Polling) 更高效,因为 CPU 不需要不断检查状态。
    \tcblower
    \textbf{[EN] Definition}: An Interrupt is a mechanism where a hardware signal suspends the CPU's current activity to execute an Interrupt Service Routine (ISR). It is more efficient than Polling because the CPU reacts only when an event occurs.
\end{studybox}

% --------------------------------------------
% 3. Peripherals (I2C)
% --------------------------------------------
\begin{studybox}{I2C 通信协议 / I2C Protocol}
    \textbf{[CN] 定义}: I2C (Inter-Integrated Circuit) 是一种同步、半双工、多主从架构的通信协议。
    它只需要两根线:\textbf{SDA} (数据) 和 \textbf{SCL} (时钟)。设备通过 7位或10位地址区分。
    \tcblower
    \textbf{[EN] Definition}: I2C is a synchronous, half-duplex, multi-master/slave communication protocol. It uses two lines: \textbf{SDA} (Serial Data) and \textbf{SCL} (Serial Clock). Devices are addressed via 7-bit or 10-bit unique IDs.
\end{studybox}

% --------------------------------------------
% Thesis Connection
% --------------------------------------------
\begin{thesisbox}[论文关联 / Project Application]
    \textbf{[CN]}: 你的智能家居项目是嵌入式系统的典型应用:
    \begin{itemize}
        \item \textbf{MCU}: 使用了 ESP32-S3 (Xtensa LX7 双核)。它结合了高性能计算和低功耗模式。
        \item \textbf{I2C 总线}:  IMU 传感器 (MPU6050, 地址 0x68) 和气压计 (BMP180, 地址 0x77) 挂载在同一个 I2C 总线上 (GPIO 4/5)。
        \item \textbf{中断}: 定时器中断 (Timer Interrupt) 用于保证以精确的 10Hz 频率采样传感器数据,确保跌倒检测算法的输入稳定性。
    \end{itemize}

    \tcblower

    \textbf{[EN]}: Your Smart Home project demonstrates embedded system principles:
    \begin{itemize}
        \item \textbf{MCU}: Utilizes ESP32-S3 (Xtensa LX7 Dual-core), balancing performance and power efficiency.
        \item \textbf{I2C Bus}: Connects sensors like MPU6050 (0x68) and BMP180 (0x77) on shared lines (GPIO 4/5).
        \item \textbf{Interrupts}: A rigid Timer Interrupt ensures precise 10Hz sampling for the fall detection algorithm, avoiding jitter issues.
    \end{itemize}
\end{thesisbox}

\begin{warnbox}[考试陷阱 / Exam Pitfalls]
    \begin{itemize}
        \item \textbf{Interrupts vs Polling}: 为什么用中断?为了节省 CPU 资源并保证实时性 (Real-time response)。
        \item \textbf{I2C Pull-up}: I2C 是开漏输出 (Open-drain),必须需要在 SDA/SCL 上接上拉电阻 (Pull-up Resistors),否则无法输出高电平。
        \item \textbf{Volatile}: 在中断服务程序 (ISR) 和主程序共享变量时,必须声明为 \texttt{volatile},防止编译器优化。
    \end{itemize}
\end{warnbox}
