% ============================================
% Topic 05: Algorithms / 算法
% SOURCE: Thesis Implementation (O(1) Check, Moving Average)
% ============================================
\section{算法与复杂性 / Algorithms \& Complexity (BE5B33ALG)}

% --------------------------------------------
% 1. Complexity
% --------------------------------------------
\begin{studybox}{算法复杂性 / Computational Complexity}
    \textbf{[CN] 定义}: 大 O 符号 ($O(\cdot)$) 描述了算法在最坏情况下的运行时间或空间需求随输入规模 $N$ 增长的趋势。
    \begin{itemize}[leftmargin=*]
        \item $O(1)$: 常数时间 (如数组索引)。
        \item $O(\log N)$: 对数时间 (如二分查找)。
        \item $O(N)$: 线性时间 (如遍历链表)。
        \item $O(N^2)$: 平方时间 (如冒泡排序)。
    \end{itemize}
    \tcblower
    \textbf{[EN] Definition}: Big-O notation describes the worst-case growth rate of an algorithm's time or space requirements relative to input size $N$.
    \begin{itemize}[leftmargin=*]
        \item $O(1)$: Constant Time (e.g., Array Access).
        \item $O(\log N)$: Logarithmic (e.g., Binary Search).
        \item $O(N)$: Linear (e.g., Iterating a list).
        \item $O(N \log N)$: Linearithmic (e.g., QuickSort/MergeSort).
    \end{itemize}
\end{studybox}

% --------------------------------------------
% 2. Sorting
% --------------------------------------------
\begin{studybox}{排序算法 / Sorting Algorithms}
    \textbf{[CN] 定义}:
    \begin{itemize}[leftmargin=*]
        \item \textbf{快速排序 (QuickSort)}: 分治法。平均 $O(N \log N)$,但不稳定。
        \item \textbf{归并排序 (MergeSort)}: 分治法。保证 $O(N \log N)$,稳定,但需要额外空间。
        \item \textbf{堆排序 (HeapSort)}: 利用二叉堆。$O(N \log N)$,原地排序。
    \end{itemize}
    \tcblower
    \textbf{[EN] Definition}:
    \begin{itemize}[leftmargin=*]
        \item \textbf{QuickSort}: Divide and Conquer. Avg $O(N \log N)$. Unstable.
        \item \textbf{MergeSort}: Stable, $O(N \log N)$, requires $O(N)$ extra space.
        \item \textbf{HeapSort}: Uses Binary Heap structure. In-place $O(N \log N)$.
    \end{itemize}
\end{studybox}

% --------------------------------------------
% Thesis Connection: Algorithms
% --------------------------------------------
\begin{thesisbox}[论文关联 / Project Algorithms]
    \textbf{[CN]}: 你的固件中实施了特定的实时算法:
    \begin{itemize}
        \item \textbf{移动平均滤波 (Moving Average)}: 
        公式: $y[n] = y[n-1] + \frac{x[n] - x[n-N]}{N}$ (递归实现)。
        复杂度: $O(1)$ (如果递归) 或 $O(N)$ (如果每次由头求和)。用于平滑传感器噪音。
        
        \item \textbf{跌倒检测 (Fall Detection)}: 
        逻辑: \texttt{if (acc > 2.4G \&\& gyro > 240 deg/s)}.
        复杂度: $O(1)$。这是实时系统的关键,保证了在资源受限的 ESP32 上极低的延迟。
    \end{itemize}

    \tcblower

    \textbf{[EN]}: Your firmware implements real-time algorithms:
    \begin{itemize}
        \item \textbf{Moving Average Filter}: 
        Used to smooth sensor noise. 
        Complexity: $O(1)$ (Recursive implementation) or $O(N)$ (Naive sum).
        
        \item \textbf{Fall Detection}: 
        Logic: Threshold comparison.
        Complexity: $O(1)$ (Constant time). Critical for low latency on the ESP32.
    \end{itemize}
\end{thesisbox}

\begin{formulabox}
\textbf{图搜索算法 / Graph Search}:
\begin{itemize}
    \item \textbf{BFS (广度优先)}: 使用队列 (Queue)。寻找无权图最短路径。
    \item \textbf{DFS (深度优先)}: 使用栈 (Stack) 或递归。用于回溯法。
    \item \textbf{Dijkstra}: 使用优先队列 (Priority Queue)。寻找加权图最短路径。
\end{itemize}
\end{formulabox}

\begin{warnbox}[考试陷阱 / Exam Pitfalls]
    \begin{itemize}
        \item \textbf{Binary Search Requirements}: 二分查找要求数据必须是\textbf{有序}的 (Sorted)。
        \item \textbf{Stack vs Queue}: Stack is LIFO (Last In First Out); Queue is FIFO (First In First Out).
        \item \textbf{Recursion Risk}: 递归虽然代码简洁,但深度过大时会导致栈溢出 (Stack Overflow),在嵌入式中需谨慎。
    \end{itemize}
\end{warnbox}
