% ============================================
% Topic 01: EMT / 电磁场理论
% Course: BE5B17EMT
% STATUS: [SKIP] OUT OF SCOPE (未修课)
% ============================================
\section{Electromagnetic Theory / 电磁场理论 (OUT OF SCOPE)}

\begin{warnbox}[\textbf{[SKIP]} NOT IN YOUR EXAM SCOPE / 不在你的考试范围]
\textbf{[CN]} 你\textbf{没有修过} BE5B17EMT。此专业题不会出现在你的考试中。\textbf{跳过此章节}!\\
\textbf{[EN]} You \textbf{did NOT take} BE5B17EMT. This topic will NOT appear in your exam. \textbf{SKIP this chapter}!\\
\textbf{Your Specialization}: Power Engineering (EN1/EN2) → 去看Topic 22
\end{warnbox}

\textit{\color{gray}(Content below is for reference only / 以下内容仅供参考)}

% --------------------------------------------
% 1. Maxwell Equations
% --------------------------------------------
\begin{studybox}{麦克斯韦方程组 / Maxwell's Equations}
    \textbf{[CN] 定义}: 
    经典电磁学的核心,包含四个方程:
    \begin{enumerate}
        \item \textbf{高斯电场定律}: 电荷产生散开的电场 ($\nabla \cdot E = \rho/\epsilon_0$).
        \item \textbf{高斯磁场定律}: 无磁单极子 ($\nabla \cdot B = 0$).
        \item \textbf{法拉第定律}: 变化的磁场产生电场 ($\nabla \times E = -\partial B/\partial t$).
        \item \textbf{安培定律}: 电流和变化的电场产生磁场 ($\nabla \times B = \mu_0 J + \mu_0\epsilon_0 \partial E/\partial t$).
    \end{enumerate}
    \tcblower
    \textbf{[EN] Definition}: 
    The core of classical electromagnetism:
    \begin{enumerate}
        \item \textbf{Gauss's Law}: Electric charges produce diverging E-fields.
        \item \textbf{Gauss's Law for Magnetism}: No magnetic monopoles.
        \item \textbf{Faraday's Law}: Changing B-field induces E-field.
        \item \textbf{Ampere's Law}: Currents and changing E-fields produce B-fields.
    \end{enumerate}
\end{studybox}

\begin{formulabox}
\textbf{传输线 / Transmission Lines}:
特性阻抗 (Characteristic Impedance):
\begin{equation}
    Z_0 = \sqrt{\frac{L}{C}} \approx 50\Omega
\end{equation}
为了防止反射 (Reflection),负载阻抗必须匹配 $Z_L = Z_0$。
\end{formulabox}

% --------------------------------------------
% Thesis Connection
% --------------------------------------------
\begin{thesisbox}[论文关联 / Project Application]
    \textbf{[CN]}: 
    \begin{itemize}
        \item \textbf{阻抗匹配}: ESP32 的天线输出被设计为 $50\Omega$。PCB 走线宽度必须精确计算以保持此阻抗,防止信号向源端反射。
        \item \textbf{去耦}: 在电源线上放置电容是为了提供高频信号的低阻抗返回路径,减小辐射回路面积。
    \end{itemize}

    \tcblower

    \textbf{[EN]}: 
    \begin{itemize}
        \item \textbf{Impedance Matching}: ESP32 antenna output requires $50\Omega$. PCB traces are dimensioned to match this to prevent signal reflection ($S_{11}$).
        \item \textbf{Decoupling}: Capacitors provide a low-impedance return path for high-frequency currents, reducing the loop area and EMI.
    \end{itemize}
\end{thesisbox}
