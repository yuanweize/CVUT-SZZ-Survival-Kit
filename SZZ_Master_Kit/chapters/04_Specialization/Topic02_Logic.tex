% ============================================
% Topic 02: Logic Systems / 逻辑系统
% Course: BE5B01LGS
% ============================================
\section{逻辑系统 / Logic Systems (BE5B01LGS)}

% --------------------------------------------
% 1. Boolean Algebra
% --------------------------------------------
\begin{studybox}{布尔代数 / Boolean Algebra}
    \textbf{[CN] 定义}: 
    数字逻辑的数学基础。
    \begin{itemize}[leftmargin=*]
        \item \textbf{基本运算}: AND ($A \cdot B$), OR ($A + B$), NOT ($\bar{A}$).
        \item \textbf{德摩根定律}: $\overline{A \cdot B} = \bar{A} + \bar{B}$, $\overline{A + B} = \bar{A} \cdot \bar{B}$.
    \end{itemize}
    \tcblower
    \textbf{[EN] Definition}: 
    Mathematical foundation of digital logic.
    \begin{itemize}[leftmargin=*]
        \item \textbf{Ops}: AND (Conjunction), OR (Disjunction), NOT (Negation).
        \item \textbf{De Morgan's Laws}: NAND equals OR of negations; NOR equals AND of negations.
    \end{itemize}
\end{studybox}

% --------------------------------------------
% 2. Sequential Logic
% --------------------------------------------
\begin{studybox}{时序逻辑 / Sequential Logic}
    \textbf{[CN] 定义}: 
    输出取决于当前输入\textbf{以及}过去输入的电路(有记忆)。
    \begin{itemize}[leftmargin=*]
        \item \textbf{触发器 (Flip-Flop)}: 基本存储单元 (如 D-FF)。在时钟边沿改变状态。
        \item \textbf{寄存器 (Register)}: 并联的 D-FF,用于存储多位数据。
    \end{itemize}
    \tcblower
    \textbf{[EN] Definition}: 
    Circuits where output depends on current inputs \textbf{and} history (Memory).
    \begin{itemize}[leftmargin=*]
        \item \textbf{Flip-Flop}: Basic storage element (e.g., D Flip-Flop). State changes on clock edge.
        \item \textbf{Register}: Parallel D-FFs storing multi-bit data.
    \end{itemize}
\end{studybox}

% --------------------------------------------
% Thesis Connection
% --------------------------------------------
\begin{thesisbox}[论文关联 / Project Application]
    \textbf{[CN]}: 
    \begin{itemize}
        \item \textbf{GPIO 模式}: ESP32 的 GPIO 输入寄存器读取引脚电压。配置为 \texttt{INPUT\_PULLUP} 实际上是在内部连接了一个上拉电阻,形成逻辑 1。
        \item \textbf{I2C 硬件}: ESP32 内部的 I2C 控制器是由大量状态机 (FSM) 和移位寄存器 (Shift Register) 组成的复杂时序逻辑电路。
    \end{itemize}

    \tcblower

    \textbf{[EN]}: 
    \begin{itemize}
        \item \textbf{GPIO}: Configured via registers. \texttt{INPUT\_PULLUP} connects an internal resistor, defaulting logic High.
        \item \textbf{I2C Controller}: Implemented in silicon using Finite State Machines (FSMs) and Shift Registers to serialize data.
    \end{itemize}
\end{thesisbox}
