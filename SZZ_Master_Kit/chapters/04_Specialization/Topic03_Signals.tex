% ============================================
% Topic 03: Signals / 信号与系统
% Course: BE5B31SAS
% ============================================
\section{信号与系统 / Signals \& Systems (BE5B31SAS)}

% --------------------------------------------
% 1. Fourier Transform
% --------------------------------------------
\begin{studybox}{傅里叶变换 / Fourier Transform}
    \textbf{[CN] 定义}: 
    将信号从时域 (Time Domain) 转换到频域 (Frequency Domain)。
    任何周期信号都可以分解为正弦波的叠加。
    \tcblower
    \textbf{[EN] Definition}: 
    Transforms a signal from Time Domain to Frequency Domain.
    States that any periodic function can be expressed as a sum of sines and cosines.
\end{studybox}

\begin{formulabox}
\textbf{采样定理 / Nyquist-Shannon Theorem}:
\begin{equation}
    f_s \ge 2 f_{max}
\end{equation}
采样率必须至少是信号最高频率的两倍,否则会发生混叠 (Aliasing)。
\end{formulabox}

% --------------------------------------------
% Thesis Connection
% --------------------------------------------
\begin{thesisbox}[论文关联 / Project Application]
    \textbf{[CN]}: 
    \begin{itemize}
        \item \textbf{采样率选择}: 你的 MPU6050 采样率为 10Hz。这意味着只能准确捕捉 5Hz 以下的运动(人体正常运动通常 $< 2\text{Hz}$)。
        \item \textbf{滤波}: 你的移动平均滤波器实际上是一个低通滤波器 (Low Pass Filter),切断了高频噪声。
    \end{itemize}

    \tcblower

    \textbf{[EN]}: 
    \begin{itemize}
        \item \textbf{Sampling}: MPU6050 @ 10Hz. Nyquist limit is 5Hz, sufficient for human motion ($<2\text{Hz}$).
        \item \textbf{Filtering}: The Moving Average Filter acts as an FIR Low Pass Filter, attenuating high-frequency noise.
    \end{itemize}
\end{thesisbox}
