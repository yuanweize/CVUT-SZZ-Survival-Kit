\section{Materials (MVE)}

\begin{tcolorbox}[colback=yellow!10!white,colframe=yellow!50!black,title=\textbf{Exam Cheat Sheet / 极速复习}]
\begin{itemize}
    \item \textbf{Classes}: Conductors (Metals), Semiconductors (Si, Ge), Insulators (Dielectrics).
    \item \textbf{Band Theory}: Energy gap ($E_g$). Metals: overlap. Semi: small gap. Insulators: large gap.
    \item \textbf{Thesis Link}: PCB substrate (FR4) is a dielectric. Copper traces are conductors.
\end{itemize}
\end{tcolorbox}

\subsection{Concepts / 核心概念}
\begin{keywords}
\textbf{Resistivity, Permittivity, Permeability, Polarization, Breakdown Voltage}
\end{keywords}

\begin{concept}{Dielectrics / 电介质}
Used in capacitors and insulation.
\begin{itemize}
    \item \textbf{Polarization}: Electric field aligns dipoles.
    \item \textbf{Losses ($\tan \delta$)}: Energy lost as heat in AC fields. Important for high-frequency PCBs!
    \item \textbf{Breakdown}: Max field strength before arcing (e.g., air $\approx 3$ MV/m).
\end{itemize}
\end{concept}

\begin{concept}{Magnetic Materials / 磁性材料}
\begin{itemize}
    \item \textbf{Ferromagnetic}: Iron, Cobalt. High $\mu_r$. Used in Transformer cores.
    \item \textbf{Hysteresis Loop}: Area inside loop = Energy loss per cycle. Soft magnetic materials have narrow loops (good for transformers).
\end{itemize}
\end{concept}

\subsection{Formulas / 公式}
\begin{equation}
    R = \rho \frac{l}{S} \quad \text{(Resistance)}
\end{equation}
\begin{equation}
    \sigma = n e \mu \quad \text{(Drift Conductivity)}
\end{equation}

\subsection{Exam Questions / 常考题型}
\begin{itemize}
    \item "Why do we laminate transformer cores?" (A: To reduce Eddy Currents by increasing resistance in the loop path).
    \item "What distinguishes a semiconductor from an insulator?" (A: Band gap energy. Si $\approx 1.1$eV).
\end{itemize}
