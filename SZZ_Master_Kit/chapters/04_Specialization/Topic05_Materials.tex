% ============================================
% Topic 05: Materials / 材料学
% Course: BE5B13MAT
% ============================================
\section{材料科学 / Materials (BE5B13MAT)}

% --------------------------------------------
% 1. Material Classes
% --------------------------------------------
\begin{studybox}{材料分类 / Classification}
    \textbf{[CN] 定义}: 
    根据导电性能分类 (能带理论):
    \begin{itemize}[leftmargin=*]
        \item \textbf{导体 (Conductor)}: 电子自由移动 (如铜 Cu)。价带与导带重叠。
        \item \textbf{半导体 (Semiconductor)}: 导电性介于导体和绝缘体之间 (如硅 Si)。带隙较小 ($E_g \approx 1.1\text{eV}$)。
        \item \textbf{绝缘体 (Insulator)}: 电子被束缚 (如玻璃)。带隙很大。
    \end{itemize}
    \tcblower
    \textbf{[EN] Definition}: 
    Based on energy bands:
    \begin{itemize}[leftmargin=*]
        \item \textbf{Conductor}: Free electron movement (Copper). Overlapping bands.
        \item \textbf{Semiconductor}: Intermediate conductivity (Silicon). Small Bandgap.
        \item \textbf{Dielectric (Insulator)}: No free charges (Glass, FR4). Large Bandgap.
    \end{itemize}
\end{studybox}

% --------------------------------------------
% Thesis Connection
% --------------------------------------------
\begin{thesisbox}[论文关联 / Project Application]
    \textbf{[CN]}: 
    \begin{itemize}
        \item \textbf{PCB 基板 (FR4)}: 玻璃纤维环氧树脂。它是绝缘体 (介电材料),其介电常数 ($\epsilon_r \approx 4.4$) 影响阻抗匹配。
        \item \textbf{导线 (Copper)}: PCB 走线是铜箔,具有非零电阻。大电流走线需要足够的宽度以减少发热。
        \item \textbf{半导体}: ESP32 芯片本身是基于硅工艺 (CMOS) 制造的。
    \end{itemize}

    \tcblower

    \textbf{[EN]}: 
    \begin{itemize}
        \item \textbf{FR4}: The PCB substrate. It is a dielectric with $\epsilon_r \approx 4.4$, affecting trace impedance.
        \item \textbf{Copper}: Traces have resistance. High-current power traces must be wide to minimize voltage drop and heating.
        \item \textbf{Silicon}: The base material for the ESP32 SoC (CMOS process).
    \end{itemize}
\end{thesisbox}
