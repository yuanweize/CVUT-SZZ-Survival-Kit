% ============================================
% Topic 06: Power Systems / 电力系统
% Course: BE5B14ESP
% ============================================
\section{电力系统 / Power Systems (BE5B14ESP)}

% --------------------------------------------
% 1. Power Concepts
% --------------------------------------------
\begin{studybox}{电力 / Electric Power}
    \textbf{[CN] 定义}: 
    \begin{itemize}[leftmargin=*]
        \item \textbf{有功功率 (Active Power, P)}: 实际做功的功率。单位:瓦特 (W)。
        \item \textbf{无功功率 (Reactive Power, Q)}: 在电感/电容中振荡但未消耗的功率。单位:VAR。
        \item \textbf{视在功率 (Apparent Power, S)}: 总功率。$S = \sqrt{P^2 + Q^2}$。单位:VA。
    \end{itemize}
    \tcblower
    \textbf{[EN] Definition}: 
    \begin{itemize}[leftmargin=*]
        \item \textbf{Active Power (P)}: Real useful power (Watts).
        \item \textbf{Reactive Power (Q)}: Oscillating power stored in L/C fields (VAR).
        \item \textbf{Apparent Power (S)}: Vector sum magnitude (VA).
    \end{itemize}
\end{studybox}

\begin{formulabox}
\textbf{功率因数 / Power Factor}:
\begin{equation}
    \text{PF} = \frac{P}{S} = \cos \phi
\end{equation}
理想情况下 PF = 1。
\end{formulabox}

% --------------------------------------------
% Thesis Connection
% --------------------------------------------
\begin{thesisbox}[论文关联 / Project Application]
    \textbf{[CN]}: 
    \begin{itemize}
        \item \textbf{电池供电}: 我们的设计使用 Li-Po 电池 (3.7V - 4.2V)。
        \item \textbf{电源管理 (LDO)}: AMS1117-3.3 线性稳压器将电池电压降至 3.3V 供 ESP32 使用。效率较低 ($\eta = V_{out}/V_{in}$),多余能量变为热量。
        \item \textbf{功耗}: ESP32 Wi-Fi 发送时峰值电流可达 240mA。必须确保 LDO 和走线能通过此电流。
    \end{itemize}

    \tcblower

    \textbf{[EN]}: 
    \begin{itemize}
        \item \textbf{Battery}: Li-Po cell nominal 3.7V.
        \item \textbf{LDO Regulator}: Steps down voltage to 3.3V. Linear regulators dissipate excess voltage as heat (Low efficiency).
        \item \textbf{Peak Current}: Wi-Fi transmission spikes up to 240mA. Power traces must handle this.
    \end{itemize}
\end{thesisbox}
