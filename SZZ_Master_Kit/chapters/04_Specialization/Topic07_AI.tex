% ============================================
% Topic 07: AI / 人工智能
% Course: BE5B33UI
% ============================================
\section{人工智能 / Artificial Intelligence (BE5B33UI)}

% --------------------------------------------
% 1. Classification
% --------------------------------------------
\begin{studybox}{分类与学习 / Machine Learning}
    \textbf{[CN] 定义}: 
    从数据中学习模式的算法。
    \begin{itemize}[leftmargin=*]
        \item \textbf{监督学习 (Supervised)}: 训练数据有标签 (如:这组数据是"跌倒",那组是"走路")。
        \item \textbf{无监督学习 (Unsupervised)}: 数据无标签,寻找内在结构 (如聚类)。
        \item \textbf{神经网络 (Neural Network)}: 模拟人脑神经元的层级结构。
    \end{itemize}
    \tcblower
    \textbf{[EN] Definition}: 
    Learning patterns from data.
    \begin{itemize}[leftmargin=*]
        \item \textbf{Supervised}: Data is labeled (Input $\to$ Output mapping). Classification/Regression.
        \item \textbf{Unsupervised}: No labels (Clustering).
        \item \textbf{Neural Networks}: Layers of artificial neurons (Perceptrons) with weights and activation functions.
    \end{itemize}
\end{studybox}

% --------------------------------------------
% Thesis Connection
% --------------------------------------------
\begin{thesisbox}[论文关联 / Project Application]
    \textbf{[CN]}: 
    \begin{itemize}
        \item \textbf{TinyML}: 在微控制器 (MCU) 上运行机器学习。
        \item \textbf{跌倒检测}: 目前使用的是基于阈值的规则 (Heuristic)。
        \item \textbf{未来改进}: 可以收集加速度数据训练一个 SVM (支持向量机) 或简单的神经网络分类器,部署在 ESP32 上以区分"跌倒"和"跳跃",减少误报。
    \end{itemize}

    \tcblower

    \textbf{[EN]}: 
    \begin{itemize}
        \item \textbf{TinyML}: Running ML inference on Edge devices (ESP32).
        \item \textbf{Current}: Heuristic Threshold-based detection.
        \item \textbf{Future Work}: Train a Classifier (SVM or Neural Net) on acceleration data to distinguish Fall vs Jump, improving Specificity.
    \end{itemize}
\end{thesisbox}
