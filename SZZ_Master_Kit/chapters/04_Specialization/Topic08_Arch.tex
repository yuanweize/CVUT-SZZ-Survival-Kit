% ============================================
% Topic 08: Architecture / 计算机体系结构
% Course: BE5B35APO
% ============================================
\section{计算机体系结构 / Computer Architecture (BE5B35APO)}

% --------------------------------------------
% 1. Pipeline and Cache
% --------------------------------------------
\begin{studybox}{流水线与缓存 / Pipeline \& Cache}
    \textbf{[CN] 定义}: 
    \begin{itemize}[leftmargin=*]
        \item \textbf{流水线 (Pipelining)}: 将指令执行分解为多个阶段 (取指, 译码, 执行...) 并行处理,提高吞吐量。
        \item \textbf{缓存 (Cache)}: 位于 CPU 和内存之间的小容量高速存储,利用局部性原理 (Locality) 减少内存访问延迟。 L1, L2, L3。
    \end{itemize}
    \tcblower
    \textbf{[EN] Definition}: 
    \begin{itemize}[leftmargin=*]
        \item \textbf{Pipelining}: Splitting instruction processing into stages (Fetch, Decode, Execute, Writeback) to increase throughput (Instructions Per Cycle).
        \item \textbf{Cache}: Fast, small memory hiding DRAM latency. Relies on Spatial and Temporal Locality.
    \end{itemize}
\end{studybox}

% --------------------------------------------
% 2. RISC vs CISC
% --------------------------------------------
\begin{studybox}{指令集架构 / ISA}
    \textbf{[CN] 定义}: 
    \begin{itemize}[leftmargin=*]
        \item \textbf{CISC (复杂指令集)}: 指令复杂,长度可变 (x86)。
        \item \textbf{RISC (精简指令集)}: 指令简单,长度固定,利用流水线 (ARM, RISC-V, Xtensa)。
    \end{itemize}
    \tcblower
    \textbf{[EN] Definition}: 
    \begin{itemize}[leftmargin=*]
        \item \textbf{CISC}: Complex instructions, variable length (Intel x86).
        \item \textbf{RISC}: Reduced instructions, fixed length, optimized for pipelining (ARM, ESP32).
    \end{itemize}
\end{studybox}

% --------------------------------------------
% Thesis Connection
% --------------------------------------------
\begin{thesisbox}[论文关联 / Project Application]
    \textbf{[CN]}: 
    \begin{itemize}
        \item \textbf{ESP32 架构}: 基于 Tensilica Xtensa LX7 核心,是一个典型的 \textbf{RISC} 架构。
        \item \textbf{双核}: 它有两个核心 (PRO\_CPU, APP\_CPU)。FreeRTOS 调度器可以在两个核心上运行任务。
        \item \textbf{DMA}: 直接内存访问控制器允许外设 (如 SPI, I2C) 在不占用 CPU 的情况下传输数据,这对于高频传感器采样至关重要。
    \end{itemize}

    \tcblower

    \textbf{[EN]}: 
    \begin{itemize}
        \item \textbf{Xtensa LX7}: A RISC architecture powering the ESP32-S3.
        \item \textbf{Dual Core}: Symmetric Multiprocessing (SMP). Contains PRO\_CPU and APP\_CPU.
        \item \textbf{DMA}: Direct Memory Access allowing peripherals to transfer data to RAM without CPU intervention, critical for efficient sensor sampling.
    \end{itemize}
\end{thesisbox}
