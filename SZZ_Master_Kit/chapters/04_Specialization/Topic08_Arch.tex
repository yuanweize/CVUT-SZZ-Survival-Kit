% ============================================
% Topic 08: Computer Architecture / 计算机体系结构
% ============================================
\section{计算机体系结构 / Computer Architecture}

% --------------------------------------------
% 1. Von Neumann Architecture
% --------------------------------------------
\begin{studybox}{冯诺依曼架构 (Von Neumann Architecture)}

\textbf{概念 (CN)}: 现代计算机的基础模型

\textbf{Term (EN)}: Stored-Program Computer, CPU, Memory, I/O

\tcblower

\textbf{五大部件}:
\begin{enumerate}
    \item \textbf{运算器 (ALU)}: 执行算术和逻辑运算
    \item \textbf{控制器 (CU)}: 指令解码和执行控制
    \item \textbf{存储器 (Memory)}: 存储程序和数据
    \item \textbf{输入设备 (Input)}: 键盘、传感器
    \item \textbf{输出设备 (Output)}: 显示器、执行器
\end{enumerate}

\textbf{Key Insight}: Programs and data share the same memory = "stored program" concept.

\end{studybox}

\begin{formulabox}
\textbf{指令周期}:
\begin{equation}
    \text{Instruction Cycle} = \text{Fetch} + \text{Decode} + \text{Execute} + \text{Memory} + \text{Write-back}
\end{equation}

\textbf{CPU 性能公式}:
\begin{equation}
    \text{CPU Time} = \text{IC} \times \text{CPI} \times \text{Clock Period}
\end{equation}

其中:
\begin{itemize}[leftmargin=*]
    \item IC = Instruction Count(指令数)
    \item CPI = Cycles Per Instruction(每条指令周期数)
    \item Clock Period = $1 / f_{clk}$
\end{itemize}
\end{formulabox}

% --------------------------------------------
% 2. Pipeline
% --------------------------------------------
\begin{studybox}{流水线 (Pipeline)}

\textbf{概念 (CN)}: 将指令执行分解为多个阶段,并行处理多条指令

\textbf{Term (EN)}: Pipeline, Pipeline Hazards, Stall, Forwarding

\tcblower

\textbf{经典 5 级流水线}:
\begin{enumerate}
    \item \textbf{IF (Instruction Fetch)}: 取指令
    \item \textbf{ID (Instruction Decode)}: 译码 + 读寄存器
    \item \textbf{EX (Execute)}: ALU 执行
    \item \textbf{MEM (Memory Access)}: 访存
    \item \textbf{WB (Write Back)}: 结果写回寄存器
\end{enumerate}

\textbf{Key Insight}: Pipelining increases throughput, not latency. Each instruction still takes 5 cycles, but one completes every cycle.

\end{studybox}

\begin{formulabox}
\textbf{理想吞吐量}:
\begin{equation}
    \text{Throughput} = \frac{1}{\text{Clock Period}} \quad \text{(1 instruction/cycle)}
\end{equation}

\textbf{加速比}:
\begin{equation}
    \text{Speedup} = \frac{n \times k}{n + k - 1} \approx k \quad \text{(当 } n \gg k \text{)}
\end{equation}
其中 $n$ = 指令数, $k$ = 流水线级数
\end{formulabox}

\begin{warnbox}[[!] 流水线冲突 / Pipeline Hazards]
\begin{enumerate}
    \item \textbf{结构冲突 (Structural)}: 硬件资源冲突(如同时访问内存)
        \begin{itemize}
            \item 解决: 增加硬件资源(分离 I-Cache 和 D-Cache)
        \end{itemize}
    \item \textbf{数据冲突 (Data)}: RAW (Read After Write) 依赖
        \begin{itemize}
            \item 解决: Forwarding(数据旁路)、Stall(插入气泡)
        \end{itemize}
    \item \textbf{控制冲突 (Control)}: 分支指令导致的冲突
        \begin{itemize}
            \item 解决: 分支预测 (Branch Prediction)、延迟槽
        \end{itemize}
\end{enumerate}
\end{warnbox}

% --------------------------------------------
% 3. Cache Memory
% --------------------------------------------
\begin{studybox}{高速缓存 (Cache Memory)}

\textbf{概念 (CN)}: 位于 CPU 和主存之间的小容量高速存储器

\textbf{Term (EN)}: Cache, Hit Rate, Miss Rate, Locality

\tcblower

\textbf{局部性原理}:
\begin{itemize}[leftmargin=*]
    \item \textbf{时间局部性}: 刚访问的数据可能很快再次访问
    \item \textbf{空间局部性}: 访问某地址后,邻近地址也可能被访问
\end{itemize}

\textbf{Cache 层次}:
\begin{itemize}[leftmargin=*]
    \item L1 Cache: 最快,最小(32-64KB),CPU 内部
    \item L2 Cache: 中等(256KB-1MB)
    \item L3 Cache: 最大,多核共享(2-32MB)
\end{itemize}

\textbf{Key Insight}: Cache exploits locality to bridge the CPU-Memory speed gap.

\end{studybox}

\begin{formulabox}
\textbf{平均访存时间 (AMAT)}:
\begin{equation}
    \text{AMAT} = \text{Hit Time} + \text{Miss Rate} \times \text{Miss Penalty}
\end{equation}

\textbf{命中率}:
\begin{equation}
    \text{Hit Rate} = \frac{\text{Hits}}{\text{Hits} + \text{Misses}}
\end{equation}

\textbf{Cache 映射方式}:
\begin{itemize}[leftmargin=*]
    \item \textbf{直接映射}: 每个主存块只能映射到一个 Cache 行
    \item \textbf{全相联}: 每个主存块可以映射到任意 Cache 行
    \item \textbf{组相联}: 折中方案(n-way set associative)
\end{itemize}
\end{formulabox}

% --------------------------------------------
% 4. Branch Prediction
% --------------------------------------------
\begin{studybox}{分支预测 (Branch Prediction)}

\textbf{概念 (CN)}: 在分支指令结果确定前预测其方向

\textbf{Term (EN)}: Static Prediction, Dynamic Prediction, BTB

\tcblower

\textbf{预测策略}:
\begin{itemize}[leftmargin=*]
    \item \textbf{静态预测}: 总是预测跳转/不跳转
    \item \textbf{1-bit 预测器}: 上次跳转则预测跳转
    \item \textbf{2-bit 预测器}: 四状态机(强跳转、弱跳转、弱不跳转、强不跳转)
\end{itemize}

\textbf{Key Insight}: Modern CPUs achieve >95\% branch prediction accuracy.

\end{studybox}

\begin{formulabox}
\textbf{2-bit 饱和计数器状态转换}:
\begin{center}
\begin{tabular}{cc}
    状态 00 (强不跳转) $\xrightarrow{\text{taken}}$ 状态 01 \\
    状态 01 (弱不跳转) $\xrightarrow{\text{taken}}$ 状态 10 \\
    状态 10 (弱跳转) $\xrightarrow{\text{not taken}}$ 状态 01 \\
    状态 11 (强跳转) $\xrightarrow{\text{not taken}}$ 状态 10 \\
\end{tabular}
\end{center}

\textbf{分支惩罚}:
\begin{equation}
    \text{Penalty} = \text{Pipeline Depth} \times (1 - \text{Prediction Accuracy})
\end{equation}
\end{formulabox}

% --------------------------------------------
% 5. RISC vs CISC
% --------------------------------------------
\begin{studybox}{RISC vs CISC}

\textbf{概念 (CN)}: 两种指令集设计哲学

\textbf{Term (EN)}: Reduced Instruction Set Computer, Complex Instruction Set Computer

\tcblower

\begin{center}
\begin{tabular}{lcc}
\toprule
\textbf{特性} & \textbf{RISC} & \textbf{CISC} \\
\midrule
指令数量 & 少(100+) & 多(1000+) \\
指令长度 & 固定 & 可变 \\
寻址模式 & 简单 & 复杂 \\
寄存器数 & 多(32+) & 少 \\
流水线 & 易实现 & 难 \\
例子 & ARM, RISC-V, MIPS & x86, x64 \\
\bottomrule
\end{tabular}
\end{center}

\textbf{Key Insight}: RISC simplifies hardware, enabling faster clocks and better pipelining.

\end{studybox}

% --------------------------------------------
% Thesis Connection
% --------------------------------------------
\begin{thesisbox}
\textbf{ESP32-S3 体系结构}:

你的论文使用 \textbf{ESP32-S3},基于 Xtensa LX7 架构(类 RISC)。

\textbf{架构特点}:
\begin{itemize}
    \item \textbf{双核}: 两个 LX7 核心,可并行处理
    \item \textbf{Harvard}: 分离的指令和数据总线
    \item \textbf{Cache}: 16KB I-Cache, 16KB D-Cache
\end{itemize}

\textbf{Jan Koller 问题}: "Is ESP32 RISC or CISC?"

\textbf{答案}: Xtensa 是一种可配置的 RISC 架构。它有固定长度指令(24-bit)、大量寄存器(64个)、简单寻址模式。
\end{thesisbox}

% --------------------------------------------
% Exam Strategy
% --------------------------------------------
\begin{warnbox}[[!] 考试陷阱 / Exam Pitfalls]
\begin{enumerate}
    \item \textbf{流水线加速比}: 不是无限的!受限于冲突和依赖。
    \item \textbf{Cache 映射}: 直接映射冲突最多,全相联冲突最少但硬件复杂。
    \item \textbf{分支预测}: 2-bit 比 1-bit 更稳定(需要连续两次错误才改变预测)。
\end{enumerate}
\end{warnbox}
