% ============================================
% Topic 09: Control Systems / 控制系统
% Course: BE5B35ARI
% STATUS: [SKIP] OUT OF SCOPE (未修课)
% ============================================
\section{Control Systems / 控制系统 (OUT OF SCOPE)}

\begin{warnbox}[\textbf{[SKIP]} NOT IN YOUR EXAM SCOPE / 不在你的考试范围]
\textbf{[CN]} 你\textbf{没有修过} BE5B35ARI。此专业题不会出现在你的考试中。\textbf{跳过此章节}!\\
\textbf{[EN]} You \textbf{did NOT take} BE5B35ARI. This topic will NOT appear in your exam. \textbf{SKIP this chapter}!\\
\textbf{Your Specialization}: Power Engineering (EN1/EN2) → 去看Topic 22
\end{warnbox}

\textit{\color{gray}(Content below is for reference only / 以下内容仅供参考)}

% --------------------------------------------
% 1. Open Loop vs Closed Loop
% --------------------------------------------
\begin{studybox}{控制结构 / Control Structures}
    \textbf{[CN] 定义}: 
    \begin{itemize}[leftmargin=*]
        \item \textbf{开环控制 (Open-Loop)}: 控制动作独立于系统输出。无法纠正干扰带来的误差。(例如:普通烤面包机)。
        \item \textbf{闭环控制 (Closed-Loop)}: 使用反馈 (Feedback) 将输出与期望值比较,根据误差调整控制动作。(例如:空调恒温)。
    \end{itemize}
    \tcblower
    \textbf{[EN] Definition}:
    \begin{itemize}[leftmargin=*]
        \item \textbf{Open-Loop}: Control action is independent of the process output. Cannot compensate for disturbances (e.g., toaster).
        \item \textbf{Closed-Loop}: Uses Feedback to compare output with reference. Adjusts control action based on error (e.g., thermostat).
    \end{itemize}
\end{studybox}

\begin{formulabox}
\textbf{闭环传递函数 / Closed-Loop Transfer Function}:
\begin{equation}
    T(s) = \frac{Y(s)}{R(s)} = \frac{G(s)}{1 + G(s)H(s)}
\end{equation}
$G(s)$: Plant (对象), $H(s)$: Sensor (传感器), $R(s)$: Reference (参考).
\end{formulabox}

% --------------------------------------------
% 2. PID Controller
% --------------------------------------------
\begin{studybox}{PID 控制器 / PID Controller}
    \textbf{[CN] 定义}: 工业中最常用的反馈控制器。包含三个项:
    \begin{itemize}[leftmargin=*]
        \item \textbf{比例 (P)}: 当前误差。响应快,但这也是主要的控制力。
        \item \textbf{积分 (I)}: 过去误差的累积。用于消除稳态误差 (Steady-state error)。
        \item \textbf{微分 (D)}: 误差变化率。预测未来趋势,增加阻尼,减少超调 (Overshoot)。
    \end{itemize}
    \tcblower
    \textbf{[EN] Definition}: The most common industrial controller.
    \begin{itemize}[leftmargin=*]
        \item \textbf{Proportional (P)}: Reacts to current error. Main drive.
        \item \textbf{Integral (I)}: Reacts to accumulated past errors. Eliminates steady-state error.
        \item \textbf{Derivative (D)}: Reacts to rate of change. Predicts future error, adds damping, reduces overshoot.
    \end{itemize}
\end{studybox}

\begin{formulabox}
\textbf{PID 公式 / PID Formula}:
\begin{equation}
    u(t) = K_p e(t) + K_i \int e(t) dt + K_d \frac{de(t)}{dt}
\end{equation}
调参口诀: P 决定响应速度,I 消除静差但导致振荡,D 抑制振荡。
\end{formulabox}

% --------------------------------------------
% 3. Stability
% --------------------------------------------
\begin{studybox}{稳定性 / Stability}
    \textbf{[CN] 定义}: 一个系统被称为 BIBO (Bounded-Input Bounded-Output) 稳定,如果任何有界输入都产生有界输出。
    在频域中,线性系统稳定的充要条件是:\textbf{所有闭环极点 (Poles) 都位于复平面的左半部分 (LHP)}。
    \tcblower
    \textbf{[EN] Definition}: A system is BIBO Stable if every bounded input produces a bounded output. In the frequency domain, a linear system is stable if and only if \textbf{all closed-loop poles are located in the Left Half Plane (LHP)}.
\end{studybox}

% --------------------------------------------
% Thesis Connection
% --------------------------------------------
\begin{thesisbox}[论文关联 / Project Application]
    \textbf{[CN]}: 
    虽然你的毕业设计主要关注数据采集,但 Home Assistant 也可以由控制理论解释:
    \begin{itemize}
        \item \textbf{Bang-Bang 控制}: 比如当温度 > 25度时开启风扇,< 24度时关闭。这是一种带有迟滞 (Hysteresis) 的非线性控制。
        \item \textbf{反馈回路}: 传感器 (SHT30) $\to$ 服务器 (HA) $\to$ 智能开关 (Relay) $\to$ 房间温度 $\to$ 传感器。
    \end{itemize}

    \tcblower

    \textbf{[EN]}: 
    Home Assistant automations can be viewed as control loops:
    \begin{itemize}
        \item \textbf{Bang-Bang Control}: Turning a fan ON if Temp > 25, OFF if < 24. This is non-linear control with Hysteresis.
        \item \textbf{Feedback Loop}: Sensor (SHT30) $\to$ Controller (HA) $\to$ Actuator (Relay) $\to$ Plant (Room Temp) $\to$ Sensor.
    \end{itemize}
\end{thesisbox}

\begin{warnbox}[考试陷阱 / Exam Pitfalls]
    \begin{itemize}
        \item \textbf{Right Half Plane (RHP)}: 只要有一个极点在右半平面,系统就不稳定!(Unstable).
        \item \textbf{Integral Windup}: 积分饱和。如果执行器达到极限,积分项会持续累积导致系统失控。解决方法是“积分分离”或“抗饱和”。
        \item \textbf{Phase Margin}: 相位裕度越小,系统越接近振荡。通常设计目标是 $45^\circ - 60^\circ$。
    \end{itemize}
\end{warnbox}
