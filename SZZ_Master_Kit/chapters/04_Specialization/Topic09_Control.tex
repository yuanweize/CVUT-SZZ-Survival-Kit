% ============================================
% Topic 09: Control Systems / 控制系统
% ============================================
\section{控制系统 / Control Systems}

% --------------------------------------------
% 1. Open-Loop vs Closed-Loop
% --------------------------------------------
\begin{studybox}{开环与闭环控制 (Open-Loop vs Closed-Loop)}

\textbf{概念 (CN)}: 控制系统的两种基本结构

\textbf{Term (EN)}: Open-Loop Control, Closed-Loop (Feedback) Control

\tcblower

\textbf{开环控制 (Open-Loop)}:
\begin{itemize}[leftmargin=*]
    \item 无反馈,输出不影响输入
    \item 简单但无法自动纠错
    \item 例:定时器控制的洗衣机
\end{itemize}

\textbf{闭环控制 (Closed-Loop)}:
\begin{itemize}[leftmargin=*]
    \item 有反馈,输出通过传感器反馈到输入
    \item 可以自动调节误差
    \item 例:恒温器、巡航控制
\end{itemize}

\textbf{Key Insight}: Feedback is the key difference. Closed-loop systems are self-correcting but can become unstable.

\end{studybox}

\begin{formulabox}
\textbf{开环传递函数}:
\begin{equation}
    Y(s) = G(s) \cdot R(s)
\end{equation}

\textbf{闭环传递函数}:
\begin{equation}
    \frac{Y(s)}{R(s)} = \frac{G(s)}{1 + G(s)H(s)}
\end{equation}

其中:
\begin{itemize}[leftmargin=*]
    \item $G(s)$ = 前向通道传递函数
    \item $H(s)$ = 反馈通道传递函数
    \item $R(s)$ = 参考输入
    \item $Y(s)$ = 输出
\end{itemize}
\end{formulabox}

% --------------------------------------------
% 2. PID Controller
% --------------------------------------------
\begin{studybox}{PID 控制器 (PID Controller)}

\textbf{概念 (CN)}: 比例-积分-微分控制器,工业控制中最常用的控制器

\textbf{Term (EN)}: Proportional-Integral-Derivative Controller

\tcblower

\textbf{三个组成部分}:
\begin{itemize}[leftmargin=*]
    \item \textbf{P (比例)}: 对当前误差的响应。$K_p$ 越大,响应越快,但可能振荡。
    \item \textbf{I (积分)}: 对累积误差的响应。消除稳态误差,但增加超调。
    \item \textbf{D (微分)}: 对误差变化率的响应。预测误差趋势,减少超调。
\end{itemize}

\textbf{Key Insight}: P = "现在的误差", I = "过去的误差", D = "未来的误差"。

\end{studybox}

\begin{formulabox}
\textbf{时域表达式}:
\begin{equation}
    u(t) = K_p e(t) + K_i \int_0^t e(\tau) d\tau + K_d \frac{de(t)}{dt}
\end{equation}

\textbf{传递函数形式}:
\begin{equation}
    C(s) = K_p + \frac{K_i}{s} + K_d s = K_p \left(1 + \frac{1}{T_i s} + T_d s\right)
\end{equation}

其中:
\begin{itemize}[leftmargin=*]
    \item $e(t) = r(t) - y(t)$ = 误差信号
    \item $K_p$ = 比例增益
    \item $K_i$ = 积分增益
    \item $K_d$ = 微分增益
    \item $T_i = K_p / K_i$ = 积分时间
    \item $T_d = K_d / K_p$ = 微分时间
\end{itemize}
\end{formulabox}

\begin{thmbox}[PID 调参口诀]
\textbf{Ziegler-Nichols 经验法则}:
\begin{enumerate}
    \item 设 $K_i = 0$, $K_d = 0$
    \item 增大 $K_p$ 直到系统临界振荡,记录临界增益 $K_u$ 和振荡周期 $T_u$
    \item 按下表设定参数:
\end{enumerate}

\begin{center}
\begin{tabular}{lccc}
\toprule
\textbf{控制器} & $K_p$ & $K_i$ & $K_d$ \\
\midrule
P & $0.5 K_u$ & -- & -- \\
PI & $0.45 K_u$ & $1.2 K_p / T_u$ & -- \\
PID & $0.6 K_u$ & $2 K_p / T_u$ & $K_p T_u / 8$ \\
\bottomrule
\end{tabular}
\end{center}
\end{thmbox}

% --------------------------------------------
% 3. Stability Analysis
% --------------------------------------------
\begin{studybox}{稳定性分析 (Stability Analysis)}

\textbf{概念 (CN)}: 判断系统是否稳定(输出是否会发散)

\textbf{Term (EN)}: BIBO Stability, Poles, Routh-Hurwitz Criterion

\tcblower

\textbf{稳定性条件}:
\begin{itemize}[leftmargin=*]
    \item \textbf{BIBO 稳定}: 有界输入产生有界输出
    \item \textbf{极点位置}: 所有闭环极点必须在左半平面 (LHP)
    \item 极点 $\text{Re}(p) < 0$ $\Rightarrow$ 稳定
\end{itemize}

\textbf{Key Insight}: A system is stable if all poles of the closed-loop transfer function have negative real parts.

\end{studybox}

\begin{formulabox}
\textbf{特征方程}:
\begin{equation}
    1 + G(s)H(s) = 0
\end{equation}

\textbf{Routh-Hurwitz 判据}:

对于特征多项式 $a_n s^n + a_{n-1} s^{n-1} + \cdots + a_1 s + a_0 = 0$:
\begin{itemize}[leftmargin=*]
    \item 所有系数 $a_i > 0$(必要条件)
    \item Routh 表第一列所有元素同号 $\Rightarrow$ 稳定
\end{itemize}

\textbf{二阶系统}:
\begin{equation}
    G(s) = \frac{\omega_n^2}{s^2 + 2\zeta\omega_n s + \omega_n^2}
\end{equation}
稳定条件: $\zeta > 0$(阻尼比为正)
\end{formulabox}

% --------------------------------------------
% 4. Frequency Response
% --------------------------------------------
\begin{studybox}{频率响应 (Frequency Response)}

\textbf{概念 (CN)}: 系统对不同频率正弦输入的响应特性

\textbf{Term (EN)}: Bode Plot, Gain Margin, Phase Margin

\tcblower

\textbf{Bode 图}:
\begin{itemize}[leftmargin=*]
    \item \textbf{幅频特性}: $20\log|G(j\omega)|$ vs $\log\omega$
    \item \textbf{相频特性}: $\angle G(j\omega)$ vs $\log\omega$
\end{itemize}

\textbf{稳定裕度}:
\begin{itemize}[leftmargin=*]
    \item \textbf{增益裕度 (GM)}: 相位 = $-180°$ 时的增益距离 0dB 多远
    \item \textbf{相位裕度 (PM)}: 增益 = 0dB 时的相位距离 $-180°$ 多远
\end{itemize}

\textbf{Key Insight}: GM > 0dB and PM > 0° means stable. Larger margins = more robust.

\end{studybox}

\begin{formulabox}
\textbf{一阶系统}:
\begin{equation}
    G(s) = \frac{K}{1 + \tau s}, \quad |G(j\omega)| = \frac{K}{\sqrt{1 + (\omega\tau)^2}}
\end{equation}

\textbf{截止频率} (Corner Frequency):
\begin{equation}
    \omega_c = \frac{1}{\tau}
\end{equation}

\textbf{渐近线}:
\begin{itemize}[leftmargin=*]
    \item $\omega \ll \omega_c$: 0 dB/decade (平坦)
    \item $\omega \gg \omega_c$: -20 dB/decade (下降)
\end{itemize}
\end{formulabox}

% --------------------------------------------
% Thesis Connection
% --------------------------------------------
\begin{thesisbox}
\textbf{智能家居与控制}:

你的论文中的 \textbf{Home Assistant} 实现了简单的反馈控制。

\textbf{闭环控制示例}:
\begin{itemize}
    \item \textbf{传感器}: SGP40 (VOC 传感器) 检测空气质量
    \item \textbf{控制器}: Home Assistant 判断 VOC > 阈值
    \item \textbf{执行器}: 打开换气扇
    \item \textbf{反馈}: 继续监测 VOC 直到下降
\end{itemize}

\textbf{Jan Koller 问题}: "Is your smart home system open-loop or closed-loop?"

\textbf{答案}: 闭环。传感器持续反馈数据,系统根据实时状态调整输出(如自动开灯、调温)。
\end{thesisbox}

% --------------------------------------------
% Exam Strategy
% --------------------------------------------
\begin{warnbox}[[!] 考试陷阱 / Exam Pitfalls]
\begin{enumerate}
    \item \textbf{混淆开环与闭环传递函数}: 闭环 = $\frac{G}{1+GH}$,开环 = $GH$。
    \item \textbf{积分器的稳态误差}: 积分项消除稳态误差,但增加超调。
    \item \textbf{极点位置}: 右半平面极点 = 不稳定!
\end{enumerate}
\end{warnbox}
